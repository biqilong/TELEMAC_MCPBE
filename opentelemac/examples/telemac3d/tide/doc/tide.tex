% case name
\chapter{tide}
%
% - Purpose & Description:
%     These first two parts give reader short details about the test case,
%     the physical phenomena involved, the geometry and specify how the numerical solution will be validated
%
\section{Purpose}
%
This test demonstrates the availability of \telemac{3d} to model the
propagation of tide in a maritime domain by computing tidal
boundary conditions.
%
\section{Description}
%
A coastal area located in the English Channel off the coast of
Brittany (in France) close to the real location of the Paimpol-Bréhat
tidal farm is modelled to simulate the tide and the tidal currents
over this area.
Time and space varying boundary conditions are prescribed over
liquid boundaries.
%
% - Reference:
%     This part gives the reference solution we are comparing to and
%     explicits the analytical solution when available;
%
% bibliography can be here or at the end
%\subsection{Reference}
%
%
\subsection{Reference}
%

%
% - Geometry and Mesh:
%     This part describes the mesh used in the computation
%
%
\subsection{Geometry and Mesh}
%
\subsubsection{Bathymetry}
%
Real bathymetry of the area bought from the SHOM (French Navy
Hydrographic and Oceanographic Service).
\copyright Copyright 2007 SHOM. Produced with the permission of SHOM.
Contract number 67/2007
%
\subsubsection{Geometry}
%
Almost a rectangle with the French coasts on one side
22~km $\times$ 24~km
%
\subsubsection{Mesh}
%
4,385 triangular elements\\
2,386 nodes\\
11 planes regularly spaced on the vertical
%
% - Physical parameters:
%     This part specifies the physical parameters
%
%
\subsection{Physical parameters}
%
Vertical turbulence model: mixing length model\\
Horizontal viscosity for velocity: $10^{-4}~\rm{m}^2$/s\\
Coriolis: yes (constant coefficient over the domain
= 1.10 $\times$ 10$^{-4}$~rad/s)\\
No wind, no atmospheric pressure, no surge and nor waves
%
% Experimental results (if needed)
%\subsection{Experimental results}
%
% bibliography can be here or at the end
%\subsection{Reference}
%
% Section for computational options
%\section{Computational options}
%
% - Initial and boundary conditions:
%     This part details both initial and boundary conditions used to simulate the case
%
%
\subsection{Initial and Boundary Conditions}
%
\subsubsection{Initial conditions}
%
Constant elevation\\
No velocity
%
\subsubsection{Boundary conditions}
%
Elevation and horizontal velocity boundary conditions computed by
\telemac{3d} from an harmonic constants database (JMJ from LNH).
%
\subsection{General parameters}
%
Time step: 20~s\\
Simulation duration: 90,000~s = 25~h
%
% - Numerical parameters:
%     This part is used to specify the numerical parameters used
%     (adaptive time step, mass-lumping when necessary...)
%
%
\subsection{Numerical parameters}
%
Non-hydrostatic version\\
Advection for velocities: Characteristics method\\
Thompson method with calculation of characteristics for open boundary
conditions\\
Free Surface Gradient Compatibility = 0.5 (not 0.9) to prevent on
wiggles\\
Tidal flats with correction of Free Surface by elements, treatments
to have $h \ge 0$
%
\subsection{Comments}
%
% - Results:
%     We comment in this part the numerical results against the reference ones,
%     giving understanding keys and making assumptions when necessary.
%
%
\section{Results}
%
Tidal range, sea levels and tidal velocities are well reproduced compared to
data coming from the SHOM or at sea measurements.
%
\section{Conclusion}
%
\telemac{3d} is able to model tide in coastal areas.
%
% Here is an example of how to include the graph generated by validateTELEMAC.py
% They should be in test_case/img
%\begin{figure} [!h]
%\centering
%\includegraphics[scale=0.3]{../img/mygraph.png}
% \caption{mycaption}\label{mylabel}
%\end{figure}
%
% bibliography
%\section{Reference}
