\chapter{Bump}\label{chapter:bump}

\section{Purpose}

This test case presents a flow over a bump on the bed with subcritical condition.
It allows to show that \telemac{2d} is able to correctly reproduce
the hydrodynamic impact of a changing bed slopes, vertical flow
contractions and expansions. Furthermore, It allows to have a good
representation of flows computed in steady and transient flow regimes. \\

The solution produced by \telemac{2d} in a frictionless channel
presenting an idealised bump on the bottom  is compared with
the analytical solution to this problem. In this test case,
all flow regimes i.e. sub-critical, critical and trans-critical are studied.

\section{Description}

\subsection{Analytic solution}

From the strict hyperbolicity of the shallow water system, the flow over the bump can be caracterized by a
criticality condition. Depending on the value of the Froud number 
the flow can either be subcritical (fluvial, $Fr<1$) 
or supercritical (torrential, $Fr>1$).
Let's recall the definition of the Froud number:
\begin{equation}
Fr = \dfrac{|{\bf u}|}{\sqrt{gh}}
\end{equation}
where $|{\bf u}|$ is the norm of the depth averaged velocity and $g$ is the gravitational acceleration. Another useful quatity in the case of a constant discharge $q$ is the critical height defined by:
\begin{equation}
h_c = \left( \dfrac{q}{\sqrt{g}}\right)^{2/3}
\end{equation}
The flow can either be subcritical if $h>h_c$ or supercritical if $h<h_c$.
In the case of a frictionless channel with a constant disharge $q_0$, the Bernoulli relation allows to express the water height in the channel at abscissa $x$ by the relation:
\begin{equation}
\dfrac{q_0^2}{2 g h(x)^2} + h(x) + z_b(x) = cst
\end{equation}
with $z_b$ the bottom elevation. Let's consider a channel of length $L$.
From the Bernoulli relation, one can express the water depth as a solution of third degree equation, which depend on the
the flow criticality:

\begin{itemize}
\item {\bf Sub-critical flow:} 
\begin{equation}
 h(x)^3 + \left( z_b(x) - \dfrac{q_0^2}{2g h(L)^2} - h(L) \right) h(x)^2 + \dfrac{q_0^2}{2g} = 0 \quad \forall x \in [0,L]
\end{equation}
\

\item {\bf Critical flow:}
\begin{equation}
h(x)^3 + \left( z_b(x) - \dfrac{q_0^2}{2g h_c^2} - h_c - z_M \right) h(x)^2 + \dfrac{q_0^2}{2g} = 0 \quad \forall x \in [0,L]  
\end{equation}
with $ z_M = \max_{x \in [0,L]}z_b$.
\

\item {\bf Trans-critical flow:}
\begin{equation}
\begin{cases}
h(x)^3 + \left( z_b(x) - \dfrac{q_0^2}{2g h_c^2} - h_c - z_M \right) h(x)^2 + \dfrac{q_0^2}{2g} = 0 \quad & \text{for  } x < x_{shock}  \\
h(x)^3 + \left( z_b(x) - \dfrac{q_0^2}{2g h(L)^2} - h(L) \right) h(x)^2 + \dfrac{q_0^2}{2g} = 0 \quad &\text{for  } x > x_{shock} \\
q_0^2 \left( \dfrac{1}{h(x_{shock}^-)} - \dfrac{1}{h(x_{shock}^+)} \right) + \dfrac{g}{2} \left( h(x_{shock}^-)^2 -h(x_{shock}^+)^2 \right) = 0
\end{cases}
\end{equation}
\end{itemize}

\subsection{Geometry and mesh}

The geometry dimensions of the channel are 2~m wide and 20.5~m long.
The mesh is regular along channel. It is made up with
quadrangles split into two triangles.
It is composed of 2,620 triangular elements (1,452 nodes)
and the size of triangles ranges between 0.25~m and 0.5~m.
It is horizontal with a 4~m long bump in its middle
(see Figure \ref{t2d:bumpsub:fig:mesh}).

\begin{figure}[!htbp]
 \centering
 \includegraphicsmaybe{[width=\textwidth]}{../img/bumpsub_mesh0.png}
 \caption{Mesh of the channel.}
 \label{t2d:bumpsub:fig:mesh}
\end{figure}

\subsection{Bathymetry}
The maximum elevation of
the bump is 0.2~m (see Figure \ref{t2d:bumpsub:fig:baty}) with
the bottom $z_b$ describes by the following equation :
\begin{equation*}
z_b = \left\{
\begin{array}{rl}
  -0.05(x-10)^2~\text{m}& \text{ if 8 m < x < 12 m} \\
-0.20~\text{m}& \text{ elsewhere}
\end{array}
\right.
\end{equation*}

\begin{figure}[H]
 \centering
 \includegraphicsmaybe{[width=0.9\textwidth]}{../img/bumpsub_bathy.png}
 \caption{Bathymetry in the channel.}
 \label{t2d:bumpsub:fig:baty}
\end{figure}

% \begin{figure}[!htbp]
 % \centering
 % \includegraphicsmaybe{[width=\textwidth]}{../img/bump_bottom_level.png}
 % \caption{Profile of the bump.}
 % \label{t2d:bumpsub:fig:profil}
% \end{figure}

\subsection{Initial and boundary conditions}
The initial conditions are a null velocity and an analytical solution for the water depth.
The boundary conditions are:
\begin{itemize}
\item {\bf Sub-critical flow:} 
\begin{itemize}
\item At the channel entrance, the flow rate is $Q =  8.858893836~\text{m}^3\text{s}^{-1}$
(note that the discharge per unit length $q_0=Q/B$,
$B$ being the channel width).
The flow rate $Q$ is taken so that the $q_0$ value is equal to the value
$\sqrt{gh}$ in entrance. 
\item At the channel outlet, the free surface elevation is $ z_{s} = 1.8~\text{m}$.
Therefore, the depth
$\displaystyle{h = z_{s}-z_b = 2~\text{m}}$ with the free surface elevation $z_{s}$.
\end{itemize}
%\begin{figure}[H]
% \centering
% \includegraphicsmaybe{[width=\textwidth]}{../img/bumpsub_mesh.png}
% \caption{Mesh and boundary conditions of the subcritical case}
% \label{t2d:bumpsub:fig:mesh_bnd_sub}
%\end{figure}

\item {\bf Critical flow:}
\begin{itemize}
\item At the channel entrance, the flow rate is $Q = 0.6~\text{m}^3\text{s}^{-1}$. 
\item At the channel outlet, the free velocity and free water level are
imposed on the liquid boundary.
\end{itemize}
%\begin{figure}[H]
% \centering
% \includegraphicsmaybe{[width=\textwidth]}{../img/bumpcri_mesh.png}
% \caption{Mesh and boundary conditions of the critical case}
% \label{t2d:bumpsub:fig:mesh_bnd_cri}
%\end{figure}

\item {\bf Trans-critical flow:}
\begin{itemize}
\item  At the channel entrance, the flow rate is $Q = 2~\text{m}^3\text{s}^{-1}$.
\item  At the channel outlet, the free surface elevation is
$z_{s} = 0.4~\text{m}$. Therefore, the depth
$\displaystyle{h = z_{s}-z_b = 0.6~\text{m}}$, with the free surface elevation $z_{s}$.
\end{itemize}
%\begin{figure}[H]
% \centering
% \includegraphicsmaybe{[width=\textwidth]}{../img/bumptrans_mesh.png}
% \caption{Mesh and boundary conditions of the transcritical case}
% \label{t2d:bumpsub:fig:mesh_bnd_trans}
%\end{figure}

\end{itemize}


\subsection{Physical parameters}

No friction is taken into account on the bottom and on the lateral walls in the subcritical and critical cases.
For the transcritical case a Strickler formula with friction coefficient
equal to 40~$\text{m}^{1/3}.\text{s}^{-1}$ is imposed.
Note that the horizontal viscosity turbulent is constant and equal to zero for all cases.
However instead of prescribing a zero viscosity, no diffusion step could
have been used (keyword DIFFUSION OF VELOCITY prescribed to NO).

\subsection{Numerical parameters}

To solve advection,
the characteristics scheme is used on velocities (scheme 1) and
the conservative PSI scheme is used for the depth (scheme 5).
In addition, the treatment of linear system is done with a wave equation.
The implicitation coefficients for depth and velocities are equal to 0.6.
\telemac{2d} is run
forward in time until a steady state flow is obtained. The resolution accuracy for the velocity is taken at $10^{-5}$.
Numerical parameters specific to each case are defined as follows:
\begin{itemize}
\item {\bf Sub-critical flow:} 
The time step is $0.01s$ for a period of $10s$.
The triangular elements are linear (P1) for velocities and for water depth.
Note that for numerical resolution, the conjugate gradient 
is used for solving the propagation step. 


\item {\bf Critical flow:}
The time step is $0.03s$ for a period of $30s$. 
The triangular elements are linear (P1) for velocities and for water depth.
Note that for numerical resolution, the conjugate gradientis used for solving the propagation step.

\item {\bf Trans-critical flow:}
The time step is $0.02s$ for a period of $50s$. 
The triangular elements are linear 
(P1, 3 values per element, the corners) for water depth and
quadratic (6 values per element, the corners and the center of the edges)
for velocities.
For numerical resolution, GMRES (Generalized Minimal Residual Method)
is used for solving the propagation step. 

\end{itemize}




\section{Results}

\subsection{Water depth and Froud number}

In this test case, the numerical results are compared with the
analytical solution, when the state flow is steady. Furthermore, the computed
Froude number ($Fr$) is also
compared with the analytical solution.
The solution produced by \telemac{2d} is in close agreement with
the analytical solution as shown on the figure \ref{fig:bumpsub:fig:hfr}.


\begin{figure}[H]
\begin{minipage}[t]{0.5\textwidth}
 \centering
 \includegraphicsmaybe{[width=\textwidth]}{../img/bumpsub_free_surface.png}
\end{minipage}%
\begin{minipage}[t]{0.5\textwidth}
 \centering
 \includegraphicsmaybe{[width=\textwidth]}{../img/bumpsub_froude_number.png}
\end{minipage}
\begin{minipage}[t]{0.5\textwidth}
 \centering
 \includegraphicsmaybe{[width=\textwidth]}{../img/bumpcri_free_surface.png}
\end{minipage}%
\begin{minipage}[t]{0.5\textwidth}
 \centering
 \includegraphicsmaybe{[width=\textwidth]}{../img/bumpcri_froude_number.png}
\end{minipage}
\begin{minipage}[t]{0.5\textwidth}
 \centering
 \includegraphicsmaybe{[width=\textwidth]}{../img/bumptrans_free_surface.png}
\end{minipage}%
\begin{minipage}[t]{0.5\textwidth}
 \centering
 \includegraphicsmaybe{[width=\textwidth]}{../img/bumptrans_froude_number.png}
\end{minipage}
  \caption{Comparison between analytical solution and \telemac{2d}
 solution for sub-critical, critical and trans-critical flows (up to down)}\label{fig:bumpsub:fig:hfr}
\end{figure}

\subsection{Velocity}

Velocity norm is presented on \ref{fig:bumpsub:fig:u} for each case.

\begin{figure}[H]
\begin{minipage}[t]{\textwidth}
 \centering
 \includegraphicsmaybe{[width=\textwidth]}{../img/bumpsub_velocity_vector.png}
\end{minipage}
\begin{minipage}[t]{\textwidth}
 \centering
 \includegraphicsmaybe{[width=\textwidth]}{../img/bumpcri_velocity_vector.png}
\end{minipage}
\begin{minipage}[t]{\textwidth}
 \centering
 \includegraphicsmaybe{[width=\textwidth]}{../img/bumptrans_velocity_vector.png}
\end{minipage}
  \caption{Velocity field of the steady state flow for sub-critical, critical and trans-critical flows (up to down)}\label{fig:bumpsub:fig:u}
\end{figure}

\section{Conclusion}

To conclude, this type of channel flow is driven by advection
and pressure gradient terms. It is adequately reproduced by
\telemac{2d} in sub-critical flow regime.



