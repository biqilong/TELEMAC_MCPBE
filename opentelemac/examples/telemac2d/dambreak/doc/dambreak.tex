\chapter{Dam Break: Ritter and Stoker}
%
% - Purpose & Problem description:
%     These first two parts give reader short details about the test case,
%     the physical phenomena involved and specify how the numerical solution will be validated
%
\section{Purpose}
%
The purpose of this case is to demonstrate that the Telemac-2D solution is able to simulate the propagation of a wave due to a dam break. We compare Telemac-2D with two classical analytic solutions: Ritter's solution and Stoker's solution.

%
\section{Description}
%
\subsection{Geometry and mesh}

We consider a channel on length $L=16m$ and width $l=0.55m$ with a flat bottom and with no friction.

\begin{figure}[H]
   \centering
   \includegraphicsmaybe{*[width=0.8\textwidth, keepaspectratio=true]}{../img/t2d_ritter_mesh.png}
 \caption{Mesh of the channel} \label{fig:rittermesh}
 \end{figure}

\subsection{Initial conditions}

Let's define the abscissa of the dam by $x_d$. The initial condition is given by a 1D Riemann problem:

\subsubsection{Dry case:}

\begin{equation}
\begin{split}h(x) = \begin{cases}
h_l > 0 \quad \text{for} \quad 0 \leq x \leq x_d \\
h_r = 0 \quad \text{for} \quad x_d < x \leq L
\end{cases}\end{split}
\end{equation}
with $h_l \geq h_r$ and zero velocity in the channel.

\subsubsection{Wet case:}

\begin{equation}
\begin{split}h(x) = \begin{cases}
h_l \quad \text{for} \quad 0 \leq x \leq x_d \\
h_r \quad \text{for} \quad x_d < x \leq L
\end{cases}\end{split}
\end{equation}

\begin{figure}[htbp]
\centerline{  \begin{tabular}{cc}
\begin{minipage}{.48\linewidth}
\centerline{
  \includegraphicsmaybe{[width=0.9\textwidth]}{../img/t2d_ritter_initial_elevation.png}}
  \caption{Initial condition for Ritter}
    \label{fig:ritterini}
\end{minipage}
&
\begin{minipage}{.48\linewidth}
\centerline{
\includegraphicsmaybe{[width=0.9\textwidth]}{../img/t2d_stoker_initial_elevation.png}}
\caption{Initial condition for Stoker}
    \label{fig:stokerini}
\end{minipage}
  \end{tabular}}
\end{figure}

\subsection{Analytic solutions}

Given the 2 Riemann problems defined as inital conditions, analytic solution is obtained with characteristics method and is given by:
\subsubsection{Dry case: Ritter's solution}
\begin{equation}
\begin{split}h(x,t) = \begin{cases}
h_l \quad & \text{if} \quad x \leq x_A(t) \\
\dfrac{4}{9g} \left( \sqrt{gh_l} - \dfrac{x-x_0}{2t} \right)^2 \quad & \text{if} \quad x_A(t) \leq x \leq x_B(t) \\
0 \quad & \text{if} \quad x_B(t) \leq x
\end{cases}\end{split}
\end{equation}

\begin{equation}
\begin{split}u(x,t) = \begin{cases}
0 \quad & \text{if} \quad x \leq x_A(t) \\
\dfrac{2}{3} \left( \dfrac{x-x_0}{t} + \sqrt{gh_l} \right) \quad & \text{if} \quad x_A(t) \leq x \leq x_B(t)\\
0 \quad & \text{if} \quad x_B(t) \leq x
\end{cases}\end{split}
\end{equation}

\begin{equation}
\begin{split}\begin{cases}
x_A(t) = x_0 - t \sqrt{g h_l} \\
x_B(t) = x_0 + 2t \sqrt{gh_l} \\
\end{cases}\end{split}
\end{equation}

\subsubsection{Wet case: Stoker's solution}

\begin{equation}
\begin{split}h(x,t) = \begin{cases}
h_l \quad & \text{if} \quad x \leq x_A(t) \\
\dfrac{4}{9g} \left( \sqrt{gh_l} - \dfrac{x-x_0}{2t} \right)^2 \quad & \text{if} \quad x_A(t) \leq x \leq x_B(t) \\
\dfrac{c_m^2}{9} \quad & \text{if} \quad x_B(t) \leq x \leq x_C(t) \\
h_r \quad & \text{if} \quad x_C(t) \leq x
\end{cases}\end{split}
\end{equation}

\begin{equation}
\begin{split}u(x,t) = \begin{cases}
0 \quad & \text{if} \quad x \leq x_A(t)\\
\dfrac{2}{3} \left( \dfrac{x-x_0}{t} + \sqrt{gh_l} \right) \quad & \text{if} \quad x_A(t) \leq x \leq x_B(t) \\
2 \left( \sqrt{gh_l} - c_m\right) \quad & \text{if} \quad x_B(t) \leq x \leq x_C(t)\\
0 \quad & \text{if} \quad x_C(t) \leq x
\end{cases}\end{split}
\end{equation}

\begin{equation}
\text{with :} \quad \begin{split}\begin{cases}
x_A(t) = x_0 - t \sqrt{g h_l} \\
x_B(t) = x_0 + t \left( 2 \sqrt{gh_l} - 3c_m \right) \\
x_C(t) = x_0 + t \left( \dfrac{2c_m^2 \left( \sqrt{gh_l} - c_m \right)}{c_m^2-g h_r} \right)
\end{cases}\end{split}
\end{equation}

$c_m$ being solution of $-8g h_r c_m^2 (\sqrt{gh_l}-c_m)^2+(c_m^2-gh_r)^2(c_m^2+gh-r)$.

\subsection{Boundary conditions}
Boundary conditions are defined by no slip walls on the entry and channel sides and torrential outflow on the outlet boundary
with free water depth and velocity.

%
\subsection{Physical parameters}
%
The molecular viscosity is set as constant and equal to 0~m$^2$/s (VELOCITY DIFFUSION=0) and
no friction is set to the bottom.

\subsection{Numerical parameters}

For this test case, duration is set to $2.5s$ for the Stoker case and $1.5s$ for the Ritter case.
Several numerical schemes of advection for velocities are confronted.
The solver used is the conjugate gradient with an accuracy of $10^{-8}$.
The parameters specific to each case are summed up in the Table~\ref{tab:dambreak:cases}.
\begin{table}[H]
  \resizebox{\textwidth}{!}{%
    \begin{tabular}{|c|c|c|c|c|}
      \hline Case & Name & Equations & Advection scheme for velocities & Time-step / Desired Courant number \\
      \hline 1 & CHAR & Saint-Venant FE & Characteristics & 0.0025~s / - \\
      \hline 2 & NERD & Saint-Venant FE & Edge-based n-scheme & 0.0025~s / - \\
      \hline 3 & ERIA & Saint-Venant FE & ERIA scheme & 0.0025~s / - \\
      \hline 4 & HLLC & Saint-Venant FV & HLLC order 1 & - /0.8 \\
      \hline 5 & KIN1 & Saint-Venant FV & Kinetic order 1 & - /0.8 \\
      \hline 6 & KIN2 & Saint-Venant FV & Kinetic order 2 & - /0.8 \\
      \hline
    \end{tabular}
  }
  \caption{List of the simulation parameters used for the six cases tested in the Ritter and Stokes cases.}
  \label{tab:dambreak:cases}
\end{table}
%

%%%%%%%%%%%%%%%%%%%%%%%%%%%%%%%%%%%%%%%%%%%%%%%%%%%%%%%%%%%%%%%%%%%%%%%%%%%%%%%%%%%%%%%%%%
\section{Results for Ritter}


\subsection{Computation time}

Simulation times for each of these cases with sequential and parallel runs (using 4 processors) are shown in the figure ~\ref{fig:ritter:cputime}\footnote{Keep in mind that these times
are specific to the validation run and the type of processors that were used for this purpose.}.

\begin{figure}[H]
  \centering
  \includegraphicsmaybe{[width=0.9\textwidth]}{../img/t2d_ritter_cpu_times.png}
  \caption{CPU times}\label{fig:ritter:cputime}
\end{figure}

\subsection{First observation}

Figure \ref{fig:ritter:firstobs1d} illsutrates the result obtained for water depth and velocity after $0.5s$ of
simulation with the Kinetic finite volume scheme. The rarefaction wave is clearly visible and well
captured as well as the moving wet/dry transition.
On figures \ref{fig:ritter:Hfirstobs1d} and  \ref{fig:ritter:Ufirstobs1d}, water depth and velocity are presented in the 2D plane.

\begin{figure}[H]
\begin{minipage}[t]{0.5\textwidth}
 \centering
 \includegraphicsmaybe{[width=\textwidth]}{../img/t2d_ritter_kin1_depth_firstobs100.png}
\end{minipage}%
\begin{minipage}[t]{0.5\textwidth}
 \centering
 \includegraphicsmaybe{[width=\textwidth]}{../img/t2d_ritter_kin1_vel_firstobs100.png}
\end{minipage}
  \caption{Water depth and velocity with the Kinetic scheme.}
  \label{fig:ritter:firstobs1d}
\end{figure}

\begin{figure}[H]
 \centering
 \includegraphicsmaybe{[width=\textwidth]}{../img/t2d_ritter_kin1_depth2d_firstobs100.png}
  \caption{Water depth in 2D with the Kinetic scheme.}
  \label{fig:ritter:Hfirstobs1d}
\end{figure}

\begin{figure}[H]
 \centering
 \includegraphicsmaybe{[width=\textwidth]}{../img/t2d_ritter_kin1_vel2d_firstobs100.png}
  \caption{Velocity norm in 2D with the Kinetic scheme.}
  \label{fig:ritter:Ufirstobs1d}
\end{figure}

\subsection{Comparison of schemes}

The results obtained with the different numerical schemes are compared to the analytic solution for both
water depth and velocity. Comparisons are shown on figure \ref{fig:ritter:comparison}.

\begin{figure}[H]
\begin{minipage}[t]{0.5\textwidth}
 \centering
 \includegraphicsmaybe{[width=\textwidth]}{../img/t2d_ritter_H_schemes_comparison_50.png}
\end{minipage}%
\begin{minipage}[t]{0.5\textwidth}
 \centering
 \includegraphicsmaybe{[width=\textwidth]}{../img/t2d_ritter_U_schemes_comparison_50.png}
\end{minipage}
\begin{minipage}[t]{0.5\textwidth}
 \centering
 \includegraphicsmaybe{[width=\textwidth]}{../img/t2d_ritter_H_schemes_comparison_100.png}
\end{minipage}%
\begin{minipage}[t]{0.5\textwidth}
 \centering
 \includegraphicsmaybe{[width=\textwidth]}{../img/t2d_ritter_U_schemes_comparison_100.png}
\end{minipage}
\begin{minipage}[t]{0.5\textwidth}
 \centering
 \includegraphicsmaybe{[width=\textwidth]}{../img/t2d_ritter_H_schemes_comparison_150.png}
\end{minipage}%
\begin{minipage}[t]{0.5\textwidth}
 \centering
 \includegraphicsmaybe{[width=\textwidth]}{../img/t2d_ritter_U_schemes_comparison_150.png}
\end{minipage}
  \caption{Comparison of water depth and velocity with analytic solution.}
  \label{fig:ritter:comparison}
\end{figure}

Kinetic and HLLC schemes are able to track the rarefaction wave with a good precision and are capable of
handling wet/dry tyransition smoothly.
NERD and ERIA are also are able to track the rarefaction wave schemes
but exibit oscillations on the wet/dry transition.
Finally characteristics are not capable to reproduce the analytic solution.

\subsection{Accuracy}

For a more quantitative comparison of schemes, the $L^1$, $L^2$ and $L^\infty$ error norms
of the water depth and velocity are calculated at each time step for each scheme.
$L^2$ errors time series and time integrated $L^1$, $L^2$ and $L^\infty$ errors are presented in the figures \ref{fig:ritter:ErrNumH}, \ref{fig:ritter:ErrNumU} and \ref{fig:ritter:ErrNumV} for $H$, $U$ and $V$ respectively.

\begin{figure}[H]
\begin{minipage}[t]{0.45\textwidth}
 \centering
 \includegraphicsmaybe{[width=\textwidth]}{../img/t2d_ritter_error_L2_H.png}
\end{minipage}%
\begin{minipage}[t]{0.55\textwidth}
 \centering
 \includegraphicsmaybe{[width=\textwidth]}{../img/t2d_ritter_errors_timeintegral_H.png}
\end{minipage}
  \caption{Error on H: timeseries (left) and integrated over time (right).}
  \label{fig:ritter:ErrNumH}
\end{figure}

\begin{figure}[H]
\begin{minipage}[t]{0.45\textwidth}
 \centering
 \includegraphicsmaybe{[width=\textwidth]}{../img/t2d_ritter_error_L2_U.png}
\end{minipage}%
\begin{minipage}[t]{0.55\textwidth}
 \centering
 \includegraphicsmaybe{[width=\textwidth]}{../img/t2d_ritter_errors_timeintegral_U.png}
\end{minipage}
  \caption{Error on U: timeseries (left) and integrated over time (right).}
  \label{fig:ritter:ErrNumU}
\end{figure}

\begin{figure}[H]
\begin{minipage}[t]{0.45\textwidth}
 \centering
 \includegraphicsmaybe{[width=\textwidth]}{../img/t2d_ritter_error_L2_V.png}
\end{minipage}%
\begin{minipage}[t]{0.55\textwidth}
 \centering
 \includegraphicsmaybe{[width=\textwidth]}{../img/t2d_ritter_errors_timeintegral_V.png}
\end{minipage}
  \caption{Error on V: timeseries (left) and integrated over time (right).}
  \label{fig:ritter:ErrNumV}
\end{figure}

For water depth, first order kinetic scheme, HLLC, NERD and ERIA exibit similar errors but kinetic and HLLC are more precise on velocity.
Finally the time integrated errors show that Kinetic schemes and HLLC are the most precise schemes, especially the second order kinetic scheme.
This is to be put in perspective with CPU time presented previously \ref{fig:ritter:cputime}.

\subsection{Positivity of the water depth}

The minimum values of the water depth are checked during the whole simulation.
Results are showed on figure \ref{t2d:ritter:minmax}.

\begin{figure}[H]
\centering
\includegraphicsmaybe{[width=0.8\textwidth]}{../img/t2d_ritter_minT.png}
\caption{Minimum values of water depths for the Ritter test case.}
\label{t2d:ritter:minmax}
\end{figure}

No negative values are recorded, which shows that the positivity is fulfilled.
In the case of finite volume schemes, positivity is ensured withouth additionnal treatment.
With finite element schemes the positivity is ensured with a treatment of negative depths
during simulation (TREATMENT OF NEGATIVE DEPTHS = 2 for characteristics and NERD and 3 for ERIA scheme).

\subsection{Mass balance}

Mass conservation can be checked by calculating the mass in the domain during time.
The lost mass is calculated as $M_{initial} - M_{final}$.
The evolution of mass for each of the schemes is shown in the Figure~\ref{fig:ritter:VoLTime}.

\begin{figure}[H]
\centering
  \includegraphicsmaybe{[width=0.75\textwidth]}{../img/t2d_ritter_mass_balance.png}
  \caption{Mass loss for the tested schemes on the Ritter test case.}
\label{fig:ritter:VoLTime}
\end{figure}

\subsection{Energy balance}

In this section, the evolution of the mean energy is checked. In fact, for the Saint-Venant equations,
the quantity $h \frac{||U||^2}{2} + g \frac{h^2}{2}$, called mean or integrated energy, is conserved,
where $||U||$ is the Saint-Venant depth averaged velocity magnitude.
%\begin{equation}
%  ||U||^2 = \left( \left(\frac{1}{h} \int_0^h u~dz\right) \overrightarrow{e_x} \right)^2 + \left( \left(\frac{1}{h} \int_0^h v~dz\right)  \overrightarrow{e_y} \right)^2
%\end{equation}
%Where $u$ and $v$ are the horizontal components of the 3D velocity. \\
The following quantities are studied:
\begin{itemize}
\item Integrated potential energy \textbf{$E_p =\int\int_{\Omega_{xy}}\rho_{water} g \frac{h^2}{2} dxdy$} where $\Omega_{xy}$ is the $2D$ domain of simulation: Figure~\ref{fig:ritter:Ep};
\item Integrated kinetic energy \textbf{$E_c =\int\int_{\Omega_{xy}} \rho_{water} h \frac{||U||^2}{2} dxdy$}: Figure~\ref{fig:ritter:Ec};
\item Integrated mechanical energy \textbf{$E_m = E_p + E_c$}: Figure~\ref{fig:ritter:Em}.
\end{itemize}

\begin{figure}[H]
\centering
  \includegraphicsmaybe{[width=0.75\textwidth]}{../img/t2d_ritter_potential_energy.png}
  \caption{Evolution of kinetic energy for the tested schemes on the Ritter test case.}
\label{fig:ritter:Ep}
\end{figure}

\begin{figure}[H]
\centering
  \includegraphicsmaybe{[width=0.75\textwidth]}{../img/t2d_ritter_kinetic_energy.png}
  \caption{Evolution of kinetic energy for the tested schemes on the Ritter test case.}
\label{fig:ritter:Ec}
\end{figure}

\begin{figure}[H]
\centering
  \includegraphicsmaybe{[width=0.75\textwidth]}{../img/t2d_ritter_total_energy.png}
  \caption{Evolution of mechanical energy for the tested schemes on the Ritter test case.}
\label{fig:ritter:Em}
\end{figure}

%%%%%%%%%%%%%%%%%%%%%%%%%%%%%%%%%%%%%%%%%%%%%%%%%%%%%%%%%%%%%%%%%%%%%%%%%%%%%%%%%%%%%%%%%%
\newpage

\section{Results for Stoker}

\subsection{Computation time}

Simulation times for each of these cases with sequential and parallel runs (using 4 processors) are shown in the figure ~\ref{fig:stoker:cputime}\footnote{Keep in mind that these times
are specific to the validation run and the type of processors that were used for this purpose.}.

\begin{figure}[h!]
  \centering
  \includegraphicsmaybe{[width=0.8\textwidth]}{../img/t2d_stoker_cpu_times.png}
  \caption{CPU times}\label{fig:stoker:cputime}
\end{figure}

\subsection{First observation}

Figure \ref{fig:ritter:firstobs1d} illsutrate the result obtained for water depth and velocity after $0.5s$ of
simulation with the Kinetic finite volume scheme. The left going rarefaction wave and the right going shock are
clearly visible and well captured by the Kinetic scheme.

\begin{figure}[H]
\begin{minipage}[t]{0.5\textwidth}
 \centering
 \includegraphicsmaybe{[width=\textwidth]}{../img/t2d_stoker_kin1_depth_firstobs100.png}
\end{minipage}%
\begin{minipage}[t]{0.5\textwidth}
 \centering
 \includegraphicsmaybe{[width=\textwidth]}{../img/t2d_stoker_kin1_vel_firstobs100.png}
\end{minipage}
  \caption{Water depth and velocity with the Kinetic scheme.}
  \label{fig:stoker:firstobs1d}
\end{figure}

\subsection{Comparison of schemes}

The results obtained with the different numerical schemes are compared to the analytic solution for both
water depth and velocity. Comparisons are shown on figure \ref{fig:stoker:comparison}.

\begin{figure}[H]
\begin{minipage}[t]{0.5\textwidth}
 \centering
 \includegraphicsmaybe{[width=\textwidth]}{../img/t2d_stoker_H_schemes_comparison_50.png}
\end{minipage}%
\begin{minipage}[t]{0.5\textwidth}
 \centering
 \includegraphicsmaybe{[width=\textwidth]}{../img/t2d_stoker_U_schemes_comparison_50.png}
\end{minipage}
\begin{minipage}[t]{0.5\textwidth}
 \centering
 \includegraphicsmaybe{[width=\textwidth]}{../img/t2d_stoker_H_schemes_comparison_100.png}
\end{minipage}%
\begin{minipage}[t]{0.5\textwidth}
 \centering
 \includegraphicsmaybe{[width=\textwidth]}{../img/t2d_stoker_U_schemes_comparison_100.png}
\end{minipage}
\begin{minipage}[t]{0.5\textwidth}
 \centering
 \includegraphicsmaybe{[width=\textwidth]}{../img/t2d_stoker_H_schemes_comparison_150.png}
\end{minipage}%
\begin{minipage}[t]{0.5\textwidth}
 \centering
 \includegraphicsmaybe{[width=\textwidth]}{../img/t2d_stoker_U_schemes_comparison_150.png}
\end{minipage}
  \caption{Comparison of water depth and velocity with analytic solution.}
  \label{fig:stoker:comparison}
\end{figure}

Kinetic and HLLC schemes are able to track the rarefaction wave and the shock with a good precision.
Characteristics, NERD and ERIA schemes exibit numerical oscillations on the shock.
The worst results are obtained with the characteristics which is not capable of reproducing the analytic solution both in terms of water levels and velocity.

\subsection{Accuracy}

For a more quantitative comparison of schemes, the $L^1$, $L^2$ and $L^\infty$ error norms
of the water depth and velocity are calculated at each time step for each scheme.
$L^2$ errors time series and time integrated $L^1$, $L^2$ and $L^\infty$ errors are presented in the figures \ref{fig:stoker:ErrNumH}, \ref{fig:stoker:ErrNumU} and \ref{fig:stoker:ErrNumV} for $H$, $U$ and $V$ respectively.

\begin{figure}[H]
\begin{minipage}[t]{0.45\textwidth}
 \centering
 \includegraphicsmaybe{[width=\textwidth]}{../img/t2d_stoker_error_L2_H.png}
\end{minipage}%
\begin{minipage}[t]{0.55\textwidth}
 \centering
 \includegraphicsmaybe{[width=\textwidth]}{../img/t2d_stoker_errors_timeintegral_H.png}
\end{minipage}
  \caption{Error on H: timeseries (left) and integrated over time (right).}
  \label{fig:stoker:ErrNumH}
\end{figure}

\begin{figure}[H]
\begin{minipage}[t]{0.45\textwidth}
 \centering
 \includegraphicsmaybe{[width=\textwidth]}{../img/t2d_stoker_error_L2_U.png}
\end{minipage}%
\begin{minipage}[t]{0.55\textwidth}
 \centering
 \includegraphicsmaybe{[width=\textwidth]}{../img/t2d_stoker_errors_timeintegral_U.png}
\end{minipage}
  \caption{Error on U: timeseries (left) and integrated over time (right).}
  \label{fig:stoker:ErrNumU}
\end{figure}

\begin{figure}[H]
\begin{minipage}[t]{0.45\textwidth}
 \centering
 \includegraphicsmaybe{[width=\textwidth]}{../img/t2d_stoker_error_L2_V.png}
\end{minipage}%
\begin{minipage}[t]{0.55\textwidth}
 \centering
 \includegraphicsmaybe{[width=\textwidth]}{../img/t2d_stoker_errors_timeintegral_V.png}
\end{minipage}
  \caption{Error on V: timeseries (left) and integrated over time (right).}
  \label{fig:stoker:ErrNumV}
\end{figure}

\subsection{Positivity of the water depth}

The minimum value of the water depth are checked during the whole simulation.
Results are showed on figure \ref{t2d:stoker:minmax}.

\begin{figure}[H]
\centering
\includegraphicsmaybe{[width=0.8\textwidth]}{../img/t2d_stoker_minT.png}
\caption{Minimum values of water depths after one rotation of the cone.}
\label{t2d:stoker:minmax}
\end{figure}

No negative values are recorded, which shows that the positivity is fulfilled.
In the case of finite volume schemes, positivity is ensured withouth additionnal treatment.
With finite element schemes the positivity is ensured with a treatment of negative depths
during simulation (TREATMENT OF NEGATIVE DEPTHS = 2 for characteristics and NERD and 3 for ERIA scheme).

\subsection{Mass balance}

Mass conservation can be checked by calculating the mass in the domain during time.
The lost mass is calculated as $M_{initial} - M_{final}$.
The evolution of mass for each of the schemes is shown in the Figure~\ref{fig:stoker:VoLTime}.

\begin{figure}[H]
\centering
  \includegraphicsmaybe{[width=0.75\textwidth]}{../img/t2d_stoker_mass_balance.png}
  \caption{Mass loss for the tested schemes on the Stoker test case.}
\label{fig:stoker:VoLTime}
\end{figure}

\subsection{Energy balance}

In this section, the evolution of the mean energy is checked. In fact, for the Saint-Venant equations,
the quantity $h \frac{||U||^2}{2} + g \frac{h^2}{2}$, called mean or integrated energy, is conserved,
where $||U||$ is the Saint-Venant depth averaged velocity magnitude.
%\begin{equation}
%  ||U||^2 = \left( \left(\frac{1}{h} \int_0^h u~dz\right) \overrightarrow{e_x} \right)^2 + \left( \left(\frac{1}{h} \int_0^h v~dz\right)  \overrightarrow{e_y} \right)^2
%\end{equation}
%Where $u$ and $v$ are the horizontal components of the 3D velocity. \\
The following quantities are studied:
\begin{itemize}
\item Integrated potential energy \textbf{$E_p =\int\int_{\Omega_{xy}}\rho_{water} g \frac{h^2}{2} dxdy$} where $\Omega_{xy}$ is the $2D$ domain of simulation: Figure~\ref{fig:stoker:Ep};
\item Integrated kinetic energy \textbf{$E_c =\int\int_{\Omega_{xy}} \rho_{water} h \frac{||U||^2}{2} dxdy$}: Figure~\ref{fig:stoker:Ec};
\item Integrated mechanical energy \textbf{$E_m = E_p + E_c$}: Figure~\ref{fig:stoker:Em}.
\end{itemize}

\begin{figure}[H]
\centering
  \includegraphicsmaybe{[width=0.75\textwidth]}{../img/t2d_stoker_potential_energy.png}
  \caption{Evolution of kinetic energy for the tested schemes on the Stoker test case.}
\label{fig:stoker:Ep}
\end{figure}

\begin{figure}[H]
\centering
  \includegraphicsmaybe{[width=0.75\textwidth]}{../img/t2d_stoker_kinetic_energy.png}
  \caption{Evolution of kinetic energy for the tested schemes on the Stoker test case.}
\label{fig:stoker:Ec}
\end{figure}

\begin{figure}[H]
\centering
  \includegraphicsmaybe{[width=0.75\textwidth]}{../img/t2d_stoker_total_energy.png}
  \caption{Evolution of mechanical energy for the tested schemes on the Stoker test case.}
\label{fig:stoker:Em}
\end{figure}

%%%%%%%%%%%%%%%%%%%%%%%%%%%%%%%%%%%%%%%%%%%%%%%%%%%%%%%%%%%%%%%%%%%%%%%%%%%%%%%%%%%%%%%%%%
\section{ Conclusions}

With the tested dambreak cases, we have shown that \telemac{2d} is able to reproduce the propagation
of rarefaction and shock waves in good accordance with analytic solutions.

%%%%%%%%%%%%%%%%%%%%%%%%%%%%%%%%%%%%%%%%%%%%%%%%%%%%%%%%%%%%%%%%%%%%%%%%%%%%%%%%%%%%%%%%%%
\section{Reference}

\begin{itemize}
\item J. J. Stoker, Water Waves: The Mathematical Theory with Applications, Interscience, 1957.
\item Ritter, A. (1892). "Die Fortpflanzung der Wasserwellen." Vereine Deutscher Ingenieure Zeitswchrift, Vol. 36, No. 2, 33,
13 Aug., pp. 947-954 (in German).
\end{itemize}
