\chapter{Wave}

\section{Purpose}
To assess the properties of TELEMAC-2D for the propagation of a long
wave in a rectilinear channel without resistance effects.
%%%%%%%%%%%%%%%%%%%%%%%%%%%%%%%%%%%%%%%%%%%%%%%%%%%%%%%%%%%%%%%%%%%%%%%%%
\section{Description}

\subsection{Analytic solution}
This case corresponds to the analytical solution of the shallow water equations
without variation in the $y$ direction, without diffusion and without advection, and
with the assumption $H_0 >> H$, where $H$ is the water depth and $H_0$ is the mean water depth:
\begin{equation}\label{eq:wave-system}
    \left\{\begin{array}{l}
    \dfrac{\partial H}{\partial t} + H_0 \dfrac{\partial U}{\partial x}=0\medskip\\
    \dfrac{\partial U}{\partial t} = -g \dfrac{\partial H}{\partial x}=0
  \end{array}\right.
\end{equation}
A solution to this problem is:
\begin{equation}
    \left\{\begin{array}{l}
    H = H_0 + A\text{sin}\left(\dfrac{2\pi t}{T} - \dfrac{2\pi x}{T\sqrt{gH_0}}\right)\medskip\\
    U = A\sqrt{\dfrac{g}{H_0}} \text{sin}\left(\dfrac{2\pi t}{T} - \dfrac{2\pi x}{T\sqrt{gH_0}}\right)
  \end{array}\right.
\end{equation}
with $A$ the amplitude of the wave and $T$ its period.

In this example, a 16 metres long and 0.3 metres wide channel with constant depth
of 10 m is considered. At the channel entrance, a sinusoïdal water surface
elevation is imposed, that corresponds to the analytical solution.
No bed resistance occurs and the advection step of TELEMAC-2D is skipped in order
to solve the system \label{eq:wave-system}.
\subsection{Geometry and mesh}
The domain is a channel with a size of $16 m \times 0.3 m$. The bottom is horizontal and
the water depth at rest is equal to 10 m.
The domain is meshed with 3840 triangular elements and 2247 nodes.
Triangles are obtained by dividing rectagular elements on their diagonals.
The mean size of obtained triangles is about 0.07 m.

\begin{figure}[H]
 \centering
 \includegraphicsmaybe{[width=\textwidth]}{../img/Mesh.png}
  \caption{Mesh}\label{fig:wave:mesh}
\end{figure}

\subsection{Initial conditions}
The initial velocity is zero and the water level is horizontal, with a water depth of 10m.
\subsection{Boundary conditions}
The incident water wave is imposed on the channel's entrance as follows :
\begin{equation}
    H = 10 + 0.05 \text{sin}\left( \dfrac{2\pi t}{0.25} \right)
\end{equation}
so that $H_0=10m$, $A=0.05m$, $T=0.25m$.
A free surface elevation of $10m$ is imposed on channel's outlet.
The Thompson scheme is applied for open boundaries.
The lateral boundaries are considered as solid walls with slip condition.
\subsection{Physical parameters}
The physical parameters used for this case are:
\begin{enumerate}
\item No friction (LAW OF BOTTOM FRICTION set to 0)
\item No diffusion
\end{enumerate}
\subsection{Numerical parameters}
\begin{enumerate}
\item Simulation type: propagation without advection
\item Type of element: Linear triangle (P1) for velocities and $H$
\item Solver: GMRES (solver 7) with an accuracy of 10${}^{-6}$
\item Implicitation for depth and for velocity: 0.5
\item Time Step: 0.0025 sec.
\item Simulation duration: 5 sec.
\end{enumerate}

%%%%%%%%%%%%%%%%%%%%%%%%%%%%%%%%%%%%%%%%%%%%%%%%%%%%%%%%%%%%%%%%%%%%%%%%%
\section{Results}
The solution produced by TELEMAC-2D shows very good agreement with
the exact solution (see the figures \label{fig:wave1} and \label{fig:wave2}).
The incident water wave is properly propagated in the channel.

\begin{figure}\label{fig:wave1}
    \includegraphicsmaybe{[width=\textwidth]}{../img/FreeSurface_t1.png}
    \includegraphicsmaybe{[width=\textwidth]}{../img/FreeSurface_t2.png}
    \includegraphicsmaybe{[width=\textwidth]}{../img/FreeSurface_t3.png}
    \includegraphicsmaybe{[width=\textwidth]}{../img/FreeSurface_t4.png}
    \caption{Wave example: shape of the free-surface at the times 1.25, 2.5, 3.75 and 5.0 seconds. The colors correspond span
    from $H = 9.95m$ (blue) to $H = 10.05m$ (red).}
\end{figure}
\begin{figure}\label{fig:wave2}
    \includegraphicsmaybe{[width=\textwidth]}{../img/FreeSurface_Y0_5.png}
    \caption{Wave example: final free-surface profile ($t=5s$).}
\end{figure}

The celerity of the wave is exactly $c = \sqrt{g.H_0}$.
The phase of the wave is correct. The amplitude of the wave is nearly
the same at the channel entrance and at the outlet. A very small
difference between the surface elevation computed by TELEMAC-2D
and the exact solution is observed. The maximum error is lower
than 3\% on the amplitude of the wave.
There is no reflection of the wave on the open boundary at the outlet
of the channel thanks to the Thompson scheme.
%%%%%%%%%%%%%%%%%%%%%%%%%%%%%%%%%%%%%%%%%%%%%%%%%%%%%%%%%%%%%%%%%%%%%%%%%
\section{Conclusions}
TELEMAC accurately reproduces the propagation of surface long waves in terms of celerity,
phase and amplitude.
