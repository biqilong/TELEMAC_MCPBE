\chapter{Cone}
%

% - Purpose & Problem description:
%     These first two parts give reader short details about the test case,
%     the physical phenomena involved and specify how the numerical solution will be validated

\section{Purpose}
This test shows the performance of the finite element advection schemes of \telemac{2d} for passive scalar transport in a time dependent case. 
It shows the advection of a tracer (or any other passive scalar) in a square basin with flat frictionless bottom and with open boundaries. 

%--------------------------------------------------------------------------------------------------
\section{Description}

\subsection{Geometry and mesh}

The dimensions of the domain are $[d \times d]=[20 \times 20]~\rm{m}^2$.
The mesh is made from a regular grid from which all square are cut in half.
The number of elements and points in the mesh are $5020$ and $2618$ respectively.

\begin{figure}[h!]
\centering
\includegraphicsmaybe{[width=0.6\textwidth]}{../img/cone_mesh_bnd.png}
\caption{2D domain and mesh of the cone test case}
\label{t2d:cone:mesh}
\end{figure}

\subsection{Initial condition}

The water depth is constant in time and in space, equal to 2~m. The velocity field is constant in time as well and is divergence free:
\begin{equation*}
  \vec{u}=\left\{
         \begin{array}{l}
          u(x,y)=-(y-y_0) \\
          v(x,y)=(x-x_0)
         \end{array}\right.
\end{equation*}
With $x_0=10$ and $y_0=10$. The initial value for the tracer is given by the Gaussian function
off-centered $5m$ to the right of $(x_0, y_0)$:
\begin{equation*}
c^0(x,y)=e^{-\frac{1}{5}[(x-x_0-5)^2+(y-y_0)^2]}
\end{equation*}

\subsection{Analytic solution}

The tracer is described by a Gaussian function and is submitted to a rotating velocity field. 
After one period we expect that the tracer function has the same position and the same values as the initial condition (i.e. maximum value equal to 1 at the center). 
The analytic solution for the tracer $c$ is given by:
\begin{equation*}
c(x,y,t)=e^{-\frac{1}{5}[X^2+Y^2]}
\end{equation*}
with: 
\begin{equation*}
\left\{
    \begin{array}{ll}
        X = x - x_0 - R \cos(\omega t) \\
        Y = y - y_0 - R \sin(\omega t)
    \end{array}
\right.
\end{equation*}
where $R=5m$.

\subsection{Physical parameters}

In this case the tracer advection equation is solved using fixed hydrodynamic conditions.
No bottom friction is imposed and diffusivity of tracer is set to zero.
Angular velocity of the rotating cone is equal to $1 rad.s^{-1}$ which gives a rotation period equal to $T=2\pi$ ($6.28s$). 

\subsection{Numerical parameters}

The simulation time is set to one period of rotation.
The time step is chosen in order to do the whole period in 64 steps, so it is equal to $0.098174771s$.

For tracers advection, all the numerical schemes available in \telemac{2d} are tested. 
For weak characteristics the number of gauss points is set to 12. For distributive schemes, like predictor-corrector (PC) schemes (scheme 4 and 5 with options 2,3) and locally implicit schemes (LIPS: scheme 4 and 5 with options 4), the number of corrections is set to 5, which is usually sufficient to converge to accurate results. For the locally implicit schemes (scheme 4 and 5 with option 4), the number of sub-steps is equal to 20.

%--------------------------------------------------------------------------------------------------
\section{Results}

\subsection{Comparison of schemes}

The final contour maps after one rotation of the cone are plotted for each scheme in Figures \ref{t2d:cone:profiles1} and \ref{t2d:cone:profiles2}. One dimentional profiles are also extracted from
slice plane $(x,y=10,z)$ at $t=T/2$ and $t=T$ on figure \ref{t2d:cone:1dslice}.

\newpage

\begin{figure}[H]
\begin{minipage}[t]{0.50\textwidth}
 \centering
 \includegraphicsmaybe{[width=0.95\textwidth]}{../img/figure_EX.png}
\end{minipage}%
\begin{minipage}[t]{0.50\textwidth}
 \centering
 \includegraphicsmaybe{[width=0.95\textwidth]}{../img/figure_WCHAR.png}
\end{minipage}
\begin{minipage}[t]{0.50\textwidth}
 \centering
 \includegraphicsmaybe{[width=0.95\textwidth]}{../img/figure_ERIA.png}
\end{minipage}
\begin{minipage}[t]{0.50\textwidth}
 \centering
 \includegraphicsmaybe{[width=0.95\textwidth]}{../img/figure_SCHAR.png}
\end{minipage}
\begin{minipage}[t]{0.50\textwidth}
 \centering
 \includegraphicsmaybe{[width=0.95\textwidth]}{../img/figure_NLIPS.png}
\end{minipage}%
\begin{minipage}[t]{0.50\textwidth}
 \centering
 \includegraphicsmaybe{[width=0.95\textwidth]}{../img/figure_PSILIPS.png}
\end{minipage}
\begin{minipage}[t]{0.50\textwidth}
 \centering
 \includegraphicsmaybe{[width=0.95\textwidth]}{../img/figure_PSIPC1.png}
\end{minipage}%
\begin{minipage}[t]{0.50\textwidth}
 \centering
 \includegraphicsmaybe{[width=0.95\textwidth]}{../img/figure_NPC1.png}
\end{minipage} 
  \caption{Cone test: contour maps of tracer after one period of rotation, for the advection schemes of \telemac{2d}.}
 \label{t2d:cone:profiles1}
\end{figure}

\begin{figure}[H]
\begin{minipage}[t]{0.50\textwidth}
 \centering
 \includegraphicsmaybe{[width=0.95\textwidth]}{../img/figure_PSIPC2.png}
\end{minipage}%
\begin{minipage}[t]{0.50\textwidth}
 \centering
 \includegraphicsmaybe{[width=0.95\textwidth]}{../img/figure_NPC2.png}
\end{minipage}
\begin{minipage}[t]{0.50\textwidth}
 \centering
 \includegraphicsmaybe{[width=0.95\textwidth]}{../img/figure_PSI.png}
\end{minipage}%
\begin{minipage}[t]{0.50\textwidth}
 \centering
 \includegraphicsmaybe{[width=0.95\textwidth]}{../img/figure_N.png}
\end{minipage}
 \caption{Cone test: contour maps of tracer after one period of rotation, for the advection schemes of \telemac{2d}.}
 \label{t2d:cone:profiles2}
\end{figure}

\begin{figure}[H]
\begin{minipage}[t]{0.50\textwidth}
 \centering
 \includegraphicsmaybe{[width=\textwidth]}{../img/figure_1d_0,5T.png}
\end{minipage}%
\begin{minipage}[t]{0.50\textwidth}
 \centering
 \includegraphicsmaybe{[width=\textwidth]}{../img/figure_1d_T.png}
\end{minipage}%
\caption{1D solution along slice plane $(x,z)$, $y=10$ at $t=T/2$ (left) and $t=T$ (right).}
\label{t2d:cone:1dslice}
\end{figure}

\newpage

\subsection{Maximum principle}

The minimum value of the gaussian function are measured after one rotation. 
The maximum value is computed as well, in order to check the respect of the maximum principle (or monotonicity). 
Results are showed on figures \ref{t2d:cone:minmax}.

\begin{figure}[H]
\centering
\includegraphicsmaybe{[width=0.9\textwidth]}{../img/t2d_cone_maxT.png}
\includegraphicsmaybe{[width=0.9\textwidth]}{../img/t2d_cone_minT.png}
\caption{Maximum and minimum values of tracer after one rotation of the cone.}
\label{t2d:cone:minmax}
\end{figure}

\subsection{Accuracy}

In order to evaluate the behaviour of the scheme, the error norms $L^1, L^2, L^{\infty}$ are computed.
Error norms are integrated over time to take into account the unsteady nature of the problem.
The error norms integrated over time for one rotation of the cone are reported in figure \ref{t2d:cone:error_timeintegrals}.

\begin{figure}[H]
\centering
\includegraphicsmaybe{[width=0.9\textwidth]}{../img/t2d_cone_errors_timeintegrals.png}
\caption{Error norms integrated over time for one rotation of the cone.}
\label{t2d:cone:error_timeintegrals}
\end{figure}

\newpage
\subsection{Convergence}
To assess the accuracy of the schemes, computation of error on one mesh is not sufficient.
In this section a mesh convergence is carried out for each numerical scheme. From a starting mesh
with $441$ points and $800$ elements with divide by 4 each triangles recursively to generates new meshes.
The first 4 meshes used in the convergence study are presented on figure \ref{t2d:cone:meshes}. 

\begin{figure}[h!]
\begin{minipage}[t]{0.5\textwidth}
 \centering
 \includegraphicsmaybe{[width=\textwidth]}{../img/mesh_0.png}
\end{minipage}%
\begin{minipage}[t]{0.5\textwidth}
 \centering
 \includegraphicsmaybe{[width=\textwidth]}{../img/mesh_1.png}
\end{minipage}
\begin{minipage}[t]{0.5\textwidth}
 \centering
 \includegraphicsmaybe{[width=\textwidth]}{../img/mesh_2.png}
\end{minipage}%
\begin{minipage}[t]{0.5\textwidth}
 \centering
 \includegraphicsmaybe{[width=\textwidth]}{../img/mesh_3.png}
\end{minipage}
 \caption{2D Mesh used in cone mesh convergence}
 \label{t2d:cone:meshes}
\end{figure}

Final time is set to $t_f=T/4$.
With decreasing space step we adjust time step to ensure a constant CFL for each mesh increment.
The CFL is computed with the max of eigen values of the shallow water system.

\

Strong characteristics, N and PSI schemes converge at first order with a slope slightly inferior to one.
LIPS and predictor-corrector (PC) schemes as well as ERIA have steeper slope of convergence comprised 
between one and two. PC2 schemes converge with a CFL close to one but PC1, LIPS and ERIA require smaller 
timestep to converge. Results are shown with a CFL of $0.5$ for which ERIA and LIPS start to diverge on  the
fourth mesh.
Convergence slopes of error in $L^1$, $L^2$ and $L^\infty$ norm at final time
are plotted for each numerical scheme on figure \ref{t2d:cone:mesh_convergence}.


\begin{figure}[h!]
\begin{minipage}[t]{0.50\textwidth}
 \centering
 \includegraphicsmaybe{[width=\textwidth]}{../img/t2d_cone_errors_tf_finemesh_WCHAR.png}
\end{minipage}%
\begin{minipage}[t]{0.50\textwidth}
 \centering
 \includegraphicsmaybe{[width=\textwidth]}{../img/t2d_cone_errors_tf_finemesh_SCHAR.png}
\end{minipage}
\begin{minipage}[t]{0.50\textwidth}
 \centering
 \includegraphicsmaybe{[width=\textwidth]}{../img/t2d_cone_errors_tf_finemesh_PSI.png}
\end{minipage}%
\begin{minipage}[t]{0.50\textwidth}
 \centering
 \includegraphicsmaybe{[width=\textwidth]}{../img/t2d_cone_errors_tf_finemesh_N.png}
\end{minipage}
\begin{minipage}[t]{0.50\textwidth}
 \centering
 \includegraphicsmaybe{[width=\textwidth]}{../img/t2d_cone_errors_tf_finemesh_PSILIPS.png}
\end{minipage}%
\begin{minipage}[t]{0.50\textwidth}
 \centering
 \includegraphicsmaybe{[width=\textwidth]}{../img/t2d_cone_errors_tf_finemesh_NLIPS.png}
\end{minipage}
\begin{minipage}[t]{0.50\textwidth}
 \centering
 \includegraphicsmaybe{[width=\textwidth]}{../img/t2d_cone_errors_tf_finemesh_PSIPC1.png}
\end{minipage}%
\begin{minipage}[t]{0.50\textwidth}
 \centering
 \includegraphicsmaybe{[width=\textwidth]}{../img/t2d_cone_errors_tf_finemesh_NPC1.png}
\end{minipage} 
\begin{minipage}[t]{0.50\textwidth}
 \centering
 \includegraphicsmaybe{[width=\textwidth]}{../img/t2d_cone_errors_tf_finemesh_PSIPC2.png}
\end{minipage}%
\begin{minipage}[t]{0.50\textwidth}
 \centering
 \includegraphicsmaybe{[width=\textwidth]}{../img/t2d_cone_errors_tf_finemesh_NPC2.png}
\end{minipage}
\begin{minipage}[t]{\textwidth}
 \centering
 \includegraphicsmaybe{[width=0.5\textwidth]}{../img/t2d_cone_errors_tf_finemesh_ERIA.png}
\end{minipage}
  \caption{Mesh convergence of numerical schemes}
 \label{t2d:cone:mesh_convergence}
\end{figure}

Convergence slopes of time integrated error in $L^2$ norm are compared on figure \ref{t2d:cone:error_timeintegrals}.
Error with strong characteristics, N and PSI schemes are of the same magnitude as well as 
LIPS, PC, ERIA.
Errors on the finest mesh are presented in figure \ref{t2d:cone:error_timeintegrals_mesh3}.



\begin{figure}[h!]
\centering
\includegraphicsmaybe{[width=0.9\textwidth]}{../img/t2d_cone_errors_timeintegrals_L2_allvars.png}
\caption{Mesh convergence in $L_2$ norm}
\label{t2d:cone:error_timeintegrals}
\end{figure}

\begin{figure}[H]
\centering
\includegraphicsmaybe{[width=0.9\textwidth]}{../img/t2d_cone_errors_timeintegrals_mesh3.png}
\caption{Error norms integrated over time for the finest mesh.}
\label{t2d:cone:error_timeintegrals_mesh3}
\end{figure}

%--------------------------------------------------------------------------------------------------

\section{Conclusion}
\telemac{2d} is able to model passive scalar transport problems in shallow water flows. This test shows that to get higher accuracy and monotonicity in passive scalar transport cases the predictor-corrector distributive schemes (N PC2, PSI PC2) are the most appropriate schemes as long as the time step is carefully chosen depending on element size.



