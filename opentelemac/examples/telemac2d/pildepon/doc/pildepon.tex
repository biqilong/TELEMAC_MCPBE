\chapter{Pildepon}
%
% - Purpose & Problem description:
%     These first two parts give reader short details about the test case,
%     the physical phenomena involved and specify how the numerical solution will be validated
%
\section{Purpose}

This test case shows that \telemac{2d} is able to represent the impact of an obstacle on a channel flow. 
It simulates a laminar and very viscous flow in a channel with two cylindrical piers. The effect of the flow
on the piers is assessed with the computation of the Stroual number.

\section{Description}

\subsection{Geometry}

The channel is 28.5 m long and 20 m wide (L=28.5~m and H=20~m) with two bridge piers positioned at $P_1=(-5,4)$, $P_2=(-5,-4)$ and a diameter D of 4~m.
The geometry is shown in the Figure \ref{fig:geo:bridge}.

\begin{figure}[H]
 \centering
 \includegraphicsmaybe{[width=0.5\textwidth]}{img/geometry.pdf}
 \caption{Geometry of the pildepon test case.}
 \label{fig:geo:bridge}
\end{figure}

\subsection{Mesh and Bathymetry}

The computational domain is made up by $4304$ triangular elements and $2280$ nodes and it is shown in Figure \ref{fig:mesh:bridge}.
The section is trapezoidal (see the bottom in the Figure \ref{fig:mesh:bridge}) and the minimum value of the bottom elevation is equal to -4~m in the main channel.

\begin{figure}[H]
\begin{minipage}[t]{0.45\textwidth}
 \centering
 \includegraphicsmaybe{[width=\textwidth]}{../img/Mesh.png}
\end{minipage}%
\begin{minipage}[t]{0.55\textwidth}
 \centering
 \includegraphicsmaybe{[width=\textwidth]}{../img/Bathy.png}
\end{minipage}
 \caption{Mesh and bathymetry of the pildepon test case.}
 \label{fig:mesh:bridge}
\end{figure}

\subsection{Initial condition}

Initially, water elevation is set to zero as well as velocity.

\begin{figure}[H]
 \centering
 \includegraphicsmaybe{[width=0.7\textwidth]}{../img/bathy.png}
 \caption{Bathymetry and initial free surface at the inlet of the channel.}
 \label{fig:init:bridge}
\end{figure}

\subsection{Boundary conditions}

At the inlet of the channel, we gradually impose as upstream boundary condition a flow discharge Q = $62~\text{m}^3\text{s}^{-1}$,
while at the outlet a null free surface is imposed, which is also the initial condition. On the lateral walls and on the cylinder, slip boundary conditions are imposed.

\subsection{Computation of the Stroual number}

For this case, neither analytical nor experimental solutions are available, but the formation of von Karman vortex is expected behind the piers. The validation is performed computing the Strouhal number for the two piers, given by the following formula:
\begin{equation*}
St=\frac{f_{lift}D}{U}
\end{equation*}
where $f_{lift}$ is the lift frequency which usually corresponds to the vortex shedding frequency, D is the diameter of the cylinder and U is the average free-stream velocity.

%Due to the alternating vortex wake the oscillations in lift force occur at the vortex shedding frequency and oscillations in drag force occur at twice the vortex shedding frequency.
In order to compute the lift frequency, a FFT (Fast Fourier Transform) has been performed on the signal which describes the variation of the force with time. The force is computed as:
\begin{equation*}
F=\int_0^l\int_0^h \rho~gz \vec{n}~dz~ds
\end{equation*}
where $\rho$ is the water density, $g$ is the acceleration of gravity and $\vec{n}$ is the normal vector. The integral is performed on the cylinder with boundary $l$ and along the vertical direction $z$. \\
Finally, in order to check the mass coservation of the advection schemes of \telemac{2d}, a tracer is released at the inlet with the following boundary condition:
\begin{equation*}
 c(x=-13.5,y)=\left\{
\begin{array}{rl}
 2 ~g/l\quad & \text{if} \quad H/2-9~\leq \text{y} \leq H/2-8\\
 1 ~g/l\quad & \text{otherwise}
\end{array}\right .
\end{equation*}
A free condition is imposed at the outlet. The error on the mass is computed as follows:
\begin{equation*}
  \epsilon_{M}= M_{start}+M_{in}-M_{end}
\end{equation*}
M$_{\text{start}}=\int_{\Omega}(hc)^nd\Omega$ is the mass at the beginning of the simulation,  M$_{\text{end}}=\int_{\Omega}(hc)^{n+1}d\Omega$ is the mass at the end of the simulation, M$_{\text{in}}=\int_{\Gamma}hc\vec{u}\vec{n}d\Gamma$ is the mass introduced (and leaved) by the boundaries; where $\Omega$ is the computational domain and $\Gamma$ is its boundary.
The realtive error is computed as:
\begin{equation*}
  \epsilon_{rel}=\frac{\epsilon_{M}}{\max(|M_{start}|,|M_{in}|,|M_{end}|)}
\end{equation*}

\subsection{Physical parameters}

The bottom friction is described by the Strickler law with a coefficient equal to $k_s=40~\text{m}^{1/3}\text{s}^{-1}$.
The fluid considered presents a kinematic viscosity $\nu=0.021~\text{m}^2\text{s}^{-1}$. The average flow velocity in the upstream undisturbed field is
about $U$ = 0.95~m/s. Taking into account the diameter of the cylindrical pier, the Reynolds number is $Re = UD/\nu=180$.

\subsection{Numerical parameters}
\label{sec:num}
To perform an appropriate analysis on several cycles, the simulation time is set to $1200~s$.
Three different numerical configurations are tested with the following parameters:
\begin{table}[H]
  \resizebox{\textwidth}{!}{%
    \begin{tabular}{|c|c|c|c|c|c|}
      \hline Case & Equations & Advection scheme for U & Element for H/U & Solver & Time-step / Desired CFL \\
      \hline A & Wave Eq. FE & Characteristics & P1/P1 & CG & 0.1~s / - \\
      \hline B & Saint-Venant FE & Characteristics & P1/P2 & GMRES & 0.1~s / - \\
      \hline C & Saint-Venant FV & Kinetic order 2 & - & - & - /0.9 \\

      \hline
    \end{tabular}
  }
  \caption{List of the simulation parameters used for the six cases tested in the Thacker example.}
  \label{tab:thacker:cases}
\end{table}

For finite element schemes the treatment of the linear system is set to wave Equation (TREATMENT OF THE LINEAR SYSTEM: 2) and the solver accuracy is set to $10^{-8}$.
For the tracer, all the advection schemes of \telemac{2D} are tested in the configuration A.
In the case of the predictor-corrector schemes, the number of corrections is set to 5; for the LIPS schemes, the number of sub-steps is equal to 10 and the accuracy for diffusion of tracers is set to $10^{-10}$. It is important to note that the last parameter is used even if the keyword \telkey{DIFFUSION OF TRACERS} is set to NO. Indeed, when using the LIPS schemes, a linear system has to be solved and this parameter defines the accuracy of the solver.


\section{Results}

\subsection{First observation}

The velocity norm iso-contour map is presented at inital time and final time on figure \ref{fig:res:velocity}.

\begin{figure}[H]
\begin{minipage}[t]{0.5\textwidth}
 \centering
 \includegraphicsmaybe{[width=\textwidth]}{../img/Velocity_t1.png}
\end{minipage}%
\begin{minipage}[t]{0.5\textwidth}
 \centering
 \includegraphicsmaybe{[width=\textwidth]}{../img/Velocity_tf.png}
\end{minipage}
 \caption{Velocity norm at the begining and at the end of the simulation.}
 \label{fig:res:velocity}
\end{figure}

%Velocity vectors are plotted on figure \ref{fig:res:vectors}.
%\begin{figure}[H]
% \centering
% \includegraphicsmaybe{[width=0.8\textwidth]}{../img/Velocity_arrows.png}
% \caption{Velocity vectors}
% \label{fig:res:vectors}
%\end{figure}

\subsection{Comparison of numerical schemes}

Comparison of free surface and velocity are presented in figure \ref{fig:res:comparison} at $t= 1000s$.

\begin{figure}[H]
\begin{minipage}[t]{0.5\textwidth}
 \centering
 \includegraphicsmaybe{[width=\textwidth]}{../img/FS_VNV_1.png}
\end{minipage}%
\begin{minipage}[t]{0.5\textwidth}
 \centering
 \includegraphicsmaybe{[width=\textwidth]}{../img/Velocity_VNV_1.png}
\end{minipage}
\begin{minipage}[t]{0.5\textwidth}
 \centering
 \includegraphicsmaybe{[width=\textwidth]}{../img/FS_VNV_2.png}
\end{minipage}%
\begin{minipage}[t]{0.5\textwidth}
 \centering
 \includegraphicsmaybe{[width=\textwidth]}{../img/Velocity_VNV_2.png}
\end{minipage}
\begin{minipage}[t]{0.5\textwidth}
 \centering
 \includegraphicsmaybe{[width=\textwidth]}{../img/FS_VNV_3.png}
\end{minipage}%
\begin{minipage}[t]{0.5\textwidth}
 \centering
 \includegraphicsmaybe{[width=\textwidth]}{../img/Velocity_VNV_3.png}
\end{minipage}
 \caption{Free surface (left) and velocity norm (right) at $t=1000s$}
 \label{fig:res:comparison}
\end{figure}


\subsection{Computation of the Stroual number}

On figures \ref{fig:res:fft1}, \ref{fig:res:fft2} and \ref{fig:res:fft3} are plotted the fast Fourier transforms
of the lift force for each cases considred in section\ref{sec:num}. 

\begin{figure}[H]
 \centering
 \includegraphicsmaybe{[width=0.9\textwidth]}{../img/force_P1.png}
 \caption{Fast Fourier transform of the lift force on the piers in case A}
 \label{fig:res:fft1}
\end{figure}

\begin{figure}[H]
 \centering
 \includegraphicsmaybe{[width=0.9\textwidth]}{../img/force_P2.png}
 \caption{Fast Fourier transform of the lift force on the piers in case B}
 \label{fig:res:fft2}
\end{figure}

\begin{figure}[H]
 \centering
 \includegraphicsmaybe{[width=0.9\textwidth]}{../img/force_vf.png}
 \caption{Fast Fourier transform of the lift force on the piers in case C}
 \label{fig:res:fft3}
\end{figure}

The Strouhal number obtained for the different numerical configurations are presented in the following table:

\begin{table}[H]
\centering
\begin{tabular}{l|}
\\ \hline CASE A \\ CASE B \\ CASE C
\end{tabular}%
\begin{tabular}{c|c}
  Strouhal for upper pier & Strouhal for lower pier \\
\hline
\InputIfFileExists{../img/table.txt}{}{}\\
\end{tabular}
\label{t2d:bridge:tab1}
\caption{Pildepon test case: Strouhal number for the upper and lower piers according to the different numerical configurations.}
\end{table}

\subsection{Mass balance}

It can be noted that only the characteristics are not mass conservative.
The mass balance at the end of the simulation according to the different advection schemes
are presented in the following table:

\begin{table}[H]
\centering
\begin{tabular}{l|}
\\ \hline Strong Char. \\ N  \\ N PC1 \\ N PC2 \\ PSI \\ PSI PC1 \\ PSI PC2 \\ PSI LIPS \\ NERD
\end{tabular}%
\begin{tabular}{c|c|c|c|c}
   M$_{\text{start}}$&  M$_{\text{end}}$ & M$_{\text{in}}$ & $\epsilon_{M}$ & $\epsilon_{rel}$\\
\hline
\InputIfFileExists{../img/massb_A.txt}{}{}\\
\end{tabular}
\label{t2d:bridge:balance}
\caption{Pildepon test case: mass balance for the different advection schemes.}
\end{table}

