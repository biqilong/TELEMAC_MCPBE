\chapter{m2wave: Transformation of the M2 tide constituent along the western European Continental shelf}

\section{Purpose}
This test case computes the transformation of the main tide constituent
(M$_2$ - semi-diurnal moon generated wave) occurring along the western Europe continental margin.

\section{Description}
\subsection{Approach}
The model extends from the Iberian Peninsula up to South England. The
whole continental margin of the Atlantic coasts of France is represented
as well as the Southern part of the North Sea. The model is forced by
the M2 tide component. 4 tide cycles are simulated.

\subsection{Geometry and mesh}
Size of the model: domain = 520000~m$\times$950000~m

The mesh is denser near the coastline than along the open Atlantic boundaries.
However, the mesh is not refined along the slope of the continental shelf in
the present model (following studies have shown that this refinement was
important for the overall quality of results).

\begin{itemize}
\item 9414 triangular elements
\item 5007 nodes
\item Maximum size range: from 10 to 50 kilometres
\end{itemize}

\subsection{Boundaries}

\begin{itemize}
\item Ocean open boundaries: H imposed with the tide
\item Coastline: solid walls with slip condition
\end{itemize}

\subsection{Bottom}
Chezy formula with friction coefficient = 40

Mesh and topography are shown on Figure \ref{fig:m2wave:Mesh} and \ref{fig:m2wave:Bathy}

\begin{figure}[H]
 \centering
 \includegraphicsmaybe{[width=.7\textwidth]}{../img/Mesh.png}
  \caption{2D-mesh of the vasque case.}\label{fig:m2wave:Mesh}
\end{figure}
\begin{figure}[H]
 \centering
 \includegraphicsmaybe{[width=.7\textwidth]}{../img/Bathy.png}
  \caption{bathymetry of the vasque case.}\label{fig:m2wave:Bathy}
\end{figure}

\subsection{Numerical parameters}
\begin{itemize}
\item No advection of U and V, Conservative + modified SUPG on depth (mandatory scheme)
\item Elements : Quasi-bubble triangle for velocities, Linear triangle (P1) for H
\item Implicitation for depth and for velocity = 0.55
\item GMRES Solver, with  Solver accuracy = 10$^{-3}$
\item  Initial guess for U = 2
\item  SUPG option: Upwinding equal to 1 for velocity and depth
\item Coriolis and spherical coordinates
\item Time step = 150~s
\item Simulation duration = 178800~s.
\end{itemize}

\section{Results}
The resulting amplitude and phase of the M2 tide component are shown in Figures \ref{fig:m2wave:Ampli} and \ref{fig:m2wave:Phase}. 

More accurate results were found in a subsequent model with a finer mesh along the
continental shelf and along the coastline. More tidal constituents were also taken
into account. Inclusion of the tide generating potential in basic equations of TELEMAC-2D
further improved the results.

\begin{figure}[H]
 \centering
 \includegraphicsmaybe{[width=.7\textwidth]}{../img/ampli.png}
  \caption{M2 Amplitude at t=178800s}\label{fig:m2wave:Ampli}
\end{figure}

\begin{figure}[H]
 \centering
 \includegraphicsmaybe{[width=.7\textwidth]}{../img/phase.png}
  \caption{M2 Phase at t=178800s }\label{fig:m2wave:Phase}
\end{figure}
