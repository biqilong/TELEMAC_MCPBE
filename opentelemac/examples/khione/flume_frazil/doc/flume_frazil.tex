\renewcommand{\labelitemi}{$\triangleright$}

\chapter{Supercooling and frazil ice growth}
%
% - Purpose & Problem description:
%     These first two parts give reader short details about the test case,
%     the physical phenomena involved and specify how the numerical solution will be validated
%
\section{Purpose}
This test case demonstrates that \khione is capable of correctly reproduce
supercooling and frazil ice growth. 
A linear thermal budget formulation is used with an air temperature set to $-10^{\circ} C$. Water is running down a flat flume over $10km$ long at a gentle slope of 1:10,000. At the end of the simulation temperature and frazil ice reach a steady state in the domain. Temperature is decreasing from the inlet to reached a max supercooling temperature of about $-0.1^{\circ} C$, and then converges towards zero as frazil ice is beeing produced.

\section{Description}

\subsection{Geometry, mesh and bathymetry}
The domain is a long flat-bottom flume, boxed within $(x=0, y=0)$ and $(x=10,000, y=150)$, its length being $10km$ along the x-axis and its width being $150m$ along the y-axis. The channel has a slope of 1:10,000 between the elevation $5m$ (upstream) and $4m$ (downstream).

\begin{figure}[H]
    \begin{center}
        \includegraphicsmaybe{[width=\textwidth]}{../img/mesh.png}
    \end{center}
    \caption{Domain mesh}
    \label{fig:growth_mesh}
\end{figure}

The mesh of the domain was created with a uniform density of about $0.35m$, resulting in a mesh with $4800$ elements and $2807$ vertices (cf. figure \ref{fig:growth_mesh}).

\subsection{Initial conditions}

A hydrodynamics steady state condition is established first separately to reach a constant and uniform water depth of $6.6265m$ and a constant and uniform discharge of $300m^3/s$.
Water temperature is initially set at $0.05^\circ C$. Initial frazil volume fraction is set to $0$. These values are also used for the upstream boundary condition.

\begin{figure}[H]
    \begin{center}
        \includegraphicsmaybe{[width=0.85\textwidth]}{../img/profile_elevation.png}
    \end{center}
    \caption{Domain mesh}
    \label{fig:growth_profile}
\end{figure}

\subsection{Boundary conditions}

There are two solid boundaries and 2 open boundaries.
The solid boundaries are on either side of the length of the flume, considered as solid walls with perfect slip conditions.
The open boundaries are at $(x=0)$ for the upstream and at $(x=10,000)$ for the downstream of the flume. A constant discharge of $300m^3/s$ is imposed at the upstream boundary with a temperature set to $0.05^{\circ} C$. A constant water level is set at $6.6265m$ at the downstream boundary.
A first simulation is carried out with for the hydrodynamics only to get to a steady state equilibrium of a constant and uniform water depth of $6.6265m$ and a constant and uniform discharge of $300m^3/s$ along the entire length of the flume.

\subsection{Coupling}

The simulation calls on the thermal exchanges between the atmosphere and the water as computed within \khione. The simulation is, therefore, coupled with \khione. Additionally, because some of the properties of the water and tracer temperature are computed within \waqtel, the simulation is also coupled with \waqtel, although none of its processes are activated.
\begin{itemize}
    \item\textit{COUPLING WITH = 'WAQTEL;KHIONE'}
    \item\textit{WATER QUALITY PROCESS = 1}, no processes activated within \waqtel
    \item\textit{ICE PROCESSES = 2}, thermal budget and frazil growth process activated within \khione
\end{itemize}

\subsection{Atmospheric drivers}

For this test case, linear heat exchanges are used i.e. within the \waqtel steering file:
\begin{itemize}
	\item\textit{ATMOSPHERE-WATER EXCHANGE MODEL = 3}
\end{itemize}

With this model, the net heat flux received by water trought the water/air interface is expressed as:
\begin{equation}
\phi^* = \phi_R + \alpha + \beta (T_{air} - T_{water})
\end{equation}
where $\phi_R$ the net solar radiation flux, which depends on the solar constant $I_{so}$ and the latitude but 
also atmospheric properties like the nebulosity and the visibility. $\alpha$ and $\beta$ are two coefficient that 
are set to $50W.m^{-2}$ and $20W.m^{-2}.K^{-1}$ by default. $I_{so}$, $\alpha$ and $\beta$ are adjusted within 
the \khione steering file:
\begin{itemize}
	\item\textit{SOLAR CONSTANT = 0.}
	\item\textit{WATER-AIR HEAT EXCHANGE CONSTANT = 0.}
	\item\textit{WATER-AIR HEAT EXCHANGE COEFFICIENT = 20.}
\end{itemize}
Air temperature is set to a constant during the simulation:
\begin{itemize}
	\item\textit{AIR TEMPERATURE      = -10.0}
\end{itemize}  

\subsection{Physical parameters}

In order to balance the slope and the discharge through the flume, a Manning friction law is used, with a coefficient of $0.025$. Within the \waqtel steering file, water and air specific heat are set to $1002J.kg^{-1}.K^{-1}$ and $4180 J.kg^{-1}.K^{-1}$.

Physical properties related to frazil growth rate are defined in the \khione steering file. Let us recall
the expression of the frazil thermal growth source term i.e. $S_{GM}$:
\begin{equation}
S_{GM} = \dfrac{q a_0 N_f}{\rho_i L_i}
\end{equation}
where $a_0$ is the surface area of frazil particles normal to the a-axis, $N_f$ is the number of frazil cristals 
per unit volume, $\rho_i$ the density of ice, $L_i$ the latent heat of fusion and $q$ the heat transfert rate 
defined by:
\begin{equation}
q = \dfrac{\lambda N_u}{d_e} (T_f-T_w)
\end{equation}
with $N_u$ the Nusselt number, $d_e$ the thickness of frazil ice particules, $\lambda$ the thermal conductivity of water and $T_f$ the fusion temperature. Note that $a_0$ is defined by $a_0 = \pi d_f d_e$ with $d_f$ the diameter of frazil ice granules. The number of particule per unit volume is defined by $N_f=C_f/V_0$ with $V_0=\pi d_e d_f^2/4$ and $C_f$ the volume fraction of frazil ice. Note that nucleation and secondary nucleation are not simulated in \khione. As a consequence, if initial frazil ice concetration is set to zero, no frazil ice will be produced no matter the supecooling.
To emulate nucleation, frazil particule per unit volume is defined as $N_f=C^0_f/V_0$ if $C_f<C^0_f$, where
$C^0_f$ is a parameter. By default $C^0_f$ is equal to $10^{-16}$. In this test case, $N_u$, $d_f$, $d_e$ and $C^0_f$ are defined as:
\begin{itemize}
	\item\textit{NUSSELT NUMBER = 4.}
	\item\textit{REPRESENTATIVE FRAZIL CRYSTAL LENGTH = 2.0E-3}
	\item\textit{REPRESENATTIVE FRAZIL CRYSTAL THICKNESS = 3.0E-4}
	\item\textit{THRESHOLD OF FRAZIL CONCENTRATION FOR THERMAL GROWTH = 1.0E-16}	
\end{itemize}

\subsection{Numerical parameters}
The time step is set to $2s$ and the number of timestep is set to $18000$ which leads to a simulation time of $24h$.

\section{Results}
The figures \ref{fig:growth_temp} and \ref{fig:growth_Cf} show the temperature and the frazil concentration along the flulme at different times. Steady state is reached after only several hours of simulation. Note that the maximum supercooling observed in the channel depends on thermal growth parameters such as Nusselt number $N_u$, frazil representative cristal size: $d_f$, $d_e$.
\begin{figure}[H]
    \begin{center}
        \includegraphicsmaybe{[width=0.9\textwidth]}{../img/profile_T.png}
    \end{center}
    \caption{Water temperature and frazil growth along the flume}
    \label{fig:growth_temp}
\end{figure}

\begin{figure}[H]
    \begin{center}
        \includegraphicsmaybe{[width=0.9\textwidth]}{../img/profile_Cf.png}
    \end{center}
    \caption{Water temperature and frazil growth along the flume}
    \label{fig:growth_Cf}
\end{figure}

On figures \ref{fig:2d_frazil} and \ref{fig:2d_temperature} the frazil concentration and temperature are plotted along the flume at final time. On figure \ref{fig:growth_Cf_timeseries} the evolution of frazil and temperature against time is plotted for the point 
(x=8000, y=75) located near the end of the flume.

\begin{figure}[H]
    \begin{center}
        \includegraphicsmaybe{[width=\textwidth]}{../img/T-2d_scalarmap.png}
    \end{center}
    \caption{Frazil ice concentration along the flume at final time}
    \label{fig:2d_temperature}
\end{figure}


\begin{figure}[H]
    \begin{center}
        \includegraphicsmaybe{[width=\textwidth]}{../img/Cf-2d_scalarmap.png}
    \end{center}
    \caption{Frazil ice concentration along the flume at final time}
    \label{fig:2d_frazil}
\end{figure}

\begin{figure}[H]
    \begin{center}
        \includegraphicsmaybe{[width=0.9\textwidth]}{../img/timeseries_TCf_at_xy=8000_75.png}
    \end{center}
    \caption{Water temperature and frazil growth against time at ($x=8000$, $y=75$)}
    \label{fig:growth_Cf_timeseries}
\end{figure}

As the water temperature goes below 0$^\circ C$, the creation of frazil floes cools down the water temperature further to a super-cool temperature. In turn, further growth of frazil ice release latent heat, which heats up again water to a temperature just below 0$^\circ C$.

\section{Conclusions}

The variation in water temperature over the length of the flume highlights the process of thermal growth of the frazil ice with a long exposure to cold air temperature. Coming in the flume at 0.05$^\circ C$, the longer water is exposed to the cold air temperature the cooler (or super-cooler) it gets until frazil floes starts to appear.

%~\newline
%\section{Steering files}
%
%\subsection{\telemac2D}
%\lstinputlisting[language=TelemacCas]{../t2d_frazil.cas}
%
%\subsection{\waqtel}
%\lstinputlisting[language=TelemacCas]{../waq_frazil.cas}
%
%\subsection{\khione}
%\lstinputlisting[language=TelemacCas]{../ice_frazil.cas}


\renewcommand{\labelitemi}{\textbullet}
