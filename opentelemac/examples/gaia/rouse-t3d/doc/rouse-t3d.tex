% case name
\chapter{rouse-t3d}
%
% - Purpose & Description:
%     These first two parts give reader short details about the test case,
%     the physical phenomena involved, the geometry and specify how the numerical solution will be validated
%
\section{Purpose}
%
This test validates the modeling of the hydrodynamics and non-cohesive
(i.e. constant settling velocity with no turbulence damping effects)
suspended sediment transport, in a permanent and uniform flow. We
compare the mean flow velocities to the logarithmic profile and the
sediment concentration to an analytical solution derived from the Rouse
profile [1], as:
%
\begin{equation}\label{equation:RouseEqu}
\frac{C_z}{C_a}=\left(\frac{(h-z)a}{(h-a)z}\right)^{\frac{w_s}{\kappa u_*}}
\end{equation}
where:\\
$\it{C_z}$ = suspended sediment concentration at height $\it{z}$ above the bed (kg/m$^3$)\\
$\it{C_a}$ = suspended concentration at reference height $\it{a}$ (kg/m$^3$)\\
$\it{h}$ = water depth (m)\\
$\it{w_s}$ = settling velocity of the suspended sediment (m/s)\\
$\it{\kappa}$ = von Karman constant (= 0.41)\\
$\it{u_*}$ = shear velocity at the bed {m/s}


\section{Description}
%
The test case consists of a steady and uniform flow in a rectangular
channel (5~km $\times$ 0.5~km) with flat bed, without friction on the
lateral boundaries, and with friction on the bottom.
The turbulence model is chosen to be consistent with the logarithmic
velocity profile on the vertical (i.e. mixing length model).
At the entrance of the channel, sediment is introduced with a constant
concentration along the vertical, and an equilibrium profile gradually
appears downstream due to combined settling and vertical diffusion.
%
% - Geometry and Mesh:
%     This part describes the mesh used in the computation
%
%
\subsection{Geometry and Mesh}
%
\subsubsection{Bathymetry}
%
The bathymetry consists of a flat bed with a level of -10m below
mean sea level.
%
\subsubsection{Geometry}
%
The geometry is a rectangular channel with a length of 5~km and
width of 0.5~km.
%
\subsubsection{Mesh}
%
The finite element mesh comprises 756 triangular elements and
444 nodes. The vertical discretisation of the model consists of
21 regularly spaced sigma planes.
%
% - Physical parameters:
%     This part specifies the physical parameters
%
%
\subsection{Physical parameters}
%
\subsubsection{Friction}
Bed friction is specified using a Nikuradse roughness length of 0.01~m

\subsubsection{Turbulence}
For this test case, turbulent diffusivity is most important in the
vertical direction since it controls the vertical profile of the
suspended concentration (along with settling velocity). Therefore, the
vertical turbulence is modelled using the mixing length model of Nezu
and Nakagawa. Horizontal turbulence, which is less important for this
test case, is applied using a constant viscosity of 0.5~m$^2$/s.

\subsubsection{Sediment}
A constant settling velocity of 0.001~m/s is used for the sediment.
Flocculation is not included since this would modfiy the vertical
profile of suspended concentration from the theoretical Rouse profile.\\
%
% Experimental results (if needed)
%\subsection{Experimental results}
%
% bibliography can be here or at the end
%\subsection{Reference}
%
% Section for computational options
%\section{Computational options}
%
% - Initial and boundary conditions:
%     This part details both initial and boundary conditions used to simulate the case
%
%
\subsection{Initial and Boundary Conditions}
%
\subsubsection{Initial conditions}
%
Initial conditions are set using a constant elevation of 0~m (i.e. 10~m
depth) and velocities are not initialised (i.e. they start at zero).
Sediment concentration is initialised to 0.1~g/L everywhere.
%
\subsubsection{Boundary conditions}
%
The upstream and downstream liquid boundaries are prescribed with
boundary conditions to generate steady state flow with a depth average velocity of 1 m/s
follows.\\
\\
Upstream boundary:
\begin{itemize}
\item prescribed flow rate of 5,000~m$^3$/s
\item no velocity profile initialised (it forms downstream)
\item prescribed constant sediment concentration of 0.1~g/L
\item no concentration profile initialised (it forms downstream)
\end{itemize}
Downstream boundary:
\begin{itemize}
\item prescribed free surface at $z$ = -0.05~m
\item free boundary for velocity
\end{itemize}
Bottom:
\begin{itemize}
\item
solid boundary with Nikuradse bed roughness of $k_s$ = 0.01~m
\end{itemize}
Lateral wall:
\begin{itemize}
\item no friction
\end{itemize}
%
\subsection{General parameters}
%
The model was run with a time step of 30~s for a total simulation
duration of 43,200~s (12 hours)
%
% - Numerical parameters:
%     This part is used to specify the numerical parameters used
%     (adaptive time step, mass-lumping when necessary...)
%
%
\subsection{Numerical parameters}
%
Advection for velocities and sediment: Scheme 13 (MURD3D\_POS)
%
\subsection{Comments}
%
% - Results:
%     We comment in this part the numerical results against the reference ones,
%     giving understanding keys and making assumptions when necessary.
%
%
\section{Results}
%
Figure \ref{velprof} compares the theoretical [1] (i.e. logarithmic)
velocity profile with the computed result and shows excellent agreement.
One can see that the point at the first plane above the bottom coincides
with the theoretical value, which guarantees that the friction velocity
is correct.\\
%
% Figure - velocity profile comparison
\begin{figure} [h]
\centering
\includegraphicsmaybe{[scale=0.3]}{../img/velocityU_profile.png}
\caption{Velocity profile comparison}\label{velprof}
\end{figure}
%
%Figure \ref{Kzprof} compares the theoretical [1] and the computed
%vertical turbulent diffusivity profiles.\\
%
% Figure - vertical diffusivity profile comparison
%\begin{figure} [!h]
%\centering
%\includegraphicsmaybe{[scale=0.3]}{../img/Kz_profile.png}
%\caption{Diffusivity profile comparison}\label{Kzprof}
%\end{figure}
%
Figure \ref{Cprof} compares the theoretical Rouse profile and the
numerical solution, showing close agreement.
%In theory, the Rouse profile is only valid beyond the viscous layer.
%The modified profile brings a notable modification only in this viscous
%layer and is presented here for its interest in software validation.
%
% Figure - vertical diffusivity profile comparison
\begin{figure} [h]
\centering
\includegraphicsmaybe{[scale=0.3]}{../img/suspconc_profile.png}
\caption{Concentration profile comparison}\label{Cprof}
\end{figure}
%
%
\section{Conclusion}
%
These comparisons with analytical solutions, in hydrodynamics or
suspended sediment transport, thoroughly validate the treatment of
vertical diffusion and settling velocity in \telemac{3d}.
The mixing length turbulence model of Nezu and Nakagawa and the
computation of vertical velocity gradients are also validated.
%
% - Reference:
%     This part gives the reference solution we are comparing to and
%     explicits the analytical solution when available;
%
% bibliography can be here or at the end
%\subsection{Reference}
%
%
\subsection{Reference}
%
[1] HERVOUET J.-M., VILLARET C. Profil de Rouse modifié, une solution
analytique pour valider TELEMAC-3D en sédimentologie.
EDF-LNHE Report HP-75/04/013/A.
