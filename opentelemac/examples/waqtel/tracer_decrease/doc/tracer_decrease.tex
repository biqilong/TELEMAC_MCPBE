% case name
\chapter{tracer\_decrease}
%
% - Purpose & Description:
%     These first two parts give reader short details about the test case,
%     the physical phenomena involved, the geometry and specify how the numerical solution will be validated
%
\section{Purpose}
%
This case is an example of the use of the degradation law of \waqtel coupled with \telemac{2d}.
%
\section{Description}
%
2 square basins at rest are considered.
%
% - Reference:
%     This part gives the reference solution we are comparing to and
%     explicits the analytical solution when available;
%
% bibliography can be here or at the end
%\subsection{Reference}
%
%
\subsection{Reference}
%

%
% - Geometry and Mesh:
%     This part describes the mesh used in the computation
%
%
\subsection{Geometry and Mesh}
%
\subsubsection{Bathymetry}
%
A flat bathymetry with elevation at 0~m is defined.
%
\subsubsection{Geometry}
%
Basin length and width = 200~m each
%
\subsubsection{Mesh}
%
1,600 triangular elements\\
882 nodes
%
% - Physical parameters:
%     This part specifies the physical parameters
%
%
\subsection{Physical parameters}
%
No diffusion neither for hydrodynamics or tracers\\
%
The degradation law is activated by setting \telkey{WATER QUALITY PROCESS} = 17
in the \telemac{2d} \telkey{STEERING FILE}\\

2 tracers are considered:
\begin{itemize}
\item the 1$^{\rm{st}}$ one without any degradation law,
\item the 2$^{\rm{nd}}$ one with a 1$^{\rm{st}}$ order degradation law with:
  \telkey{COEFFICIENT 1 FOR LAW OF TRACERS DEGRADATION} = 6.
\end{itemize}

%
% Experimental results (if needed)
%\subsection{Experimental results}
%
% bibliography can be here or at the end
%\subsection{Reference}
%
% Section for computational options
%\section{Computational options}
%
% - Initial and boundary conditions:
%     This part details both initial and boundary conditions used to simulate the case
%
%
\subsection{Initial and Boundary Conditions}
%
\subsubsection{Initial conditions}
%
Initial depth~=~2~m
and no initial velocity\\
Initial values for tracers:\\
(Tracer 1, Tracer 2) =
(100, 200)
%
\subsubsection{Boundary conditions}
%
Closed boundaries. No bottom friction
%
\subsection{General parameters}
%
Time step: 10~s\\
Simulation duration: 86,400~s =~1~day
%
% - Numerical parameters:
%     This part is used to specify the numerical parameters used
%     (adaptive time step, mass-lumping when necessary...)
%
%
\subsection{Numerical parameters}
%
Basin at rest (no advection nor diffusion)\\

2 default values for the diffusion of the tracers are modified
in the \telemac{2d} \telkey{STEERING FILE} :
\begin{itemize}
\item \telkey{ACCURACY FOR DIFFUSION OF TRACERS} = 10$^{-10}$ (current default = 10$^{-6}$),
\item \telkey{IMPLICITATION COEFFICIENT OF TRACERS} = 1 (current default = 0.6).
\end{itemize}
%
\subsection{Comments}
%
% - Results:
%     We comment in this part the numerical results against the reference ones,
%     giving understanding keys and making assumptions when necessary.
%
%
\section{Results}
%

%
\section{Conclusion}
%

%
% Here is an example of how to include the graph generated by validateTELEMAC.py
% They should be in test_case/img
%\begin{figure} [!h]
%\centering
%\includegraphics[scale=0.3]{../img/mygraph.png}
% \caption{mycaption}\label{mylabel}
%\end{figure}
%
% bibliography
%\section{Reference}
