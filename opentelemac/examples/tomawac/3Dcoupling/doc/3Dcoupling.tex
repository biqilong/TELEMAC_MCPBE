\chapter{3D coupling}
%
% - Purpose & Problem description:
%     These first two parts give reader short details about the test case,
%     the physical phenomena involved and specify how the numerical solution will be validated
%
\section{Purpose}
%
This test case has been created to test the coupling between tomawac and Telemac3D. Two kind of coupling are tested. In the first one that we will call the calssical one the forces due to waves are constant along the depth of water. This classical coupling is very close to the coupling that is made with telemac2d. In the second coupling that we will call 3D coupling, all the quantities are dependant of the depth. The second coupling is closer to what happen in reality. For more detail about the second coupling the reader can read \cite{Teles2013}.

%
\section{Description of the problem}
We took the geometry from the classical test case 'littoral', a coupling case between Tomawac and Telemac2D and Sisyphe
\section{Geometry and Mesh}
%
The beach is 1000 m long, 200 m wide
 The beach slope (Y=200m) is 5\%.
 The water depth along the open boundary (Y=0) is h=10m
We use a trianglular regular grid.  

The mesh is as shown on Figure \ref{3Dcouplingmesh}
\begin{figure} [!h]
\centering
\includegraphicsmaybe{[width=0.85\textwidth]}{../img/fond.png}
 \caption{Maillage 2D of the domain.}
\label{3Dcouplingmesh}
\end{figure}

\section{Results}
The results are presented Figures \ref{figres3Dcoupl2} (Velocity U on a vertical plan) \ref{figres3Dcoupl3} (velocity on the bottom) \ref{figres3Dcoupl}(Wave heigth Hm0) with the classical coupling.

On Figures \ref{figres3Dcoupl4}, \ref{figres3Dcoupl5} and \ref{figres3Dcoupl6}, we present the results of the 3D coupling. 

\begin{figure} [!h]
\centering
\includegraphicsmaybe{[width=0.85\textwidth]}{../img/resultsTOM.png}
 \caption{Wave Heigth calculated by Tomawac}
\label{figres3Dcoupl}
\end{figure}

\begin{figure} [!h]
\centering
\includegraphicsmaybe{[width=0.85\textwidth]}{../img/resultscoupVert.png}
 \caption{Vitesse U on a vertical plan at x=500}
\label{figres3Dcoupl2}
\end{figure}

\begin{figure} [!h]
\centering
\includegraphicsmaybe{[width=0.85\textwidth]}{../img/resultshori.png}
 \caption{Celerity U on the bottom calculated by Telemac3D}
\label{figres3Dcoupl3}
\end{figure}

\begin{figure} [!h]
\centering
\includegraphicsmaybe{[width=0.85\textwidth]}{../img/resultsTOM3.png}
 \caption{Wave Heigth calculated by Tomawac with the 3D coupling}
\label{figres3Dcoupl4}
\end{figure}

\begin{figure} [!h]
\centering
\includegraphicsmaybe{[width=0.85\textwidth]}{../img/resultscoupVert3.png}
 \caption{Celerity U on a vertical plan at x=500}
\label{figres3Dcoupl5}
\end{figure}

\begin{figure} [!h]
\centering
\includegraphicsmaybe{[width=0.85\textwidth]}{../img/resultshori3.png}
 \caption{Celerity U on the bottom calculated by Telemac3D with the 3D coupling}
\label{figres3Dcoupl6}
\end{figure}

