\chapter{Angular Spred}
\section{Purpose}
This test has been created to check a new functionnality in initial spectrum and boundary spectrum.
Classically the function given at the boundary (or initial spectrum) is of the form :
\bequ
S(f,\theta) = J(f) G(\theta)
\eequ
with $f$ the frequency, $\theta$ the angular dependance, $J$ is often taken as a Jonswap function, and $G(\theta)$ is a function of $\cos$ for example in \cite{Mitsuyasu1975},
\bequ
G(\theta) = G_0 cos^{2s}\left(\frac{\theta-\theta_0}{2}\right)
\eequ
where $\theta_0$ denotes the principal wave direction, $G_0$ is a constant so that the integration of $G$ over $\theta$ is unity. $S$ is the spread parameter that sizes the angle of directions. In the three first option of {\it BOUNDARY} (or {\it INITIAL}) {\it ANGULAR DISTRIBUTION FUNCTION}, s is a constant. Here we check a fourth option proposed by Goda in \cite{Goda1975}, where s depends on frequency too.
\bequ
s= \left\{ \barr{l} s_{max} \left(\frac{f}{f_p} \right)^5 : f\le f_p \\
 s_{max} \left(\frac{f}{f_p} \right)^{-2.5} : f > f_p \earr\right.
 \eequ
 Where $f_p$ is the peak frequency, and $s_{max}$ should be 10, 25 and 75 for wind waves, swell with short decay distance, and swell with long decay distance. In our cas we take an intermediary value {\it GODA COEFFICIENT FOR ANGULAR SPEADING = 35}. The default value is 25. 

\section{Description of the problem}
The problem is the same as the test case Shoal. 

\section{Results}

\begin{figure} [!h]
\centering
\includegraphicsmaybe{[width=0.85\textwidth]}{../img/hm0.png}
 \caption{Wave Heigth}
\label{figresAngularSpred}
\end{figure}

\begin{figure} [!h]
\centering
\includegraphicsmaybe{[width=0.85\textwidth]}{../img/spectrum.png}
 \caption{Spectrum imposed at the left bounday of the domain.}
\label{figresAngularSpectrum}
\end{figure}

