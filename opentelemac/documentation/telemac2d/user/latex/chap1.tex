\chapter{Introduction}
\label{ch:intro}


\section{Presentation of the \telemac{2D} software}

The \telemac{2D} code solves depth-averaged free surface flow equations
as derived first by Barr\'{e} de Saint-Venant in 1871
(also known as Shallow Water Equations).
The main results at each node of the computational mesh are the water depth
and the depth-averaged velocity components.
The main application of \telemac{2D} is in free-surface maritime
or river hydraulics and the program is able to take into account the following
phenomena:

\begin{itemize}
\item Propagation of long waves, including non-linear effects,
\item Friction on the bed,
\item The effect of the Coriolis force,
\item The effects of meteorological phenomena such as atmospheric pressure, rain
 or evaporation and wind,
\item Turbulence,
\item Supercritical and subcritical flows,
\item Influence of horizontal temperature and salinity gradients on density,
\item Cartesian or spherical coordinates for large domains,
\item Dry areas in the computational field: tidal flats and flood-plains,
\item Transport and diffusion of a tracer by currents, including creation
and decay terms,
\item Particle tracking and computation of Lagrangian drifts,
\item Treatment of singularities: weirs, dykes, culverts, etc.,
\item Dyke breaching,
\item Drag forces created by vertical structures,
\item Porosity phenomena,
\item Wave-induced currents (by coupling or chaining with the \artemis and
\tomawac modules),
\item Coupling with sediment transport,
\item Coupling with water quality tools,
\item Coupling with ice/frazil.
\end{itemize}

The software has many fields of application.
In the maritime sphere, particular mention may be given of the sizing of port
structures, the study of the effects of building submersible dykes or dredging,
the impact of waste discharged from a coastal outfall or the study of thermal
plumes.
In river applications, mention may also be given of studies relating to the impact
of construction works (bridges, weirs, tubes), dam breaks,
flooding or the transport of decaying or non-decaying tracers.
\telemac{2D} has also been used for a number of special applications,
such as the bursting of industrial reservoirs, avalanches falling
into a reservoir, etc.

\telemac{2D} was developed initially by the National Hydraulics and Environment
Laboratory (Laboratoire National d'Hydraulique et Environnement - LNHE)
of the Research and Development Directorate of the French Electricity Board
(EDF R\&D), and is now managed by a consortium
of other consultants and research institutes, more informations can be found
on the website www.opentelemac.org.
Like previous releases of the program, release 8.1 complies with EDF-R\&D's
Quality Assurance procedures for scientific and technical programs.
This sets out rules for developing and checking product quality at all stages.
In particular, a program covered by Quality Assurance procedures is accompanied
by a validation document that describes the field of use of the software
and a set of test cases.
This document can be used to determine the performance and limitations of the
software and define its field of application.
The test cases are also used for developing the software
and are checked at least each time new releases are produced.


\section{Position of the \telemac{2d} code within the telemac modelling system}

The \telemac{2d} software is part of the \tel modelling system developed by
the LNHE of EDF R\&D. \tel is a set of modelling tools allowing to treat
every aspects of natural free surface hydraulics: currents, waves, transport of
tracers and sedimentology.

The pre-processing and post-processing of simulations can be done either
directly within the \tel system or with different software that present an
interface of communication with the system. We can particularly mention the
following tools:

\begin{itemize}
\item The FUDAA-PREPRO software, developed from the FUDAA platform
by the CEREMA's Recherche, Informatique et Modélisation Department, covers all
the pre-processing tasks involved by the achievement of a numerical hydraulic
study, as well as a graphical post-processing tool,
\item The Blue Kenue software, developed the Hydraulic Canadian
Center, proposes a powerful mesh generation tool and a user-friendly
post-processing tool,
\item The Janet software, developed by Smile Consult GmbH, which offers among
others, a mesh generation tool,
\item The ParaView software, developed by Sandia National Laboratories, Los
Alamos National Laboratory and Kitware, which enables to visualise 3D results,
big data in particular and is open source,
\item The SALOME-HYDRO software based on the SALOME platform, developed by EDF,
CEA and OPENCASCADE which enables to handle raw data (bathymetry, maps,
pictures, LIDAR\ldots) until the mesh generation.
The post-processing tool ParaViS available in the SALOME platform is based on
the ParaView software and can visualise 1D, 2D or 3D results.
A first version of SALOME-HYDRO has been available since Spring 2016,
\item The Tecplot 360 software, developed by Tecplot which enables to visualise
2D and 3D results,
\item The QGIS software, which is an open source Geographic Information System.
\end{itemize}


\section{User programming}

Users may wish to program particular functions of a simulation module
that are not provided for in the standard version of the \tel system.
This can be done in particular by modifying specific subroutines called user
subroutines.
These subroutines offer an implementation that can be modified
(provided that the user has a minimum knowledge of FORTRAN and with the help
of the guide for programming in the \tel system).

The following procedure should be followed:

\begin{itemize}
\item Recover the standard version of the subroutines provided with the system,
and copy them into a single file or in a directory that will be the specific
\telkey{FORTRAN FILE} of the given case, see section \ref{subs:FORT:user:file}
for more details,

\item Modify the subroutines according to the model you wish to build,

\item Link up the set of subroutines into a single file or in a single directory
that will be compiled during the \telemac{2D} start procedure.
The name for the file or directory is given by the \telkey{FORTRAN FILE} keyword.
\end{itemize}

During this programming phase, users must access the various software variables.
By using the structures of FORTRAN 90 gathered into a "module" type component,
access is possible from any subroutine.

The set of data structures is gathered in FORTRAN files referred to as modules.
In the case of \telemac{2D}, the file is called \telfile{DECLARATION\_TELEMAC2D}
and is provided with the software.
To access \telemac{2D} data, simply insert the command
\telfile{USE DECLARATIONS\_TELEMAC2D} at the beginning of the subroutine.
It may also be necessary to add the command \telfile{USE BIEF}.


Almost all the arrays used by \telemac{2D} are declared in the form of
a structure with pointers.
For example, access to the water depth variable is in the form \telfile{H\%R}
where the \telfile{\%R} indicates that a pointer of real type is being used.
If the pointer is of integer type, the \telfile{\%R} is replaced by a
\telfile{\%I}.
However, to avoid having to handle too many \telfile{\%R} and \telfile{\%I},
a number of aliases have been defined, such as for example the variables
\telfile{NPOIN, NELEM, NELMAX} and \telfile{NPTFR}.

