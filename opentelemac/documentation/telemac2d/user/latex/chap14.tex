\chapter{Other configurations}
\label{ch:oth:conf}

\section{Modification of bottom topography (\telfile{USER\_CORFON})}
\label{sec:mod:bott:topo}
Bottom topography may be introduced at various levels, as stated
in section \ref{subs:topo:bathy:data}.

\telemac{2d} offers the possibility of modifying the bottom topography
at the beginning of a computation using the \telfile{USER\_CORFON} subroutine.
This is called up once at the beginning of the computation and enables the value
of variable \telfile{ZF} to be modified at each point of the mesh.
To do this, a number of variables such as the point coordinates, the element
surface value, connectivity table, etc. are made available to the user.

By default, the \telfile{CORFON} subroutine (which calls the
\telfile{USER\_CORFON} subroutine) carries out a number
of bottom smoothings equal to \telfile{LISFON},
i.e. equal to the number specified by the keyword \telkey{BOTTOM SMOOTHINGS}
for which the default value is 0 (no smoothing).

The \telfile{CORFON} subroutine is not called up
if a computation is continued.
This avoids having to carry out several bottom smoothings or modifications
of the bottom topography during the computation.


\section{Modifying coordinates (\telfile{USER\_CORRXY})}

\telemac{2d} also offers the possibility of modifying the mesh point coordinates
at the beginning of a computation.
This means, for example, that it is possible to change the scale
(from that of a reduced-scale model to that of the real object),
rotate or translate the object.

The modification is done in the \telfile{USER\_CORRXY} subroutine
(\bief library), which is called up at the beginning of the computation.
This subroutine is empty by default and gives an example of programming
a change of scale and origin, within commented statements.

It is also possible to specify the coordinates of the origin point of the mesh.
This is done using the keyword \telkey{ORIGIN COORDINATES}
which specify 2 integers (default = (0;0)).
These 2 integers will be transmitted to the results file in the SERAFIN format,
for a use by post-processors for superimposition of results with digital maps
(coordinates in meshes may be reduced to avoid large real numbers).
These 2 integers may also be used in subroutines under the names
\telfile{I\_ORIG} and \telfile{J\_ORIG}.
Otherwise they do not have a use yet.


\section{Spherical coordinates (\telfile{LATITU})}
\label{sec:spher:coord:LATI}
If a simulation is performed over a large domain, \telemac{2d} offers
the possibility of running the computation with spherical coordinates.

This option is activated when the keyword \telkey{SPHERICAL COORDINATES}
is set to YES (default value = NO).
In this case, \telemac{2d} calls a subroutine named \telfile{LATITU}
through the subroutine \telfile{INBIEF} at the beginning of the computation.
This calculates a set of tables depending on the latitude of each point.
To do this, it uses the Cartesian coordinates of each point provided
in the geometry file, and the latitude of origin point of the mesh
provided by the user in the steering file with the keyword
\telkey{LATITUDE OF ORIGIN POINT} (default value = 48 degrees).

By default, \telemac{2d} assumes that the mesh coordinates are given
in Cartesian coordinates.
The user can change this choice by using the keyword
\telkey{SPATIAL PROJECTION TYPE} (default is 1 which corresponds to Cartesian
coordinates).
Indeed, when choosing the value 2, the coordinates are considered
in accordance with Mercator's projection.
The value 3, the mesh has to be in longitude-latitude (in degrees!).
It is important to notice here that, if option \telkey{SPHERICAL COORDINATES}
= YES, \telkey{SPATIAL PROJECTION TYPE} has to be 2 or 3.

The \telfile{LATITU} subroutine (\bief library) may be modified by the user
to introduce any other latitude-dependent computation.


\section{Adding new variables (\telfile{USER\_NOMVAR\_TELEMAC2D}
and \telfile{USER\_PRERES\_TELEMAC2D})}

A standard feature of \telemac{2d} is the storage of some computed variables.
In some cases, the user may wish to compute other variables and store them
in the results file (the number of variables is currently limited to four).

\telemac{2d} has a numbering system in which, for example,
the array containing the Froude number has the number 7.
The new variables created by the user may have the numbers 23, 24, 25 and 26.

In the same way, each variable is identified by a letter in the keyword
\telkey{VARIABLES FOR GRAPHIC PRINTOUTS}.
The new variables are identified by the letters N, O, R and Z,
which correspond respectively to the numbers 23, 24, 25 and 26.

In the \telfile{USER\_NOMVAR\_TELEMAC2D} subroutine, it is
possible to change the abbreviations (mnemonics) used for the keywords
\telkey{VARIABLES FOR GRAPHIC PRINTOUTS} and
\telkey{VARIABLES FOR LISTING PRINTOUTS}.
Sequences of 8 letters may be used.
Consequently, the variables must be separated by spaces, commas or
semicolons in the keywords, e.g.:

\begin{lstlisting}[language=TelemacCas]
VARIABLES FOR GRAPHIC PRINTOUTS : 'U, V, H, B'
\end{lstlisting}

In the software data structure, these four variables correspond to the tables
\telfile{PRIVE\%ADR(1)\%P\%R(X), PRIVE\%ADR(2)\%P\%R(X), PRIVE\%ADR(3)\%P\%R(X)}
and \telfile{PRIVE\%ADR(4)\%P\%R(X)} (in which \telfile{X} is the number of
nodes in the mesh).
These may be used in several places in the programming, like all \tel variables.
For example, they may be used in the subroutines \telfile{USER\_CORRXY},
\telfile{USER\_CORSTR}, \telfile{USER\_BORD} etc.
If a \telfile{PRIVE} table is used to program a case, it is essential
to check the value of the keyword \telkey{NUMBER OF PRIVATE ARRAYS}.
This value fixes the number of tables used (0, 1, 2, 3 or more)
and then determines the amount of memory space required.
The user can also access the tables via the aliases
\telfile{PRIVE1, PRIVE2, PRIVE3} and \telfile{PRIVE4}.

An example of programming using the second \telfile{PRIVE} table is given below.
It is initialised with the value 10.

\begin{lstlisting}[language=TelFortran]
DO I=1,NPOIN
  PRIVE%ADR(2)%P%R(I) = 10.D0
ENDDO
\end{lstlisting}

New variables are programmed in two stages:

\begin{itemize}
\item Firstly, it is necessary to define the name of these new variables
by filling in the \telfile{USER\_NOMVAR\_TELEMAC2D} subroutine.
This consists of two equivalent structures, one for English and the other
for French.
Each structure defines the name of the variables in the results file
that is to be generated and then the name of the variables to be read
from the previous computation if this is a restart.
This subroutine may also be modified when, for example, a file generated with
the English version of \telemac{2d} is to be continued with the French version.
In this case, the \telfile{TEXTPR} table of the French part of the subroutine
must contain the English names of the variables,

\item Secondly, it is necessary to modify the \telfile{USER\_PRERES\_TELEMAC2D}
subroutine in order to introduce the computation of the new variable(s).
The variables \telfile{LEO, SORLEO, IMP, SORIMP} are also used to
determine whether the variable is to be printed in the printout file
or in the results file at the time step in question.
\end{itemize}


\section{Array modification or initialization}

When programming \telemac{2d} subroutines, it is sometimes necessary to
initialize a table or memory space to a particular value.
To do that, the \bief library furnishes a subroutine called \telfile{FILPOL}
that lets the user modify or initialize tables in particular mesh areas.

A call of the type \telfile{CALL FILPOL (F, C, XSOM, YSOM, NSOM, MESH)}
fills table \telfile{F} with the \telfile{C} value in the convex polygon
defined by \telfile{NSOM} nodes (coordinates \telfile{XSOM, YSOM}).
The variable \telfile{MESH} is needed for the \telfile{FILPOL} subroutine
but has no meaning for the user.


\section{Validating a computation (\telfile{BIEF\_VALIDA})}

The structure of the \telemac{2d} software offers an entry point for validating
a computation, in the form of a subroutine named \telfile{BIEF\_VALIDA},
which has to be filled by the user in accordance with each particular case.
Validation may be carried out either with respect to a reference file
(which is therefore a file of results from the same computation that is taken
as reference, the name of which is supplied by the keyword
\telkey{REFERENCE FILE}), or with respect to an analytical solution that must
then be programmed entirely by the user.

When using a reference file, the keyword \telkey{REFERENCE FILE FORMAT}
specifies the format of this binary file ('SERAFIN ' by default).

The \telfile{BIEF\_VALIDA} subroutine is called at each time step
when the keyword \telkey{VALIDATION} has the value YES,
enabling a comparison to be done with the validation solution at each time step.
By default, the \telfile{BIEF\_VALIDA} subroutine only does a comparison
with the last time step.
The results of this comparison are given in the output listing.


\section{Changing the type of a boundary condition (\telfile{PROPIN\_TELEMAC2D})}
\label{sec:chang:type:bc:propin}
During a simulation, the type of boundary condition is generally fixed and, in
the case of \telemac{2d}, is provided by the \telkey{BOUNDARY CONDITIONS FILE}.
However, in some cases, it may be necessary to change the type of boundary
conditions during the computation (section of a river subject to tidal effects
where the current alternates, for instance).

This change in boundary condition type must be done in the
\telfile{PROPIN\_TELEMAC2D} subroutine.

N.B: modifying \telfile{PROPIN\_TELEMAC2D} is a difficult operation
and must be done with great care!


\section{Coupling}
\label{sec:coupling}

The principle of coupling two or several simulation modules involves
running the two calculations simultaneously and exchanging the various results
at each time step.
For example, the following principle is used to couple the hydrodynamic module
and the sediment transport module:

\begin{itemize}
\item The two codes perform the calculation at the initial instant
with the same information (in particular the mesh and bottom topography),

\item The hydrodynamic code runs a time step and calculates the water depth
and velocity components.
It provides this information to the sediment transport code,

\item The sediment transport code uses this information to run the solid
transport calculation over a time step and thus calculates a change in the
bottom,

\item The new bottom value is then taken into account by the hydrodynamic module
at the next time step, and so on.
\end{itemize}

Several modules can be coupled in the current version of the code:
the sediment transport modules \gaia or \sisyphe,
the sea state computational module \tomawac,
the water quality module \waqtel (and even DELWAQ)
and the ice module \khione.
The time step used for the two calculations is not necessarily the same and is
managed automatically by the coupling algorithms
and the keyword \telkey{COUPLING PERIOD FOR SISYPHE} and
\telkey{COUPLING PERIOD FOR TOMAWAC} with default values 1
(coupling at every iteration).

This feature requires two keywords.
The keyword \telkey{COUPLING WITH} indicates which simulation code is to be
coupled with \telemac{2d}.
The values of this keyword can be:

\begin{itemize}
\item \telkey{COUPLING WITH} = `GAIA' for coupling with the \gaia module,

\item \telkey{COUPLING WITH} = `SISYPHE' for coupling with the \sisyphe module,

\item \telkey{COUPLING WITH} = `TOMAWAC' for coupling with the \tomawac module,

\item \telkey{COUPLING WITH} = `TOMAWAC2' for coupling with the \tomawac module
and possibly different geometry files (see below),

\item \telkey{COUPLING WITH} = `WAQTEL' for coupling with the \waqtel module,

\item \telkey{COUPLING WITH} = `KHIONE' for coupling with the \khione module.

%\item \telkey{COUPLING WITH} = `SISYPHE, TOMAWAC' for coupling with both.
\end{itemize}

If wanting to couple with 2 modules or more, the differents modules are to
be written in the \telkey{COUPLING WITH} separated with semicolon, for example:
\telkey{COUPLING WITH} = `GAIA, TOMAWAC'.

Depending on the module(s) used, the keywords
\telkey{GAIA STEERING FILE}, \telkey{SISYPHE STEERING FILE},
\telkey{TOMAWAC STEERING FILE}, \telkey{WAQTEL STEERING FILE}
and \telkey{KHIONE STEERING FILE}
indicate the names of the steering files
of the coupled modules.\\

If coupling with the water quality module \waqtel, the integer keyword
\telkey{WATER QUALITY PROCESS} must be set to a value different from 1
(default = 1) in the \telemac{2d} steering file.
The possible choices are:
\begin{itemize}
\item 0: all available processes,
\item 1: nothing (default value),
\item 2: O$_2$ module,
\item 3: BIOMASS module,
\item 5: EUTRO module,
\item 7: MICROPOL module,
\item 11: THERMIC module,
%\item 13: AED2 model,
\item 17: degradation law.
\end{itemize}

Several modules can be combined by giving the multiplication of the process
choices, e.g. 55 = 5 $\times$ 11 activates EUTRO and THERMIC modules.
Note that AED2 is currently not coupled with \telemac{2d} contrary to
\telemac{3d}.
%It is noted that AED2 should be used on its own, for the time being,
%without possible combination with other processes.
Please refer to the \waqtel documentation for additional informations for
\waqtel.\\

If coupling with the ice module \khione, the integer keyword
\telkey{ICE PROCESSES} must be set to a value different from 1
(default = 1) in the \telemac{2d} steering file.
This keyword provides the ice process number with the number being defined on
the basis of a multiplication of primary numbers (2, 3, 5, 7, 11, 13\ldots).
For instance, 14 (= 2 $\times$ 7) activates processes 2 and 7.
The possible choices are:
\begin{itemize}
\item 0: all available processes are included,
\item 1: no ice process included (default value),
\item 2: thermal budget+frazil growth,
\item 3: ice cover impact on hydrodynamics,
\item 5: clogging on racks,
\item 7: static border ice growth.
\end{itemize}

The FORTRAN files of the different modules can be used and are compiled
independently (check that the FORTRAN files of \gaia, \sisyphe and \tomawac
do not contain a main program).\\

The keyword \telkey{COUPLING WITH} is also used if the computation
has to generate the appropriate files necessary to run a water quality
simulation with DELWAQ.
In that case, it is necessary to specify \telkey{COUPLING WITH} = 'DELWAQ'.
Please refer to Appendix \ref{tel2d:app4} for all informations concerning
communications with DELWAQ.\\

In the case of coupling \telemac{2d} and \sisyphe, the bed roughness can be
determined directly by \sisyphe if the keyword
\telkey{BED ROUGHNESS PREDICTION} is enabled in the settings file
sediment transport model. This option can be useful for applications
accounting for rippled or mega-rippled bed or dunes.
If this option is used, the friction law on the bottom used
in the hydrodynamic calculation of \telemac{2d} must necessarily be
the law of Nikuradse (\telkey{LAW OF BOTTOM FRICTION} = 5).\\

A very particular coupling can be performed using \telkey{COUPLING WITH} =
'TOMAWAC2'.
In that case, the geometry files of \telemac{2d} and \tomawac do not need
to be the same anymore.
It only works if MPI is available and if the two geometry files contain
variables indicating nodes and weights to be used to interpolate for each node
of the mesh.

For example, if the \tomawac geometry file contains the variables:

\begin{lstlisting}
TELTOM01 = 1, 1, 2, 4
TELTOM02 = 2, 2, 3, 5
TELTOM03 = 3, 3, 4, 6
TELTOM01WTS = 0.33, 0., 0.33, 0.5
TELTOM02WTS = 0.33, 1., 0.33, 0.
TELTOM03WTS = 0.34, 0., 0.34, 0.5
\end{lstlisting}

Then to interpolate the values on the first point,
\tomawac uses the points 1, 2, 3 of the \telemac{2d} geometry file with the
weights 0.33, 0.33, 0.34,
for the second one: 1, 2, 3 with the weights 0., 1., 0.,
for the third one: 2, 3, 4 with the weights 0.33, 0.33, 0.34
and for the fourth one: 4, 5, 6 with the weights 0.5, 0., 0.5.
Obviously for each triplet, the sum of weights has to be equal to 1.

An example can be found in the littoral \sisyphe example with the two
geometry files geo\_tom\_tom2tel\_diff.slf and
geo\_t2d\_tel2tom\_diff.slf.

In the future, this feature could be extended to the coupling of other modules.


\section{Assigning a name to a point}

During some types of processing, for example a Fourier series analysis
(see 14.10),
it may be useful to assign a name to a point.
This is easy to do by using the two keywords \telkey{LIST OF POINTS}
and \telkey{NAMES OF POINTS}.
The former provides a list of node numbers (100 max) in the general numbering
system,
and the second provides the corresponding names (string of 32 characters max).

For example, in the case of a model of the Channel, point 3,489 corresponds
to the port of Saint-Malo and point 56,229 to the port of Cherbourg.
In this case, the names will be assigned as follows:
\begin{lstlisting}[language=bash]
 LIST OF POINTS: 3489; 56229
 NAMES OF POINTS: `SAINT MALO';'CHERBOURG'
\end{lstlisting}

\section{Fourier analysis}

\telemac{2d} allows the user to analyze free surface variations
in order to determine the phase and amplitude of one or more waves.
This can only be done if the mean level is zero.
Amplitudes and phases are supplied for each point and for each period.

This feature is activated by the keyword \telkey{FOURIER ANALYSIS PERIODS} and
provides a list of the analysis periods (e.g. the periods of tide-induced waves
that are to be studied).
The results are supplied directly at the last time step in the results file with
the names \telfile{AMPLITUDE1}, \telfile{AMPLITUDE2} etc. for the amplitudes
and \telfile{PHASE1}, \telfile{PHASE2} etc. for the phases.
The user estimates the minimum duration of the simulation.
The keyword \telkey{NUMBER OF FIRST TIME STEP FOR GRAPHIC PRINTOUTS}
can be used to reduce the size of the results file.

It is also necessary to specify the time range using the keyword
\telkey{TIME RANGE FOR FOURIER ANALYSIS} associated with 2 real values:
the starting time in seconds and the ending time in seconds separated by a
semicolon.
If this keyword is left with its default values (0;0),
the computation will stop with an error message.

\section{Checking the mesh (\telfile{CHECKMESH})}

The \telfile{CHECKMESH} subroutine of the \bief library is available to look for
errors in the mesh, e.g. superimposed points \ldots
The keyword \telkey{CHECKING THE MESH} (default value = NO) should be activated
to YES to call this subroutine.

This option works in both sequential and parallel modes.
