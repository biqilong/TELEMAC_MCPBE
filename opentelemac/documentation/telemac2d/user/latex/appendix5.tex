\chapter{Defining friction by domains}
\label{tel2d:app5}
When a complex definition of the friction has to be used for a computation, this
option can be chosen, which divides the domain in sub-domains (domains of friction)
where different parameters of friction can be defined and easily modified.
The procedure is triggered by the keyword \telkey{FRICTION DATA} = YES
(default = NO) and the data are contained in a file \telkey{FRICTION DATA FILE}.

The user has to:

\begin{itemize}
\item  define the domains of friction in the mesh,

\item  define the parameters of friction for each domain of friction,

\item  add the corresponding keywords in the steering file of \telemac{2d}
in order to use this option.
\end{itemize}

\textbf{I -- Friction domains}

In order to make a computation with variable coefficients of friction,
the user has to describe, in the computational domain, the zones where the
friction parameters will be the same.
For that, a code number, which represents a friction domain, has to be given to
each node.
The nodes with the same code number will use the same friction parameters.

This allocation is done thanks to the \telfile{FRICTION\_USER} user subroutine.
All nodes can be defined ''manually'' in this subroutine, or this subroutine can
be used in order to read a file where the link between nodes and code numbers is
already generated (for example with the Janet software from the SmileConsult).
This file is called \telkey{ZONES FILE} and will be partitioned in case of
parallelism.

\textbf{II -- Friction parameters}

The frictions parameters of each friction domain are defined in the
\telkey{FRICTION DATA FILE}.
In this file we find, for each code number of friction domain:

\begin{itemize}
\item  a law for the bottom and their parameters,

\item  a law for the boundary conditions and their parameters
(only if the option $k-\epsilon$ is used),

\item  the parameters of non-submerged vegetation (only if the option is used).
\end{itemize}

Example of friction data file:



\begin{tabular}{|p{0.4in}|p{0.5in}|p{0.3in}|p{0.5in}|p{0.6in}|p{0.5in}|p{0.5in}|p{0.6in}|p{0.6in}|} \hline
*Zone & Bottom &  &  & Boundary & condition &  & Non submerged & vegetation \\ \hline
*no & TypeBo & Rbo & MdefBo & TypeBo & Rbo & MdefBo & Dp & sp \\ \hline
From 4 to 6 & NFRO &  &  & LOGW & 0.004 &  & 0.002 & 0.12 \\ \hline
20 & NIKU & 0.10 &  & NIKU & 0.12 &  & 0.006 & 0.14 \\ \hline
27 & COWH & 0.13 & 0.02 & LOGW & 0.005 &  & 0.003 & 0.07 \\ \hline
END &  &  &  &  &  &  &  &  \\ \hline
\end{tabular}


\begin{itemize}
\item The first column defines the code number of the friction domain.
Here, there are 3 lines with the code numbers: 4 to 6, 20, 27,

\item The columns from 2 to 4 are used in order to define the bottom law:
the name of the law used (\telkey{NFRO}, \telkey{NIKU} or \telkey{COWH} for this
example, see below for the name of the laws),
the roughness parameter used and the Manning's default value
(used only with the Colebrook-White law).
If the friction parameter (when there is no friction) or the Manning's default
are useless, nothing has to be written in the column,

\item The columns from 5 to 7 are used to describe the boundary conditions laws:
name of the law, roughness parameter, Manning's default.
These columns have to be set only if the boundaries are considered rough
(keyword \telkey{TURBULENCE REGIME FOR SOLID BOUNDARIES} = 2), otherwise,
nothing has to be written in these columns,

\item The columns 8 and 9 are used for the non-submerged vegetation:
diameter of roughness elements and spacing of roughness elements.
These columns have to be set only if the option non-submerged vegetation is used,
else nothing has to be written in these columns,

\item The last line of the file must have only the word END, (or FIN or ENDE).
\end{itemize}

In order to add a comment in the \telkey{FRICTION DATA FILE},
the line must begin with a star ''*''.

Link between the laws implemented and their names in the friction data file:

\begin{tabular}{|p{1.0in}|p{0.5in}|p{0.8in}|p{1.0in}|} \hline
Law &  Number & Name for data file & Parameters used \\ \hline
No Friction & 0 & NOFR & No parameter \\ \hline
Haaland & 1 & HAAL & Roughness coefficient \\ \hline
Ch\'{e}zy & 2 & CHEZ & Roughness coefficient \\ \hline
Strickler & 3 & STRI & Roughness coefficient \\ \hline
Manning & 4 & MANN & Roughness coefficient \\ \hline
Nikuradse & 5 & NIKU & Roughness coefficient \\ \hline
Log law of wall (1) & 6 & LOGW & Roughness coefficient \\ \hline
Colebrook-White & 7 & COWH & Roughness coefficient\newline Manning coefficient \\ \hline
\end{tabular}

(1) : can be used only for boundaries conditions

\textbf{III -- Steering file}

In order to use a friction computation by domains, the next keywords have to be
added:

For the friction data file:

\telkey{FRICTION DATA} = YES.

\telkey{FRICTION DATA FILE} = `name of the file where friction is given'.

For the non-submerged vegetation (if used) :

\telkey{NON-SUBMERGED VEGETATION} = YES.

By default, 10 zones are allocated, this number can be changed with the keyword:

\telkey{MAXIMUM NUMBER OF FRICTION DOMAINS} = 80.

Link between nodes and code numbers of friction domains is achieved with:

\telkey{ZONES FILE} = `name of the file'.

\textbf{IV -- Advanced options}

If some friction domains with identical parameters have to be defined,
it is possible to define them only with one line thanks to the keyword:
from... to... (it is also possible to use French de... a...
or German von... bis...).

The first code number of the domains and the last code number of the domains
have to be set.
All domains of friction with a code number between these two values will be
allocated with the same parameters, except:
\begin{itemize}
\item If a friction domain is defined in two different groups,
the priority is given to the last group defined,

\item A single friction domain has ever the priority on a group even
if a group with this domain is defined afterwards,

\item If a single friction domain is defined twice, the priority is given to the
last definition.
\end{itemize}

\textbf{V -- Programming}

A new module, \telfile{FRICTION\_DEF}, has been created in order to save the
data read in the friction file.
This module is built on the structure of the \telfile{BIEF} objects.
The domain of friction ''I'' is used as follows:
\begin{lstlisting}[language=TelFortran]
 TYPE(FRICTION_DEF) :: TEST_FRICTION

 TEST_FRICTION%ADR(I)%P
\end{lstlisting}
 The components of the structure are:



\begin{tabular}{|p{3.in}|p{2.5in}|} \hline
\telfile{TEST\_FRICTION\%ADR(I)\%P\%GNUM(1)} & 1$^{\rm{st}}$ code number of the friction domains \\ \hline
\telfile{TEST\_FRICTION\%ADR(I)\%P\%GNUM(2)} & Last code number of the friction domains \\ \hline
\telfile{TEST\_FRICTION\%ADR(I)\%P\%RTYPE(1)} & Law used for the bottom \\ \hline
\telfile{TEST\_FRICTION\%ADR(I)\%P\%RTYPE(2)} & Law used for the boundaries conditions \\ \hline
\telfile{TEST\_FRICTION\%ADR(I)\%P\%RCOEF(1)} & Roughness parameters for the bottom \\ \hline
\telfile{TEST\_FRICTION\%ADR(I)\%P\%RCOEF(2)} & Roughness parameters for the boundaries conditions \\ \hline
\telfile{TEST\_FRICTION\%ADR(I)\%P\%NDEF(1)} & Default Manning for the bottom \\ \hline
\telfile{TEST\_FRICTION\%ADR(I)\%P\%NDEF(2)} & Default Manning for the boundary conditions \\ \hline
\telfile{TEST\_FRICTION\%ADR(I)\%P\%DP} & Diameter of the roughness elements \\ \hline
\telfile{TEST\_FRICTION\%ADR(I)\%P\%SP} & Spacing of the roughness elements \\ \hline
\end{tabular}


\telfile{TEST\_FRICTION\%ADR(I)\%P\%GNUM(1)} and \telfile{TEST\_FRICTION\%ADR(I)\%P\%GNUM(2)}
have the same value if a single friction domain is defined.

\telfile{TEST\_FRICTION\%ADR(I)\%P\%RTYPE(1)} is \telfile{KFROT} when there is only one domain.

\telfile{TEST\_FRICTION\%ADR(I)\%P\%RCOEF(1)} is \telfile{CHESTR} when there is only one domain.



The link between \telemac{2d} and the computation of the friction is done
with the \telfile{FRICTION\_CHOICE} subroutine.
It is used in order to initialize the variables for the option
\telkey{FRICTION DATA} at the beginning of the program and/or in order to call
the right friction subroutine for the computation at each iteration.

\textbf{\underbar{Initializing:}}

During the initialization, the parameters of the friction domains are saved
thanks to the \telfile{FRICTION\_READ} subroutine and the code number of each nodes
are saved thanks to \telfile{FRICTION\_USER} in the array \telfile{KFROPT\%I}.
With the subroutine \telfile{FRICTION\_INIT}, the code numbers for all nodes are
checked and the arrays \telfile{CHESTR\%R} and \telfile{NKFROT\%I}
(\telfile{KFROT} for each node) are built.
\telfile{KFROT} is used in order to know if all friction parameters are null or
not.
This information is used during the computation.

\textbf{\underbar{Computing:}}

For the optimization, the computation of the friction coefficient is done
in the \telfile{FRICTION\_CALC} subroutine for each node thanks to the loop
\telfile{I = N\_START, N\_END}.
When the option \telkey{FRICTION DATA} is not used, \telfile{N\_START} and
\telfile{N\_END} are initialized to 1 and \telfile{NPOIN} in the subroutine
\telfile{FRICTION\_UNIF}.
Else, they take the same value and the loop on the node is done in the
\telfile{FRICTION\_ZONES} subroutine (the parameters used for each node can be
different).

With this choice, the subroutine \telfile{FRICTION\_UNIF} is not optimized
when the option \telkey{NON-SUBMERGED VEGETATION} is called
(\telfile{FRICTION\_LINDNER}).
This option aims to correct the value of the bottom friction coefficient
when there is partial submerged vegetation.

\textbf{VI -- Accuracy}

When the option \telkey{FRICTION DATA} is not used, \telfile{CHESTR} can be read
in the \telkey{GEOMETRY FILE}.
The values stored in this file are in single precision.
However \telfile{CHESTR} is defined in double precision, then,
the \telfile{CHESTR} value is not exactly the right value.

With the option \telkey{FRICTION DATA}, \telfile{CHESTR} is set
thanks to the \telkey{FRICTION DATA FILE} where the value of each domains
are stored in double precision.

Then when a comparison is done between both methods, the difference may come
from the difference between single and double precision.
