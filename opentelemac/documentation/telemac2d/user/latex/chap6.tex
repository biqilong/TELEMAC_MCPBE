\chapter{Physical parameter definition}
\label{ch:phys:param:def}
A number of physical parameters may or must be specified during a simulation.
If the parameter is space-dependent, it is sometimes preferable to define
various zones within the mesh and then assign the parameter as a function of the
zone number.
To do this, it is necessary to activate the logical keyword
\telkey{DEFINITION OF ZONES} and fill the \telfile{USER\_DEF\_ZONES} subroutine
which assigns a zone number to each point.
This zone number may then be used in the various subroutines for specifying
a space-dependent physical parameter.
Another possibility is to fill the \telkey{ZONES FILE}, see Appendix
\ref{tel2d:app5}.

The spatial distribution of some parameters may be specified interactively
using FUDAA-PREPRO (friction coefficient and velocity diffusivity coefficient
especially).


\section{Friction parameter definition}
\label{sec:frict:param}
We describe hereafter the simplest case, when the friction law is the same
in all the computation domain, when it is variable in space, refer to Appendix
\ref{tel2d:app5} after reading this paragraph.
The friction law used to model friction on the bed is defined by the keyword
\telkey{LAW OF BOTTOM FRICTION}.
This may have the following values:

\begin{itemize}
\item 0: No friction,

\item 1: Haaland's law,

\item 2: Ch\'{e}zy's law,

\item 3: Strickler's law,

\item 4: Manning's law,

\item 5: Nikuradse law,

\item 6: Log law of the wall (only for boundary conditions),

\item 7: Colebrooke-White law.
\end{itemize}

Option 6 is only for boundary conditions,

In the case of options 1 to 6, it is necessary to specify the value
of the coefficient corresponding to the law chosen by means of the keyword
\telkey{FRICTION COEFFICIENT}.
This is of course only valid if the friction is constant in time and space.

In the case of option 7, an additional coefficient is needed which can be given
by the keyword
\telkey{MANNING DEFAULT VALUE FOR COLEBROOK-WHITE LAW}
(whose default value = 0.02).

If the friction coefficient varies in time (and possibly in space as well),
it is necessary to use the \telfile{USER\_STRCHE} and/or \telfile{USER\_CORSTR}
subroutines, which supply the friction coefficient at each mesh point.

The following example shows how \telfile{USER\_STRCHE} is programmed for a
domain in which the friction coefficient is 50 for the left part ($X<10000$)
and 55 for the right part.
\begin{lstlisting}[language=TelFortran]
! LOOP ON ALL POINTS OF DOMAIN
 DO I = 1,NPOIN
   IF (X(I) .LT. 10000.D0) THEN
     CHESTR%R (I) = 50.D0
   ELSE
     CHESTR%R (I) = 55.D0
   ENDIF
 ENDDO
\end{lstlisting}
When evaluating the friction term, it is possible to specify which depth is used
for this computation through the keyword \telkey{DEPTH IN FRICTION TERMS}.
The two possibilities are:
\begin{itemize}
\item 1: the classical nodal depth (default value),
\item 2: a depth averaged on the test function area.
\end{itemize}
The second one is available since release 6.0 and seems to be slightly better
on dam break studies.


\subsection{Vegetation friction}

The effect of additional roughness by vegetation can be added 
with the keyword \telkey{VEGETATION FRICTION} (default = NO).

This option can only be used with a definition of friction by domains,
see Appendix \ref{tel2d:app5}. The vegetation laws and their parameters
must be given in an extra file. Currently 8 vegetation laws are implemented. 
Please find the details of the vegetation laws there description and application range in 
\cite{folke2019vegetation} and the cited original literature. 

In case of unsteady flow conditions which lead to 
submerged and non-submerged vegetation 
the vegetation law of Baptist is recommended. 
In order to take the flexibility of vegetation 
into account the approach of Jaervelae is recommended.
The Lindner approach is not recommended as it can increase the overall computing
time by approx. 20~\% because of the iteration of the drag coefficient. 
This is only useful if the shape of the vegetation is cylindrical, 
regularly arranged, non-submerged and 
diameter and spacing are well known.

\subsection{Wave friction enhancement}

Combined wave and current flows influence the resultant bed shear stress due to
additional mass transport~\cite{OConnorYoo1988}.
A proper consideration of the wave--current interaction effects on the friction
coefficient can be accounted by activating with the keyword
\telkey{WAVE ENHANCED FRICTION FACTOR} (default = NO).
This feature is only possible by coupling \telemac{2d} with \tomawac.


\subsection{Sidewall friction}

By default, \telemac{2D} ignores friction phenomena on the solid boundaries
of the model (sidewall).
This consideration may still be enabled using the keyword
\telkey{LAW OF FRICTION ON LATERAL BOUNDARIES} (default = 0 i.e. no friction).
This keyword offers the same option than the keyword
\telkey{LAW OF BOTTOM FRICTION} described above.
The coefficient of friction to take into account is then provided
by the keyword \telkey{FRICTION COEFFICIENT FOR LATERAL SOLID BOUNDARIES}
(default value = 60!) if it is constant on all solid boundaries.
If this value varies spatially, the user can fill the column \telfile{AUBOR}
in the \telkey{BOUNDARY CONDITIONS FILE} (see Section \ref{sub:bc:file}).
\telfile{AUBOR} is then considered as a quadratic coefficient whose
interpretation depends on the chosen friction law.

The friction is activated using the keyword
\telkey{TURBULENCE REGIME FOR SOLID BOUNDARIES}.
The available options are:

\begin{itemize}
\item 1: smooth regime,

\item 2: rough (default value).
\end{itemize}

That option changes the formulation of the velocity profile and consequently,
the friction velocity. See previous section for more information.


\section{Modelling of turbulence}
\label{sec:mod:turbul}
Whether or not the diffusion of velocity is taken into account is determined
by the logical keyword \telkey{DIFFUSION OF VELOCITY} (default value = YES).

The modelling of turbulence is a delicate problem.
\telemac{2D} offers the user six options of different complexity by setting
the keyword \telkey{TURBULENCE MODEL}:
\begin{itemize}
\item 1: a constant viscosity coefficient.
In this case, the coefficient represents the molecular viscosity,
turbulent viscosity and dispersion,

\item 2: an Elder model,

\item 3: a $k$-$\epsilon$ model.
This is a 2D model that solves the transport equations for $k$ (turbulent energy)
and $\epsilon$ (turbulent dissipation).
The model equations are solved by a fractional step method, with convection of
turbulent variables being processed at the same time as the hydrodynamic
variables, and the other terms relating to the diffusion and
production/dissipation of turbulent values being processed in a single step.
The use of the $k$-$\epsilon$ model also often requires a finer mesh
than the constant viscosity model and in this way increases computational time,

\item 4: a Smagorinski model, generally used for maritime
domains with large-scale eddy phenomena,

\item 5: a mixing length model,

\item 6: a Spalart-Allmaras model.
This is a 2D model that solves a transport equation for $\tilde{\nu}$
(a derivated turbulent viscosity).

\end{itemize}

More detailed information on the formulation of the $k$-$\epsilon$ model,
the Elder model, the Smagorinski model, the mixing length model and the
Spalart-Allmaras model can be found in the literature.

In addition, \telemac{2D} offers two possibilities for processing the diffusion
term.
The option is selected by the keyword
\telkey{OPTION FOR THE DIFFUSION OF VELOCITIES}
which can take the value 1 (default) or 2.
The first value selects a computation with the form
div$\left({\vartheta }_t\overrightarrow{grad}\left(U\right)\right)$,
and the second one with the form
$\frac{1}{h}\rm{div}\left({h\vartheta_t }\overrightarrow{grad}\left(U\right)\right)$.

This latter option is the only one offering good mass conservation,
but difficulties may occur with tidal flats.


\subsection{Constant viscosity}

The first possibility is activated by giving the keyword
\telkey{TURBULENCE MODEL} the value 1 (default value).
Turbulent viscosity is then constant throughout the domain.
The overall viscosity coefficient (molecular + turbulent viscosity) is provided
with the keyword \telkey{VELOCITY DIFFUSIVITY} which has a default value of
10$^{-6}$~m$^2$/s (corresponding to the molecular viscosity of water).

The value of this coefficient has a definite effect on the extent and shape of
recirculation.
A low value will tend to dissipate only small eddies, whereas a high value will
tend to dissipate large recirculations.
The user must therefore choose this value with care, depending on the case
treated (in particular as a function of the size of the recirculation he or she
wishes to dissipate and the mean angular velocity of the recirculation).
It should also be noted that a value which results in the dissipation of eddies
smaller than two mesh cells has virtually no effect on the computation.

\telemac{2D} makes it possibile to have a coefficient that varies in time and
space.
This is defined in the \telfile{CORVIS} subroutine.
This subroutine gives information on the geometry and basic hydrodynamic
variables (water depth, velocity components) and time.


\subsection{Elder model}

This option is used when the keyword \telkey{TURBULENCE MODEL} is set to 2.

The Elder model offers the possibility of specifying different viscosity values
along and across the current ($K_l$ and $K_t$ respectively).
The formulae used are:

\begin{equation}
  K_l = a_l U^* h  \textrm{~~~and~~~}  K_t = a_t U^* h
\end{equation}

where:

$U^*$ is the friction velocity (m/s) and $h$ the water depth (m),
$a_l$ and $a_t$ are the dimensionless dispersion coefficients equal to 6 and 0.6
respectively.
Other values can be found in the literature.

The two coefficients can be supplied by the user with the keyword
\telkey{NON-DIMENSIONAL DISPERSION COEFFICIENTS} (format: Kl,Kt).


\subsection{$k$-$\epsilon$ model}

If constant viscosity is not sufficient, \telemac{2D} offers the possibility
of using a $k$-$\epsilon$ model.
This is activated by assigning a value of 3 to the keyword
\telkey{TURBULENCE MODEL}.

In this case, the keyword \telkey{VELOCITY DIFFUSIVITY} has its real physical
value ($10^{-6}$~m$^2$/s for molecular diffusion of water), as this is used as
such by the turbulence model.

In the case of a solid boundary, the user may configure the turbulence regime
for the walls using the keyword \telkey{TURBULENCE REGIME FOR SOLID BOUNDARIES}.
If friction at the wall is not to be taken into account, the user must use the
value corresponding to a smooth wall (option 1).
In contrast, friction will be taken into account by using option 2 (rough wall).
In this case, the friction law used for the wall is the same as the bottom
friction law (keyword \telkey{LAW OF BOTTOM FRICTION}).
The friction coefficient is then supplied by the keyword
\telkey{ROUGHNESS COEFFICIENT OF BOUNDARIES} (default value is 100).
This numerical value must of course be in agreement with the chosen law,
in appropriate units.

If a $k$-$\epsilon$ model is used, the information concerning the solution phase
must be obtained by activating the keyword
\telkey{INFORMATION ABOUT K-EPSILON MODEL} (default = YES).

Parameter definition for the $k$-$\epsilon$ model is described in Chapter
\ref{ch:num:par:def}.

A good level of expertise in turbulence is necessary to use the $k$-$\epsilon$
model, especially to know when it is relevant to resort to it.
As a matter of fact the turbulence should be larger than the dispersion terms.
We quote here W. Rodi:
``It should be emphasized that the model described here does not account
for the dispersion terms appearing in the (depth-averaged) momentum equations''.


\subsection{Smagorinski model}

The use of this model is activated by assigning a value of 4 to keyword
\telkey{TURBULENCE MODEL}.
Same remark as the $k$-$\epsilon$ model, it does not take into account the
dispersion terms (see \cite{Smagorinsky1963} for more details).


\subsection{Mixing length model}

The use of this model is activated by giving the value 5 to keyword
\telkey{TURBULENCE MODEL}.
A description of this model, implementation and validation can be found in
\cite{Dorfmann2016}.
It is a combination of the depth-averaged parabolic eddy viscosity model
with the Prandl's mixing length theory for the horizontal in order to account
both the vertical and horizontal turbulence production.
The calibration coefficients $C_l$ and $\alpha_t$ used in the
\telfile{MIXLENGTH} subroutine may be changed by the means of the
\telkey{MIXING LENGTH MODEL COEFFICIENTS} which is an array of 2 values
in the same order: $C_l$ ; $\alpha_t$.
Default values are $C_l = 0.1066667$ ($C_l = \frac{4 \kappa}{15}$) and
$\alpha_t$ = 0.0666667 ($\alpha_t = \frac{\kappa}{6}$)
with $\kappa$ the Von Karman constant set by default at 0.4 in \telemac{2d}.

Warning: until release 8.1, the $C_l$ default value was hard-coded in the
\telfile{MIXLENGTH} subroutine at $\frac{4}{15}$ without the $\kappa$ factor.

\subsection{Spalart-Allmaras model}

The use of this model is activated by giving the value 6 to keyword
\telkey{TURBULENCE MODEL}.
If a Spalart-Allmaras model is used, the information concerning
the solution phase can be obtained by activating the keyword
\telkey{INFORMATION ABOUT SPALART-ALLMARAS MODEL} (default = YES).

Parameter definition for the Spalart-Allmaras model is described in Chapter
\ref{ch:num:par:def}.
Some parameters are common with the $k-\epsilon$ model's ones.


\section{Setting up of meteorological phenomena}
\label{sec:param:met:phen}

\subsection{Wind influence}

\telemac{2D} can be used to simulate flow while taking into account the
influence of a wind blowing on the water surface.
The force induced by wind is considered in the same way as the friction effect
on the bottom.
The following force is consequentely added to the right handside term of the
momentum equation:
\begin{equation}
F_x = \frac{1}{h} \frac{\rho_{air}}{\rho_{water}}a_{wind}U_{wind}\sqrt{U^2_{wind}+V^2_{wind}},
\end{equation}
\begin{equation}
F_y = \frac{1}{h} \frac{\rho_{air}}{\rho_{water}}a_{wind}V_{wind}\sqrt{U^2_{wind}+V^2_{wind}}.
\end{equation}
This expression does not consider the wind influence as drag force,
like it is the case commonly.
In order to retrieve a drag-like expression, the reader should write this force
in the following form:
$F_x = \frac{1}{2}\rho_{air}C_d U^2 A$ where $C_d$ is the drag coefficient,
$A$ the effective area and $U$ is the velocity magnitude of the wind.
The logical keyword \telkey{WIND} (default = NO, i.e. no wind) is used first of
all for determining whether this influence is to be taken into account.
If so, the coefficient is then provided with the keyword
\telkey{COEFFICIENT OF WIND INFLUENCE} (see below) or automatically calculated
if \telkey{COEFFICIENT OF WIND INFLUENCE VARYING WITH WIND SPEED} = YES
which is the default value since release 8.2.
Since release 7.0, wind effect is managed using a new keyword,
\telkey{OPTION FOR WIND} (default = 0): this keyword can have the following
values:

\begin{itemize}
\item 0: this means no wind effect (this is equivalent to put the
 keyword \telkey{WIND} to FALSE),

\item 1: wind is constant in time and space, wind speed in directions
$x$ and $y$ are supplied with the keywords \telkey{WIND VELOCITY ALONG X} and
\telkey{WIND VELOCITY ALONG Y} (default values = 0.),
or through the keyword \telkey{SPEED AND DIRECTION OF WIND} (default = 0.;0.)
which gives the speed (in m/s) and the direction (in degrees from 0 to 360) of
the wind,

\item 2: wind variable in time, constant is space,
it is given through the formatted file \telkey{ASCII ATMOSPHERIC DATA FILE},

\item 3: wind is variable in time and space, this option is not
implemented as there are multiple choices of implementation.
In this case, the user must program him/herself the \telfile{METEO} subroutine.
An example of implementation is given in validation test case ``wind\_txy''
(in folder examples/telemac2d/wind\_txy).
\end{itemize}

The coefficient of wind influence hides complex phenomena.
In fact, the influence of the wind depends on the smoothness (or, lack of it)
of the free surface and the distance over which it acts (called the ``fetch'').
The coefficient value can be obtained from many different formulas.

This is the formula used by the Institute of Oceanographic Sciences
(United Kingdom):

\begin{table}[h!]
\begin{center}
\begin{tabular}{ll}

if $\| \vec{U}_{wind} \| < 5$ m/s & $a_{wind}  = 0.565 \times 10^{-3}$ \\
if $5 < \| \vec{U}_{wind} \| < 19.22$ m/s &
 $a_{wind} = (- 0.12 + 0.137 \| \vec{U}_{wind} \| ) 10^{-3}$ \\
if $\| \vec{U}_{wind} \| > 19.22$ m/s & $a_{wind} = 2.513 \times 10^{-3}$ \\
\end{tabular}
\end{center}
\end{table}

This formula can be activated by setting the keyword
\telkey{COEFFICIENT OF WIND INFLUENCE VARYING WITH WIND SPEED} to YES
(default value is YES since release 8.2).
If YES, the value of \telkey{COEFFICIENT OF WIND INFLUENCE} is overwritten and
the coefficient is automatically computed depending on the wind velocity.

If the previous keyword is set to NO (no automatic computation), the parameter
\telkey{COEFFICIENT OF WIND INFLUENCE} asked for by \telemac{2D} is:
$a_{wind} \frac{\rho_{air}}{\rho}$ and not only $a_{wind}$.

$\rho_{air}$ is approximately 1.2 kg/m$^3$ and $\rho$ is 1,000 kg/m$^3$.
Thus it is necessary to divide the value of a$_{wind}$ by 1,000
to obtain the value of the \telemac{2D} keyword.
The default value of \telkey{COEFFICIENT OF WIND INFLUENCE} has been set to
$1.55 \times 10^{-6}$ since release 8.2.

If there are tidal flats or dry zones in the domain, the wind may trigger
unphysical velocities as it becomes the only driving term in the equations.
To avoid this, the influence of the wind is cancelled below a threshold value
of depth, with the keyword \telkey{THRESHOLD DEPTH FOR WIND}
(default value = 1~m).
Be careful if the model includes shallow waters, lower than this value.

\subsection{Atmospheric pressure}

Atmospheric pressure is taken into account by setting the keyword
\telkey{AIR PRESSURE} to YES (the default value is NO).
Since release 7.0, the pressure value is set in the \telfile{METEO}
subroutine by the keyword \telkey{VALUE OF ATMOSPHERIC PRESSURE}
(default 10$^5$ Pa).
By default, the latter initializes a pressure of $10^5$ Pa ($\approx$ 1 atm)
over the whole domain.

The \telfile{METEO} subroutine is called if the wind or atmospheric pressure
or water quality options are activated.
By default, the subroutine is called only at the beginning of the computation
(time value = 0) in order to set the wanted pressure throughout the domain
and the wind speed at the values provided by the corresponding keywords.
The user has geometrical information on the mesh, and as well as time
information for programming any case, in particular winds that may vary in time
and space (in this case, a test must be programmed for time values other than 0).

The following example shows a wind programmed in space and in time.
For the left part of the domain ($ X < 1,000,000$~m) the wind in direction $x$
is fixed at 10~m/s for the first 3,600~s, and at 5~m/s subsequently.
The $x$ and $y$ wind components in the right part of the domain are nil.

\begin{lstlisting}[language=TelFortran]
! INITIALISATION WIND Y AND WIND X FOR LT=0
 IF (LT.EQ.0) THEN
   CALL OV ('X=C     ',WINDX, Y, Z, 0.D0, NPOIN)
   CALL OV ('X=C     ',WINDY, Y, Z, 0.D0, NPOIN)
 ELSE
! INITIALISATION WINDX LEFT PART FOR NON-ZERO TIMES
   DO I=1,NPOIN
     IF (X(I).LT.1000000.D0) THEN
       IF (LT.LT.3600.D0) THEN
         WINDX (I) = 10.D0
       ELSE
         WINDY (I) = 5.D0
       ENDIF
     ENDIF
   ENDDO
 ENDIF
\end{lstlisting}

\subsection{Rain and evaporation}

The modelling of the influence of precipitation or evaporation is activated
with the logical keyword \telkey{RAIN OR EVAPORATION} (default value = NO).
The value of the contribution or the loss of water at the surface is specified
using the keyword \telkey{RAIN OR EVAPORATION IN MM PER DAY} which default value
is 0. (a negative value reflects an evaporation).
The duration of the rainfall or evaporation can be set with the keyword
\telkey{DURATION OF RAIN OR EVAPORATION IN HOURS} (units: hours, default is
infinite).
Rain and evaporation can also vary in time and space.
They can be introduced through the \telkey{ASCII ATMOSPHERIC DATA FILE}
or \telkey{BINARY ATMOSPHERIC DATA FILE}.
See water quality, wind and rain validation test cases (in folder
examples/telemac2d).

In case of calculation with consideration of tracers, it is possible to specify
the contribution related to the rain with the keyword
\telkey{VALUES OF TRACERS IN THE RAIN} (default value is 0.).
It is important to note that, in the case of evaporation, no tracer is taken
into account in the water loss, which is incorrect if the tracer is the
temperature.

\subsection{Rainfall-runoff Modelling}

Rainfall-runoff can be modelled with the Curve Number runoff model developed
by USA Soil Conservation Service \cite{soil1972national}. It takes into 
account infiltration processes to compute the effective rain induced runoff.
Runoff potential is defined by a unique parameter called the Curve Number (CN)
which is function of hydrological soil groups, land use, hydrologic surface
condition of native pasture and antecedent moisture conditions.
See Applied Hydrology \citep{chow1988mays} 
for more details and typical CN values. 

The Curve Number runoff model is activated when the keyword
\telkey{RAINFALL-RUNOFF MODEL} is set to 1 (default is 0, no runoff model).
The keyword \telkey{TIDAL FLATS} must be set to YES (which is default value).
The CN values should be spatially defined.
Two methods are available: coordinates of polygons with constant CN values
given in a formatted data file (\telkey{FORMATTED DATA FILE 2}, default)
or CN values stocked directly in the \telkey{GEOMETRY FILE} as
an additional variable (see validation case pluie in folder examples/telemac2d).

The antecedent moisture conditions class can be defined with the keyword
\telkey{ANTECEDENT MOISTURE CONDITIONS} (1: dry, 2: normal [default], 3: wet).
Two options regarding the definition of the initial abstraction ratio are
available
and can be set with the keyword \telkey{OPTION FOR INITIAL ABSTRACTION RATIO}
(1: original method with $\lambda$ = 0.2 [default];
 2: revised method with $\lambda$ = 0.05 and automatic conversion of CN values).
CN values to be given in input must correspond to the original method
(initial abstraction ratio $\lambda$ = 0.2) and to normal antecedent moisture
conditions. See \cite{woodward2003runoff} for more details about the initial
abstraction ratio.

Other options can be activated manually in the \telfile{RUNOFF\_SCS\_CN}
subroutine (in folder sources/telemac2d):
\begin{itemize}
\item Correction of CN values to account for steep slopes,
\item Rainfall defined as a so-called CDS-type hyetograph (Chicago Design
Storm) based on a three-parameter Intensity-Duration-Frequency equation
(constant in space),
\item Rainfall defined as a block-type hyetograph giving the rainfall depth
in mm between two consecutive times provided in a formatted data file (constant
in space).
\end{itemize}
Several examples of use are provided in the validation test case pluie
(in folder examples/telemac2d).

Evaporation is not supported.


\section{Astral potential}
\label{sec:astral:pot}
When modelling large maritime areas, it is sometimes necessary to take into
account the effect of astral forces generating tide inside the area.
For this, the user has several keywords at his disposal.

First of all, the logical keyword \telkey{TIDE GENERATING FORCE}
(default value = NO) allows these phenomena to be taken into account.
If YES, the keyword \telkey{SPHERICAL COORDINATES} has to be activated,
it is impossible to account tide generating force in cartesian coordinates.

The keyword \telkey{LONGITUDE OF ORIGIN POINT} must be positioned at the right
value (default = 0 degree).

Lastly, the two keywords \telkey{ORIGINAL DATE OF TIME} (format YYYY;MM;DD)
and \telkey{ORIGINAL HOUR OF TIME} (format HH;MM;SS) must be used
to give the beginning time of the simulation.
This information is necessary for \telemac{2D} to compute the respective
position of the moon and the sun.


\section{Wave induced currents}

We describe here the chaining procedure.
A more dynamic solution, coupling, is described in section \ref{sec:coupling}
and should be preferred.

It is possible to include wave-induced currents by recovering the information
calculated by the wave propagation modules (mainly \tomawac but also possible
with \artemis).
In the present state of the system, only a steady state can be taken into
account.
The procedure is as follows:

\begin{itemize}
\item Run a wave propagation calculation on the same mesh as the \telemac{2D}
calculation, asking for the driving forces to be stored.
In the case of \tomawac, these are the variables \telfile{FX} and \telfile{FY},

\item Recover the wave results file and specify its name using the keyword
\telkey{BINARY DATA FILE 1},

\item Activate the keyword \telkey{WAVE DRIVEN CURRENTS} (default value = NO),

\item Complete the keyword \telkey{RECORD NUMBER IN WAVE FILE}
(default value = 1).
This value corresponds to the iteration number stored in the wave file that must
be taken into account by \telemac{2D}. This is usually the last iteration stored.
\end{itemize}

If the user wishes to take into account several results from the wave
propagation module again (e.g. in order to take into account changes in sea
level), FORTRAN programming is required.\\

The user can also have variables stored, not used by \telemac{2d} but used
when coupling \telemac{2d} with another code.
Thus the stored variables belong to the other code and are given back
in the \telkey{RESULTS FILE}.
To do this, the user can set the \telkey{NAMES OF CLANDESTINE VARIABLES}
keyword and implement what he/she wants to do.


\section{Vertical structures}

It may be necessary to take into account the presence of an obstacle to flow,
such as bridge piers, trees or even buildings, without having to model them in
the mesh, especially as, in this case, the force opposing the flow generally
varies with the depth of water.

To handle this problem, \telemac{2D} can include drag forces connected with the
presence of vertical structures in the model.
This function is activated with the logical keyword \telkey{VERTICAL STRUCTURES}
(default = NO).

The drag forces must then be defined in the \telfile{DRAGFO} user subroutine.
An example of programming is given in the subroutine itself.


\section{Other physical parameters}

When modelling large areas, it is necessary to take into account the inertia
effect of the Coriolis force.
This is done by activating the logical keyword \telkey{CORIOLIS} (which is set
to NO by default).
In such case, the value of the Coriolis coefficient (see Formulation Document)
is defined by the keyword \telkey{CORIOLIS COEFFICIENT} (default value = 0.).
This must be calculated according to the latitude $\lambda$ through the formula:

 FCOR = 2 $\omega$ sin($\lambda$) where $\omega$ is the angular velocity of the
Earth, equal to 7.2921 $\times$ 10$^{-5}$~rad/s.

The components of the Coriolis force are thus:

FU = FCOR $\times$ V         and         FV = -FCOR $\times$ U

In the case of very large domains such as portions of oceans,
it is necessary to carry out a simulation with spherical coordinates,
in which case the Coriolis coefficient is adjusted automatically at each point
of the domain by activating the keyword \telkey{SPHERICAL COORDINATES}
(see \ref{sec:spher:coord:LATI}).
Its default value is NO.

\telemac{2D} also offers the opportunity of defining the water density
with the keyword \telkey{WATER DENSITY}.
Its default value is 1,000~kg/m$^3$, i.e. a value corresponding to a fresh river
water.

Gravity acceleration can be changed with the keyword \telkey{GRAVITY ACCELERATION}
whose default value is set at 9.81 m/s$^2$.


\section{Tsunami generation}

\telemac{2d} can model tsunami generation by computing the free surface
displacement according to Okada model (1992), assuming it is similar to that
of the seabed.
It is treated as an initial condition on water depth, depending on the
location given by the keywords
\telkey{LONGITUDE OF ORIGIN POINT} and \telkey{LATITUDE OF ORIGIN POINT}.
Tsunami generation can be activated with the keyword
\telkey{OPTION FOR TSUNAMI GENERATION}:
\begin{itemize}
\item 1: no tsunami (default value),
\item 2: tsunami generated on the basis of the Okada model (1992).
\end{itemize}

The main physical characteristics of the tsunami can be set by the keyword
\telkey{PHYSICAL CHARACTERISTICS OF THE TSUNAMI} which is an array of
10 values (in the order):
\begin{itemize}
\item $HH$ focal depth (in m),
\item $L$ fault length (in m),
\item $W$ fault width (in m),
\item $D$ dislocation (in m),
\item $TH$ strike direction (in decimal degrees),
\item $DL$ dip angle (in decimal degrees),
\item $RD$ slip angle (in decimal degrees),
\item $Y0$ epicentre latitude (in decimal degrees),
\item $X0$ epicentre longitude (in decimal degrees),
\item $C0$ size of the ellipse of influence ($L$ $\times$ $W$).
\end{itemize}
Default values for this keyword are:
(100.;210000.;75000.;13.6;81.;41.;110.;0.;0.;3.).


\section{Parameter estimation}

\telemac{2D} contains a function for automatically determining an unknown
physical parameter.
In the current release of the software, it is only possible to determine the
friction coefficient when using the Strickler or Ch\'{e}zy laws (keyword
\telkey{LAW OF BOTTOM FRICTION} with value of 2 or 3).

The principle for determining a parameter involves performing a series of
calculations and comparing the results provided by \telemac{2D} with the
available measurements.
The parameter to be determined is then adjusted in order to obtain identical
values.

The algorithm for estimating this parameter is activated with the keyword
\telkey{PARAMETER ESTIMATION}, which provides the name of the parameter to be
determined.
The user can specify `FRICTION' or `FRICTION, STEADY'.
In the second configuration, only the last time step of the simulation is
checked.
In the current release of the software, it is strongly recommended to work only
in permanent mode.

Measurement data are supplied via the \telfile{USER\_MESURES} user subroutine
which contains the arguments \telfile{ITER} (iteration number) and \telfile{TT}
(absolute time).
The latter argument is used in processing real measurements.
Each time the \telfile{USER\_MESURES} subroutine is called up,
it must supply the measured water depth (\telfile{HD}),
the two velocity components (\telfile{UD} and \telfile{VD}),
as well as the weightings \telfile{ALPHA1, ALPHA2} and \telfile{ALPHA3}
connected respectively with \telfile{HD, UD, VD}.
The weighting is 1 if a measurement is available and 0 if it is not.
For example, an \telfile{ALPHA1} value of 1 for a given point means
that a depth measurement is available for that point.
Similarly, an \telfile{ALPHA3} of 1 for a given point means that a velocity
measurement \telfile{V} is available for that point.
When a measurement is available, it may be advisable to replace the value 1
by a vector proportional to the local mesh size
(see \telfile{VECTOR(`MASBAS',{\dots}}) in the \telfile{USER\_MESURES}
subroutine).

The comparison data may also be provided by a file in SERAFIN format,
in which case the name is specified with the keyword \telkey{REFERENCE FILE}.
The data are read automatically in this case.

If the parameter is space-dependent, it is necessary to activate the logical
keyword \telkey{DEFINITION OF ZONES} (default value = NO)
and to complete the \telfile{USER\_DEF\_ZONES} subroutine,
which assigns a zone number to each point.
In this case, a parameter value will be estimated for each zone.
This value will be constant within the zone.

From the numerical point of view, the user must specify a number of parameters.

The cost function used must be indicated with the integer keyword
\telkey{COST FUNCTION} which may have the value 1 (cost function based on the
difference between depths and velocities, which is the default value)
or 2 (cost function based on the difference between celerities and velocities).
2 seems to be preferable, even though the effect of this parameter is slight.
Anyway, \telkey{COST FUNCTION} must be 1 with tidal flats.

The integer keyword \telkey{IDENTIFICATION METHOD} is used to specify
the technique used for minimizing the cost function.
It may have the value 1 (gradient, which is the default value), 2 (conjugate
gradient) or 3 (Lagrangian interpolation).

As parameter estimation is based on an iterative procedure, it is necessary to
specify the required level of accuracy and a maximum number of iterations.
Accuracy is indicated with the keyword \telkey{TOLERANCES FOR IDENTIFICATION}.
This is an array of four integers corresponding respectively to the absolute
accuracy for the depth, velocity $u$ and velocity $v$, and the relative accuracy
of the cost function (default values = (1.E-3; 1.E-3; 1.E-3; 1.E-4)).
The iteration process is stopped when the absolute accuracy levels are reached
or when the relative accuracy for the cost function is.
The maximum number of iterations is specified with the keyword
\telkey{MAXIMUM NUMBER OF ITERATIONS FOR IDENTIFICATION} which has the default
value 20.
As each iteration corresponds to two simulations,
the value of this keyword should not be too high.

The results of estimating this parameter are provided in the
\telkey{RESULTS FILE}.
This is a geometry file in which the \telfile{FRICTION} variable has been added.
The file can thus be re-used as a \telkey{GEOMETRY FILE} for a new simulation.
