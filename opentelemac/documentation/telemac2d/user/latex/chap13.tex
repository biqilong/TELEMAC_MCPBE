\chapter{Construction works modelling}
\label{ch:constr:wm}

\section{Weirs}
\label{sec:weirs}
Weirs are considered as linear singularities.
Their use is possible in parallel computing (since release 6.2).
The number of weirs is specified by the keyword \telkey{NUMBER OF WEIRS}
(default value 0).
Information about weirs is given in the \telkey{WEIRS DATA FILE}.

A weir must be prepared in the mesh and consists of two boundary lines which are
actually linked by the weir.
In principle, these boundaries should be sufficiently far apart,
upstream and downstream of the weir.
The upstream and downstream boundary points should correspond 1 to 1,
and the distance between two points should be the same on both sides.
The following file gives an example of two weirs (the comments are part of the
file):
\begin{lstlisting}[language=bash]
Nb of weirs     Option for tangential velocity
      2                    0
---------------------------- singularity 1
Nb of points for 1 side
11
Points side 1
71 72 73 74 75 76 77 78 79 80 41
Points side 2
21 20 19 18 17 16 15 14 13 12 11
Level of the dyke
1.8 1.8 1.8 1.8 1.8 1.8 1.8 1.8 1.8 1.8 1.8
Flowrate coefficients
 .4  .4  .4  .4  .4  .4  .4  .4  .4  .4  .4
---------------------------- singularity 2
Nb of points
11
Points side 1
111 112 113 114 115 116 117 118 119 120 81
Points side 2
61 60 59 58 57 56 55 54 53 52 51
Level of the dyke
1.6 1.6 1.6 1.6 1.6 1.6 1.6 1.6 1.6 1.6 1.6
Flowrate coefficient
 .4  .4  .4  .4  .4  .4  .4  .4  .4  .4  .4
\end{lstlisting}

Line 2 indicates the number of weirs and then an option for the treatment
of tangential velocities on the weir, with the following meaning:

\begin{itemize}
\item 0: the velocities are null (recommended option),

\item 1: the velocities will be calculated with the Ch\'{e}zy
formula (as a function of the local free surface slope).
\end{itemize}

For each weir, it is then necessary to indicate:
the number of points for the first side of the weir (line 5 for the first weir)
and the list of their global numbers (line 7 for the first weir).
Note, that before and for release 6.1, the numbering to provide was not the
global one but the local numbering of the boundary defined in the
\telkey{BOUNDARY CONDITIONS FILE}.
However, it is necessary to provide the weirs number in the order of the
boundary points.

The numbers of their twin points on side 2 should be given on line 9 in the
reverse order.
On line 11, the level of the weir is specified for each couple of points
and at line 13 the discharge coefficient noted m.
All these data are repeated for all weirs.

The formulae used to calculate the discharge for each point are the following:

\begin{itemize}
\item unsubmerged weir: $Q=\mu \sqrt{2g}\ {\left(upstream-weir\right)}^{\frac{3}{2}}$,

\item submerged weir:
\[Q=\ {\left(\frac{2}{3}\sqrt{\frac{1}{3}}\right)}^{-1}\mu \sqrt{2g}
\left(downstream-weir\right)\sqrt{\left(upstream-weir\right)},\]

\item the weir is not submerged if:
\[upstream\ level<\frac{weir\ level+2 \times upstream\ level}{3}.\]
\end{itemize}
Depending on the shape and roughness of the weir, the value of $\mu$
is between 0.4 and 0.5.
However, the above formulae neglect the velocity of the upstream head in the
computation.
If this is not the case, the value of $\mu$ may be higher.

If the user wants to modify the different laws, it is possible to modify the
appropriate subroutines (\telfile{LOIDEN} and \telfile{LOINOY}).\\

The keyword \telkey{TYPE OF WEIRS} gives the method to treat weirs.
2 options are available:
\begin{itemize}
\item horizontal with same number of nodes upstream/downstream
(Historical solution with the \telfile{BORD} subroutine,
which is the default value),
\item general (new solution with sources points).
\end{itemize}


\section{Culverts}
\label{sec:culverts}
As for weirs, the keyword \telkey{NUMBER OF CULVERTS} (default value = 0)
specifies the number of culverts to be treated.
Culverts are described as couples of points between which flow may occur,
as a function of the respective water level at these points.
Since release 6.2 of \telemac{2d}, it is no longer necessary to describe
each culvert inflow and outflow as a source point.

There are two options to treat culverts in \telemac{2d}.
The choice can be done with \telkey{OPTION FOR CULVERTS}
(default value = 1).
For more information about this choice, the reader is invited to refer to the
\telemac{3d} theory guide.\\

Information about culvert characteristics is stored in the
\telkey{CULVERTS DATA FILE}.

The following file gives an example of a culvert:
\begin{lstlisting}[language=bash]
Relaxation, Number of culverts
0.2 1
I1  I2  CE1 CE2 CS1 CS2 LRG  HAUT1 CLP LBUS Z1  Z2  CV  C56 CV5 C5  CT  HAUT2 FRIC LENGTH CIRC D1  D2 A1 A2  AA
199 640 0.5 0.5 10  1.0 2.52 2.52  0   0.2  0.3 0.1 0.0 0.0 0.0 0.0 0.0 2.52  0.0  0.0    1    90. 0. 0. 90. 0

\end{lstlisting}

The relaxation coefficient is initially used to prescribe the discharge
in the culvert on a progressive basis in order to avoid the formation of an eddy.
Relaxation, at time $T$, between result computed at time $T$ and result
computed at previous time step.
A relaxation coefficient of 0.2 means that 20\% of time $T$ result is mixed
with 80\% of the previous result.
I1 and I2 are the numbers of each end of the culvert in the global point
numbering system.

The culvert discharge is calculated based on the formulae given
in the \telemac{3d} theory guide and in the release notes of \telemac{2d}:
CE1 and CE2 are the head loss coefficients of 1 and 2 when they are operating as
inlets.
CS1 and CS2 are the head loss coefficients of 1 and 2 when they are operating as
outlets.
LRG is the width of the culvert.
HAUT1 and HAUT2 are the heights of the construction work (in meters)
at the inlet and outlet.
The flow direction is also imposed through the keyword CLP:\\
CLP = 0, flow is allowed in both directions,\\
CLP = 1, flow is only allowed from section 1 to section 2,\\
CLP = 2, flow is only allowed from section 2 to section 1,\\
CLP = 3, no flow allowed.\\
LBUS is the linear head loss in the culvert, generally equal to
$\lambda {\kern 1pt} {\kern 1pt} {\kern 1pt} {\kern 1pt} {\kern 1pt} \frac{L}{D} $
where $L$ is the length of the pipe, $D$ its diameter and $l$ the friction
coefficient.
Z1 and Z2 are the levels of the inlet and outlet.
CV refers to the loss coefficient due to the presence of a valve and
C56 is the constant used to differentiate flow types 5 and 6 in the formulation
by Bodhaine.
C5 and CV5 represent correction coefficients to C1 and to CV coefficients 
due to the occurrence of the type 5 flow in the Bodhaine formulation.
CT is the loss coefficient due to the presence of trash screens.
FRIC is the Manning Strikler coefficient.
LENGTH is the length of the culvert, and the culvert's shape can be specified
through the parameter CIRC (equal to 1 in 
case of a circular section, 0 for a rectangular section).
A1 and A2 are the angles with respect to the $x$ axis. 
D1 and D2 are the angles that the pipe makes with respect to the bottom,
in degrees.
For a vertical intake, the angle with the bottom will therefore be 90$^\circ$.
They are used to account for the current direction at source or sink point.
AA is a parameter which allows the user to choose whether A1 and A2 are
automatically computed by \telemac{2d}
or whether the data file values are used to set these angles:
AA=1 -- automatic angle; AA=0 -- user-set angle.


\section{Dykes breaches}
\label{sec:dykes}
\telemac{2d} allows simulating dykes breaching by suddenly or gradually
lowering the altitude of some points.
This feature is enabled using the logical keyword \telkey{BREACH} (default = NO).
The description of the breaching process is provided in the file specified
by the keyword \telkey{BREACHES DATA FILE}.

In the current release, 3 types of breaching process are available:

\begin{itemize}
\item at a given time,

\item when the water level above the dyke reaches a given value,

\item when the water level at a given point reaches a certain value.
\end{itemize}

The breaching zone is defined by a polyline of several points associated to a
bandwidth.
The final situation is characterized by a bottom altitude that will be reached
by all the points located in the breaching zone.
If after the dyke breaching, the bottom level is not constant, it is thus
necessary to divide the dyke into several breaching polylines.

Since release 7.0, it is possible to take into account a lateral growth
of the breach (dyke opening by widening).
Old breaching processes are not affected by this new feature.
However, the \telkey{BREACHES DATA FILE} is modified as follows:

\begin{itemize}
\item addition of two new lines for selecting breach opening option.
These two lines -- comment line and the value for the option -- come after
the breach duration. The options are selected using:
\begin{itemize}
\item 1: for dyke opening by bottom lowering (the old implementation),

\item 2: for dyke opening by widening (the newly added option),

\end{itemize}
\item the width of polygon defining breach is given for each breach.
\end{itemize}

A commented example of \telkey{BREACHES DATA FILE} is provided below.
This example is taken from the test case telemac2d/breach.

\begin{lstlisting}[language=bash]
# Number of breaches
3
# Bandwidth of the polyline defining the breach
15.0
# Upstream breach definition
# Width of Polygon defining the breaches
25.0
# Option for the breaching process
2
# Duration of the breaching process (0.0 = instant opening)
0.0
# option of lateral growth
#(1= bottom lowering, 2=dyke opening by widening)
2
# Final bottom altitude of the breach
5.9
# Control level of breach 
#(breach exist if water level exceed this value)
7.2
# Number of points of the polyline
4
# Description of the polyline
2000.0 37.5
2041.0 37.5
2082.0 37.5
2100.0 37.5
# Central breach definition
# Width of Polygon defining the breaches
20.0
# Option for the breaching process
3
# Duration of the breaching process (0.0 = instant opening)
300.0
# Option of lateral growth
# (1= bottom lowering, 2=dyke opening by widening)
1
# Final bottom altitude of the breach
5.5
# Number (global mesh) of the point controlling the breaching
9406
# Water level initiating the breaching process
6.0
# Number of points of the polyline
4
# Description of the polyline
2450.0 37.5
2500.0 37.5
2520.0 37.5
2550.0 37.5
# Downstream breach definition
# Width of Polygon defining the breach
10.0
# Option for the breaching process
1
# Start time of the breaching process
2000.0
# Duration of the breaching process (0.0 = instant opening)
600.0
# Option of lateral growth
# (1= bottom lowering, 2= opening by widening)
1
# Final bottom altitude of the breach
5.0
# Number of points on the dyke axis where the breach will appear
4
# Description of the polyline
2900.0 37.5
2920.0 37.5
2950.0 37.5
3000.0 37.5
\end{lstlisting}
