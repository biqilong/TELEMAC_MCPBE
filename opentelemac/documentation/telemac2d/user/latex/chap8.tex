\chapter{MANAGING WATER SOURCES}
\label{ch:manag:ws}
\telemac{2D} offers the possibility of placing water sources (with or without
tracer discharge) at any point of the domain.

The user has three options to place the various sources at differents points
of the domain:
\begin{enumerate}
\item with the keywords \telkey{ABSCISSAE OF SOURCES} and
\telkey{ORDINATES OF SOURCES}.
These are tables of real numbers,
giving the source coordinates in meters. In fact, \telemac{2D} will position a
source at the closest mesh point to that specified by these keywords.
In parallel mode, the sources must coincide exactly with one point of the mesh,
so this is recommended in all cases;
\item with the keyword \telkey{GLOBAL NUMBERS OF SOURCE NODES}. This is an
array of integers which contains the global numbers of nodes that correspond
 to source point locations.
\end{enumerate}
The program itself will determine the number of sources as a function of the
number of values given to these keywords.

Another option consists in associating the source(s) to a spatial area(s) of the
domain and not to a single point.
To use this option the user must create an external file which contains the coordinates
of every source region (polygons).
In this case the source will be placed at every point contained in the
region (the region must contain at least one node). The advantage of this option
is that the user can control the surface of the source. However, it is worth to
notice that the surface computed by the program will not correspond exactly to the
surface of the polygon indicated by the user. Indeed, the surface will be computed as
the sum of the surfaces of every point (integral of basis functions).
An example of file is given here: it indicates three source regions, defined
through 4 pair of coordinates.
\begin{lstlisting}[language=TelemacCas]
#
# COORDINATES AT SOURCE REGION 1
#
X(1)   Y(1)
198.7  25.85
198.7  24.06
201.5  24.06
201.5  25.85
#
# COORDINATES AT SOURCE REGION 2
#
X(2)   Y(2)
248.7  27.85
248.7  22.06
251.5  22.06
251.5  27.85
#
# COORDINATES AT SOURCE REGION 3
#
X(3)  Y(3)
372.0 8.0
372.0 2.0
376.0 2.0
376.0 8.0
\end{lstlisting}
The program itself will determine the number of points contained in each region.
The keyword to read the file is \telkey{SOURCE REGIONS DATA FILE}.
The maximum number of regions is limited by the keyword
\telkey{MAXIMUM NUMBER OF SOURCES}, which is 20 by default, while the number
of coordinates for every polygons is limited by the keyword
\telkey{MAXIMUM NUMBER OF POINTS FOR SOURCES REGIONS}, which is 10 by default.

At each source (single point or region), the user must indicate the discharge
(and the values of the tracers, if there are tracers).
The discharge is specified in m${}^{3}$/s using
the keyword \telkey{WATER DISCHARGE OF SOURCES} (and the value of the tracer by
the keyword \telkey{VALUES OF THE TRACERS AT THE SOURCES}). However, if these
two variables are time-dependent, the user can then program the two functions
\telfile{DEBSCE} (source discharge) and \telfile{TRSCE} (value of tracer at
source). It is also possible to use a specific file to define the time
evolution of the sources:
the source file (keyword \telkey{SOURCES FILE}). This file has exactly the same
structure as the one of the liquid boundary file. An example is presented here
with 2 sources and 2 tracers. Between 2 given times, the values are obtained by
linear interpolation.
\begin{lstlisting}[language=TelemacCas]
#
# TIME-DEPENDENT DISCHARGES AND TRACERS AT SOURCES 1 AND 2
#
#  T IS TIME
#
#  Q(1) IS DISCHARGE AT SOURCE 1 Q(2) IS DISCHARGE AT SOURCE 2
#
#  TR(1,1) IS TRACER 1 AT SOURCE 1 TR(1,2) IS TRACER 2 AT SOURCE 1 TR(2,1) IS
#  TRACER 1 AT SOURCE 2 TR(2,1) IS TRACER 2 AT SOURCE 2
#
#
T     Q(1)   TR(1,1)    TR(1,2)   Q(2)   TR(2,1)   TR(2,2)
s     m3/s      C          C       m3/s     C         C
0.     0.      99.        20.      0.      30.       40.
2.     1.      50.        20.      2.      30.       20.
4.     2.      25.        80.      4.      30.       20.
\end{lstlisting}
By default, when using the keyword \telkey{WATER DISCHARGE OF SOURCES}, the
sources are added in the continuity equations without contribution to the
momentum equations. In this case we consider that the velocity of sources is
equal to the velocity of the flow.

To take into account a momentum flux from the sources with \telemac{2D}, the
user must prescribe a particular velocity, which will be considered in the
momentum equations.  If this is constant throughout the simulation, the value
may be given with the keywords \telkey{VELOCITIES OF THE SOURCES ALONG X} and
\telkey{VELOCITIES OF THE SOURCES ALONG Y.} If not, the user must program the
two functions \telfile{VUSCE} (for the velocity along X) and \telfile{VVSCE}
(for the velocity along Y). In both functions, the time, source number and
depth of water are available to the user.

When the sources are located through the keywords \telkey{ABSCISSAE OF SOURCES}
or \telkey{GLOBAL NUMBERS OF SOURCE NODES}, the \telkey{WATER DISCHARGE OF SOURCES}
is mandatory to take into account the velocity of the source.
On the contrary, if the source is located in a region (through
\telkey{ASCII SOURCE DATA FILE}), the user can choose
to only impose the velocity of the source.
The program will automatically compute the water discharge, using the surface
of the region.

Although it is possible, in a practical point of view, it is not recommended
to use source point at the boundaries of the domain. In these cases, the
velocity field could not be as expected by the user even though \telfile{VUSCE}
and \telfile{VVSCE} functions are used. The imposition of hydrodynamics
boundary conditions could modify the prescribed components of the velocity of
the source.

If source terms are to be taken into account for the creation or decay of the
tracer, these must be introduced in the \telfile{DIFSOU} subroutine.

From a theoretical point of view, complete mass conservation can only be
ensured if the source is treated as a Dirac function and not as a linear
function. The type of treatment is indicated by the user with the keyword
\telkey{TYPE OF SOURCE}, which may have a value of 1 (linear function, default
value) or 2 (Dirac function). It should be noted that in the second case, the
solutions are of course less smoothed.

It is possible to manage sources without simulating tracer transport.


