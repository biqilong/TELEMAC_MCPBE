\chapter{BIOMASS Module}

The BIOMASS module is a water quality module which allows the calculation of algal biomass.
It estimates the extent of vegetal colonization in terms of various parameters:
sunshine, water temperature, degree of fertilization, ratio of water renewal,
water turbidity and toxicity \cite{gosse_biomass_1983}.
It takes into account of five tracers:

\begin{itemize}
\item phytoplankton biomass PHY,
\item the principal nutrients that influence its production (phosphorus, nitrogen)
  as well as the associated mineral forms, namely:

\begin{itemize}
\item dissolved mineral phosphorus assimilable by phytoplankton PO$_4$,
\item degradable phosphorus not assimilable by phytoplankton POR,
\item dissolved mineral nitrogen assimilable by phytoplankton NO$_3$,
\item degradable nitrogen not assimilable by phytoplankton NOR.
\end{itemize}
\end{itemize}

These variables are all expressed in mg/l except biomass which is expressed in $\mu$g(Chlorophyl a)/l.\\

The hypothesis assumes these substances act as tracers,
i.e. they are carried and dispersed in the mass of water.
In addition, they react with each other through biochemical processes.\\

The following paragraphs explain the internal source terms relating to the 5 tracers.

\section{Phytoplankton}

\subsection{Algal growth}

The rate of algal growth $CP$ (d$^{-1}$) is given by the equation:

\begin{equation}
  CP = C_{max} RAY g_1 LNUT \alpha_1,
\end{equation}

with $C_{max}$ = maximum rate of algal growth at 20$^{\circ}$C, one can take $C_{max}$ = 2.
$RAY$ represents the effect of sunshine on algal growth;
this dimensionless parameter belongs to the interval [0;1];
$g_1$ represents the effect of temperature on algal growth;
$g_1 = T/20$, where $T$ is the water temperature ($^{\circ}$C) (valid for 5$^{\circ}$C < $T$ < 25$^{\circ}$C).
$LNUT$ represents the effects of phosphoric and nitric nutrients on algal growth.
$\alpha_1$ = water toxicity coefficient for algae ($\alpha_1$ = 1 in the absence of toxicity).\\
$RAY$ is calculated by the Smith formula integrated over the vertical:

\begin{equation}
  RAY = \frac{1}{k_e h} \log \left( \frac{I_0 + \sqrt{IK^2+I_0^2} }{ I_h + \sqrt{IK^2+I_h^2} }  \right),
\end{equation}

where $k_e$ designates the extinction coefficient of solar rays in water (m$^{-1}$).
It is calculated either by the Secchi depth $Z_s$ by the Atkins formula: $k_e$ = 1.7/$Z_s$, or,
if $Z_s$ is unknown, by the Moss relation: $k_e$ = $k_{pe}$+$ \beta $ [PHY],
where $k_{pe}$ is the coefficient of vegetal turbidity without phytoplankton (m$^{-1}$)
and $ \beta $  the coefficient of the Moss relation ($ \beta $  = about 0.015).
$IK$ is a calibrating parameter in the Smith formula (W/m$^2$), of an order of magnitude 100.
$I_0$ is the flux density of solar radiation on the surface (W/m$^2$)
and $I_h$ is the flux density of solar radiation at the bottom (W/m$^2$), calculated by the following formula:

\begin{equation}
  I_h = I_0 \exp (-k_e h).
\end{equation}

$LNUT$ is calculated by the formula:

\begin{equation}
  LNUT = \min \left( \frac{[PO_4]}{KP+[PO_4]}, \frac{[NO_3]}{KN+[NO_3]} \right),
\end{equation}

with $KP$ = constant of phosphate half-saturation (mg/l) (about 0.005 mgP/l),
and $KN$ = constant of nitrate half-saturation (mg/l) (about 0.03 mgN/l).\\

Note: Nutrients affect phytoplankton growth PHY only by means of the limiting factor $LNUT$.
When [PO$_4$] and [NO$_3$] are big enough, $LNUT$ is close to 1 and
phytoplankton evolution no longer depends on nutrients.
In this case, there is no need to model the cycles of phosphorus and nitrogen
in order to simulate the evolution of phytoplankton.

\subsection{Algal disappearance}

The rate of algal disappearance $DP$ (d$^{-1}$) is given by the equation:

\begin{equation}
  DP = (RP+MP) g_2,
\end{equation}

with $RP$ = the rate of algal biomass respiration at 20$^{\circ}$C (d$^{-1}$),
$MP$ = the rate of algal biomass disappearance at 20$^{\circ}$C (d$^{-1}$).
$g_2$ represents the effect of temperature on algal disappearance.
$g_2 = T/20$ (valid for 5$^{\circ}$C < $T$ < 25$^{\circ}$C).
$MP$ is given by the following relation:

\begin{equation}
  MP = M_1 + M_2 [PHY] + \alpha_2,
\end{equation}

with $M_1$ and $M_2$ = coefficients of algal mortality at 20$^{\circ}$C,
$\alpha_2$ = coefficient of water toxicity for algae.

\section{Nitric and phosphoric nutrients}

The following physical and biochemical parameters are used
to describe the processes influencing the evolution of nitric and phosphoric nutrients:

\begin{itemize}
\item $fp$: average proportion of phosphorus in the cells of living phytoplankton (mgP/$\mu$gChlA),
\item $dtp$: proportion of directly assimilable phosphorus in dead phytoplankton ($\%$),
\item $k_1$: rate of transformation of POR into PO$_4$ through bacterial mineralization (d$^{-1}$),
\item $k_2$: rate of transformation of NOR into NO$_3$ through heterotrophic
  and autotrophic bacterial mineralization (d$^{-1}$),
\item $fn$: average proportion of directly assimilable nitrogen in living phytoplankton (mgN/$\mu$gChlA),
\item $dtn$: proportion of directly assimilable nitrogen in dead phytoplankton ($\%$),
\item $F_{POR}$: deposition flux of non-algal organic phosphorus (g/m$^2$s).
  $F_{POR} = W_{POR} [POR]$, $W_{POR}$ is the velocity of sedimentation of non-algal organic phosphorus (m/s),
\item $F_{NOR}$: deposition flux of non-algal organic nitrogen (g/m$^2$s).
  $F_{NOR} = W_{NOR} [NOR]$, $W_{NOR}$ is the velocity of sedimentation of non-algal organic nitrogen (m/s).
\end{itemize}

Note: the bottom and the processes that occur there are not modeled in the BIOMASS model.
Deposition is only represented by the deposition flux and,
once organic matter is deposited,
it no longer appears in the equations and can no longer be resuspended.
These deposition fluxes therefore correspond to a definitive loss of mass.

\section{Solved equations}

The equations of the BIOMASS model are described below, detailing the internal source terms $S_{intern\_i}$.\\

Tracer $\#$1: phytoplankton biomass

\begin{equation}
  F([PHY]) = (CP-DP) [PHY].
\end{equation}

Tracer $\#$2: assimilable mineral phosphorus

\begin{equation}
  F([PO_4]) = fp(dtp DP - CP) [PHY] + k_1 g_2 [POR].
\end{equation}

Tracer $\#$3: non-assimilable phosphorus

\begin{equation}
  F([POR]) = fp(1-dtp) DP [PHY] - k_1 g_2 [POR] - \frac{F_{POR}}{h}.
\end{equation}

Tracer $\#$4: assimilable mineral nitrogen

\begin{equation}
  F([NO_3]) = fn (dtn DP - CP) [PHY] + k_2 g_2 [NOR].
\end{equation}

Tracer $\#$5: non-assimilable nitrogen

\begin{equation}
  F([NOR]) = fn (1- dtn) DP [PHY] - k_2 g_2 [NOR] - \frac{F_{NOR}}{h},
\end{equation}

with $C_1$ = [PHY], $C_21$ = [PO$_4$], $C_3$ = [POR], $C_4$ = [NO$_3$] and $C_5$ = [NOR],
the matrices (5 $\times$ 5) $[\lambda]$ and $[\mu]$ containing the coefficients
$\lambda_i^j$ and $\mu_i^j$ are written thus (only non-zero terms are included):

\begin{equation}
\lambda_i^j = \frac{1}{86400}
\left(
%  \begin{array}{c;{2pt/2pt}c;{2pt/2pt}c;{2pt/2pt}c;{2pt/2pt}c}
  \begin{array}{ccccc}
    CP-DP & 0 & 0 & 0 & 0 \\ %\hdashline[2pt/2pt]
    fp (dtp DP -CP) & 0 &  k_1 g_2 & 0 & 0 \\ %\hdashline[2pt/2pt]
    fp (1-dtp) DP   & 0 & -k_1 g_2 & 0 & 0 \\ %\hdashline[2pt/2pt]
    fn (dtn DP -CP) & 0 &        0 & 0 &  k_2 g_2 \\ %\hdashline[2pt/2pt]
    fn (1-dtn) DP   & 0 &        0 & 0 & -k_2 g_2
  \end{array}
\right)
\end{equation}

$$
  \mu_i^j = 
  \begin{pmatrix}
   0 & 0 & 0 & 0 &  0 \\
   0 & 0 & 0 & 0 &  0 \\
   0 & 0 & -W_{POR} & 0 & 0 \\
   0 & 0 & 0 & 0 &  0 \\
   0 & 0 & 0 & 0 & -W_{NOR}
  \end{pmatrix}
$$  

The terms $\lambda_i^0$ and $\mu_i^0$ are zero for every $i$.\\

Divisions by 86,400 are performed to scale down time to one second.
