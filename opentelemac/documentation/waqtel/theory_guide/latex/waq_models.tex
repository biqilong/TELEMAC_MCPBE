\chapter{Water quality models}
\label{waq_models}
To simplify the document, the operator $F(C)$ is defined as follows:

\begin{equation}
  F(C) = \frac{\partial C}{\partial t} + \vec{U} \cdot \vec \nabla C
       - \nabla \cdot \left( k \vec \nabla C \right),
\end{equation}

with $C(x,y,z,t)$ is the concentration of tracer,
$k$ the coefficient of diffusion (m$^2$/s),
$\vec{U}$ the velocity vector (m/s).\\


In a water quality model, studied substances are advected and dispersed in the mass of water.
Their dispersion in the aquatic environment is linked to transport due to the flow on the one hand
and to the turbulent dispersion of the flow on the other hand.

The concentration of a substance (pollutant, oxygen, etc.) is also influenced by:
\begin{itemize}
\item punctual contributions, caused by releases (factories, sewage treatment plants, etc.)
  which are called \emph{external sources},
\item the presence of other substances in the mass of water, with which the tracer
  may react through biochemical transformations or the existence of forcings
  linked to its own concentration (e.g. the reaeration phenomenon for oxygen).
  These source terms for one tracer are called \emph{internal sources}
  (because internal to the mass of water) and they characterize the water quality module
  (description of the interactions between the tracers).
\end{itemize}

Solving a water quality problem consists in solving a system of $N$ advection-dispersion
equations (one equation per tracer) in which there are
possible external sources (releases, etc.)
and internal sources that exist through biochemical reactions.

A water quality model is characterized in WAQTEL by the coupled treatment of different tracers
and the description of the \emph{internal} source terms.\\

The internal source term for tracer $i$ can be written through the following form:
\begin{equation}
  S_{intern_i} = \lambda_i^0 + \sum_{j=1}^{N} \lambda_i^j C_j + \frac{\mu_i^0}{h}
               + \frac{\sum_{j=1}^{N}\mu_i^j C_j}{h}
\end{equation}

with $\lambda_i^0$ and $\mu_i^0$ the terms not depending on the tracer concentration $i$.

In the above equality, the 1$^{\rm{st}}$ term represents the volumic internal sources
(chemical reactions, etc.) whereas the 2$^{\rm{nd}}$ term represents the surfacic internal sources
(deposition, re-suspension, evaporation, etc.).

Depending on the water quality module, the matrices [$\lambda$] and [$\mu$]
containing the coefficients $\lambda_i^j$ and $\mu_i^j$ will be written differently.
The internal source terms relating water quality processes are treated
in an explicit way in the advection-diffusion equation,
as they depend on the concentration of other tracers,
unknown at the time step $n+1$.

If there are interactions between tracers or specific evolution laws of tracers,
a water quality model can be used to determine the internal sources of tracers
that can be involved in the transport equation.

Internal sources of tracers are computed in the water quality module \waqtel
at each time step, with respect to physical parameters and
the concentrations of different tracers.
Then they are given to \telemac{2d} or \telemac{3d} which compute
the tracer evolution (by advection and diffusion) taking the
source terms into account.\\

Several water quality modules are available in the \waqtel library:
\begin{itemize}
  \item O$_2$ module: simplified model of dissolved oxygen,
  \item BIOMASS module: phytoplankton biomass model,
  \item EUTRO model: eutrophication model (dissolved oxygen and algal biomass),
  \item MICROPOL: model for the evolution of heavy metals or radioelements,
    taking into account their interaction with fine sediments (suspended matter).
    Anyway, no modification of bottom is done,
  \item THERMIC module: model for the evolution of water temperature under the influence
    of atmospheric fluxes,
  \item AED2 model: the water quality and aquatic ecology model,
  \item degradation law.
\end{itemize}

The different modules and their features are described in the following chapters
in which the internal source terms are also detailed.

Note that the structure chosen for \waqtel enables to easily add new
water quality modules, by implementing terms relating to internal sources.
