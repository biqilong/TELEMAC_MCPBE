\chapter{MICROPOL Module}

The MICROPOL module simulates the outcome of a micropollutant (radioelement or heavy metal)
in the three compartments considered to be of major importance in a river ecosystem:
water, suspended particulate matter (S.P.M.) and the bottom sediments.
Each of these compartments represents an homogeneous class:
S.P.M. and sediments taken into account represent the grain-size class of clays and silts
(cohesive-type fine sediments, of about < 20-25 $\mu$m diameter),
likely to attach the majority of micropollutants.\\

Thanks to its capacity for adsorption and desorption of micropollutants,
suspended particulate matter (S.P.M.) is one of the first links in the chain of contamination.
Suspended particulate matter is carried and dispersed in the mass of water
like a tracer and is also subject to the laws of sedimentary physics:
it settles in calm waters and produces bottom sediments,
and can be re-suspended by a strong flow.
Deposits cannot move. They are treated as tracers that can be neither advected
nor dispersed by the mass of water, but are likely to be re-suspended.\\

The model considers 5 tracers:

\begin{itemize}
\item suspended matter (SS),
\item bottom sediments (SF), neither advected nor dispersed,
\item dissolved form of micropollutant,
\item the fraction adsorbed by suspended particulate matter,
\item the fraction adsorbed by bottom sediments, neither advected nor dispersed.
\end{itemize}

\subsubsection{Notes, and limitations of the MICROPOL module}

\begin{itemize}
\item whether in suspension or deposited on the bottom, the matter is considered
  to be a passive tracer:
  in other words, it does not influence the flow (no feedback).
  This hypothesis involves that the depth of deposits must be negligible compared
  to the depth of water (the bed is assumed to be unmodified).
\item there is no direct adsorption/desorption of dissolved micropollutants
  on the deposited matter, only on the suspended particulate matter
  (the model assumes a preponderance of water – S.P.M. exchanges over direct water
  – bottom sediment exchanges).
  Bottom sediments only become radioactive by means of polluted S.P.M. deposition. 
\end{itemize}

\section{Suspended matter}

\subsection{Description of phenomena}

The model describing the evolution of S.P.M. and bottom sediments involved in MICROPOL
is a classic representation of the laws of deposition and re-suspension
of cohesive types of S.P.M., that are the laws of Krone \cite{krone_flume_1962}
and Partheniades \cite{partheniades_erosion_deposition_1965}.\\

Both processes require the knowledge of two characteristic constants:

\begin{itemize}
\item deposition occurs when shear stress on the bottom $\tau_b$,
  which varies according to the flow rate, becomes less than a threshold value $\tau_s$,
  known as the shear stress critical for sedimentation.
  It is then assumed that the S.P.M. sediments at a constant velocity $w$
  (known as the settling velocity or velocity of sedimentation),
\item re-suspension occurs when a new threshold $\tau_r$,
  known as the shear stress critical for re-suspension, is exceeded.
  Its importance is weighted by a constant $e$, the rate of erosion characteristic
  of deposited S.P.M. (also known as the Partheniades constant).
\end{itemize}

\subsection{Equations}

These phenomena translate into the following expressions of deposition flux ($SED$) and erosion ($RS$),
in kg/m$^2$/s:

\begin{equation}
  SED = \left\{
    \begin{array}{ccl}
      wSS \left( 1 - \frac{\tau_b}{\tau_s} \right) & \rm{if} & \tau_b < \tau_s\\
      0 & \rm{if} & \tau_b \ge \tau_s
    \end{array}
    \right .
\end{equation}

\begin{equation}
  RS = \left\{
    \begin{array}{ccl}
      e \left( \frac{\tau_b}{\tau_r} -1 \right) \varepsilon(SF) & \rm{if} & \tau_b > \tau_r\\
      0 & \rm{if} & \tau_b \le \tau_r
    \end{array}
    \right .
\end{equation}

where $\varepsilon(x)$ is a function such that $\varepsilon(x) = 0$
if $x = 0$ and $\varepsilon(x) = 1$ else.\\

The bottom shear stress $\tau_b$ (in Pa) is given by $\tau_b = \frac{1}{2} \rho C_f U^2$,
with $C_f$ = the friction coefficient and $U^2$ the square of the velocity.\\

The equations of the evolution of S.P.M. tracers (variable $SS$)
and bottom sediments (variable $SF$) are as follows:\\

Tracer $\#$1: suspended particulate matter

\begin{equation}
  F(SS) = \frac{RS-SED}{h}.
\end{equation}

Tracer $\#$2: bottom sediments (tracer neither advected nor diffused)\\

\begin{equation}
  \frac{\partial (SF)}{\partial t} = SED - RS.
\end{equation}

The model relating to S.P.M. therefore has four parameters: the velocity of sedimentation $w$,
the rate of erosion $e$, the critical stress of deposition $\tau_s$
and the critical stress of erosion $\tau_r$.

\section{Micropollutants}

\subsection{Description of phenomena}

The model representing the evolution of micropollutants assumes
that the transfers of micropollutants (radioelement, metal)
between the dissolved and particulate phases correspond to either
direct adsorption or ionic exchanges modeled by a reversible reaction,
of kinetic order 1 \cite{ciffroy_doubs_1995}.
In the case of direct adsorption, the reaction can be represented in the form of the following formula:\\

\begin{figure}[H]
  \centering
  \includegraphics[scale=0.4]{graphics/image60.png}
\end{figure}

%$R^\ast +.- S    k_1/k_{-1}     R - S^\ast$

with R$^\ast$ = micropollutant in dissolved form, – S = surface site associated with S.P.M.,
R–S$^{\ast}$ = adsorbed micropollutant.\\

It is a reversible reaction, controlled by velocities of adsorption ($k_1$ in l/g/s)
and desorption ($k_{-1}$ in s$^{-1}$).
It leads to an equilibrium, and then a distribution of micropollutant
between the dissolved and particulate phase described
by the distribution coefficient $K_d$ (in l/g):

\begin{equation}
  K_d = \frac{[R-S^\ast]}{[R^\ast]} = \frac{k_1}{k_{-1}},
\end{equation}

where [R$^\ast$] is the activity (or concentration of micropollutant)
in dissolved phase (in Bq/m$^3$ or kg/m$^3$), [R-S$^\ast$] is the activity
(or concentration of micropollutant) associated to S.P.M. (in Bq/kg or kg/kg).\\

Once adsorbed, the fixed micropollutants act like S.P.M. (deposition, re-suspension)
and can also produce areas of polluted sediment.\\

The model includes a law of exponential decay (radioactive decay type) of micropollutant
concentrations in each compartment of the modeled ecosystem,
through a constant written $L$ (expressed in s$^{-1}$).\\

\subsection{Equations}

The system includes an equation for each of the micropollutant phases, namely 3 tracers:

\begin{itemize}
\item $C$: concentration of micropollutants in water (Bq/m$^3$),
\item $C_s$: concentration of micropollutants adsorbed by S.P.M. (Bq/kg) (dry weight),
\item $C_f$: concentration of micropollutants adsorbed by bottom sediments (Bq/kg).
\end{itemize}

Note: The unit of concentration chosen for the demonstration is Bq/m$^3$,
but it could also be written in kg/m$^3$ (for example, in the case of a metal).\\

The internal sources of each of these tracers correspond to the phenomena
of adsorption/desorption, deposition/re-suspension and exponential decay.
Taking these phenomena into account leads to the following equations
in each of the three compartments, water, S.P.M. and bottom sediments
(with section \ref{waq_models} notations):

\begin{itemize}
\item dissolution phase
\begin{equation}
  F(C) = -k_{-1}.SS (K_d.C - C_s ) - L.C,
\end{equation}

\item adsorption by suspended particulate matter phase
\begin{equation}
  F(SS.C_s) = k_{-1}.SS (K_d.C - C_s ) + \frac{RS.C_f-SED.C_s}{h} - L.SS.C_s,
\end{equation}

\item adsorption by bottom sediments phase (tracer neither advected nor diffused)
\begin{equation}
  \frac{\partial (SF.C_f)}{\partial t} = SED.C_s - RS.C_f - L.SF.C_f.
\end{equation}

\end{itemize}

Setting the new variables as: $C_{ss}$  = $SS.C_s$, in Bq/m$^3$ of water,
$C_{ff} = SF.C_f$, in Bq/m$^2$ of bottom sediments,
and $SEDP = \frac{SED}{SS} = w \left (1-\frac{\tau_b}{\tau_s} \right)$
if $\tau_b < \tau_s$ or 0 otherwise, the equations thus become:\\

Tracer $\#$3: dissolution phase

\begin{equation}
  F(C) = -k_{-1}.SS.K_d.C + k_{-1}.C_{SS} - L.C.
\end{equation}

Tracer $\#$4: phase of adsorption by suspended particulate matter

\begin{equation}
  F(C_{SS}) = k_{-1}.K_d.SS.C - k_{-1}.C_{SS} + \frac{\frac{RS}{SF}C_{ff}-SEDP.C_{SS}}{h}- L.C_{SS}.
\end{equation}

Tracer $\#$5: phase of adsorption by bottom sediments

\begin{equation}
  \frac{\partial C_{ff}}{\partial t} = SEDP.C_{SS} - \frac{RS}{SF} C_{ff} - L.C_{ff}.
\end{equation}

Terms with $SF$ as denominator are nullified when $SF$ is equal to 0.\\

Therefore, there are three parameters of the micropollutant model:
the distribution coefficient at equilibrium $K_d$,
the kinetic constant of desorption $k_{-1}$,
and the exponential decay constant $L$ (radioactive decay, for example).
In total, the MICROPOL model thus includes 7 physical parameters.

\section{Solved equations}

Thus, noting $C_1 = SS$, $C_2 = SF$, $C_3 =C$, $C_4 = C_{ss}$, $C_5 = C_{ff}$ ,

the matrices (5 $\times$ 5) $[\lambda]$ and $[\mu]$ containing the coefficients
$\lambda_i^j$ and $\mu_i^j$ are written as:

$$  \lambda = 
  \begin{pmatrix}
    0 & 0 & 0 & 0 & 0\\
    SEDP & 0 & 0 & 0 & 0\\
    0 & 0 & -L -k_{-1}.SS.K_d & k_{-1} & 0\\
    0 & 0 & -k_{-1}.K_d.SS & -k_{-1} - L & 0\\
    0 & 0 & 0 & SEDP & -\frac{RS}{SF}-L
  \end{pmatrix}
$$  

$$  \mu = 
  \begin{pmatrix}
    -SEDP & 0 & 0 & 0 & 0\\
    0 & 0 & 0 & 0 & 0\\
    0 & 0 & 0 & 0 & 0\\
    0 & 0 & 0 & -SEDP & \frac{RS}{SF}\\
    0 & 0 & 0 & 0 & 0
  \end{pmatrix}
$$  

Note the case where $SF$ = 0: $C_{ff} = 0$, $\lambda_5^5 = 0$ and $\mu_4^5 = 0$.
The only non-zero terms $\lambda_i^0$ and $\mu_i^0$ are $\lambda_2^0 = -RS$ and $\mu_2^0 = RS$.
