%%%%%%%%%%%%%%%%%%%%%%%%%%%%%%%%%%%%%%%%%
%  WAQTEL Documentation
%  Technical manual
%
%%%%%%%%%%%%%%%%%%%%%%%%%%%%%%%%%%%%%%%%%

%----------------------------------------------------------------------------------------
%	PACKAGES AND OTHER DOCUMENT CONFIGURATIONS
%----------------------------------------------------------------------------------------
\documentclass[Waqtel]{../../data/TelemacDoc} % Default font size and left-justified equations


\begin{document}

\let\cleardoublepage\clearpage
%----------------------------------------------------------------------------------------
%	TITLE PAGE
%----------------------------------------------------------------------------------------
\title{WAQTEL}
\subtitle{Technical manual}
\version{\telmaversion}
\date{\today}
\maketitle
\clearpage

%----------------------------------------------------------------------------------------
%	AUTHORS PAGE
%----------------------------------------------------------------------------------------

%----------------------------------------------------------------------------------------
%	TABLE OF CONTENTS
%----------------------------------------------------------------------------------------


\pagestyle{empty} % No headers

\tableofcontents% Print the table of contents itself

%\cleardoublepage % Forces the first chapter to start on an odd page so it's on the right

\pagestyle{fancy} % Print headers again

\thispagestyle{empty}

\chapter*{Abstract}
This technical manual is mainly based on the translation of the TRACER principle note
``Outil de simulation 1-D MASCARET V7.1. Module de qualité d'eau TRACER. Note de principe''
written by Kamal El Kadi Abderrezzak and Marilyne Luck in 2012
\cite{elkadi_tracer_2012} (ref: EDF R\&D-LNHE H-P73-2011-01786-FR).
It also includes the heat atmosphere exchange subsection of the \telemac{3d} theory guide.

TRACER is the transport and water quality module of the 1D free surface code MASCARET.
TRACER simulates the evolution of several coupled tracers
without retroaction on the flow with respect to hydraulic conditions,
boundary conditions, external sources and biochemical interactions between tracers.\\

This note contains an explanation of the method retained for solving the advection-dispersion equation,
and its application to water quality modeling.

All the processes existing in TRACER were implemented in \telemac{2D} and \telemac{3D} through
a new module called \waqtel between v7.0 and v7.2.
The following water quality modules are described:
\begin{itemize}
\item O$_2$ (dissolved oxygen, organic and ammonia loads),
\item BIOMASS (phytoplankton biomass and nutrients),
\item EUTRO (dissolved oxygen, phytoplankton biomass, nutrients, organic and ammonia loads),
\item MICROPOL (micropollutants and suspended matter),
\item THERMIC (water temperature),
\item degradation law.
\end{itemize}
A reference to the library documentation of the water quality and aquatic ecology model
AED2 is also given.

\newpage

\chapter{Water quality models}
\label{waq_models}
To simplify the document, the operator $F(C)$ is defined as follows:

\begin{equation}
  F(C) = \frac{\partial C}{\partial t} + \vec{U} \cdot \vec \nabla C
       - \nabla \cdot \left( k \vec \nabla C \right),
\end{equation}

with $C(x,y,z,t)$ is the concentration of tracer,
$k$ the coefficient of diffusion (m$^2$/s),
$\vec{U}$ the velocity vector (m/s).\\


In a water quality model, studied substances are advected and dispersed in the mass of water.
Their dispersion in the aquatic environment is linked to transport due to the flow on the one hand
and to the turbulent dispersion of the flow on the other hand.

The concentration of a substance (pollutant, oxygen, etc.) is also influenced by:
\begin{itemize}
\item punctual contributions, caused by releases (factories, sewage treatment plants, etc.)
  which are called \emph{external sources},
\item the presence of other substances in the mass of water, with which the tracer
  may react through biochemical transformations or the existence of forcings
  linked to its own concentration (e.g. the reaeration phenomenon for oxygen).
  These source terms for one tracer are called \emph{internal sources}
  (because internal to the mass of water) and they characterize the water quality module
  (description of the interactions between the tracers).
\end{itemize}

Solving a water quality problem consists in solving a system of $N$ advection-dispersion
equations (one equation per tracer) in which there are
possible external sources (releases, etc.)
and internal sources that exist through biochemical reactions.

A water quality model is characterized in WAQTEL by the coupled treatment of different tracers
and the description of the \emph{internal} source terms.\\

The internal source term for tracer $i$ can be written through the following form:
\begin{equation}
  S_{intern_i} = \lambda_i^0 + \sum_{j=1}^{N} \lambda_i^j C_j + \frac{\mu_i^0}{h}
               + \frac{\sum_{j=1}^{N}\mu_i^j C_j}{h}
\end{equation}

with $\lambda_i^0$ and $\mu_i^0$ the terms not depending on the tracer concentration $i$.

In the above equality, the 1$^{\rm{st}}$ term represents the volumic internal sources
(chemical reactions, etc.) whereas the 2$^{\rm{nd}}$ term represents the surfacic internal sources
(deposition, re-suspension, evaporation, etc.).

Depending on the water quality module, the matrices [$\lambda$] and [$\mu$]
containing the coefficients $\lambda_i^j$ and $\mu_i^j$ will be written differently.
The internal source terms relating water quality processes are treated
in an explicit way in the advection-diffusion equation,
as they depend on the concentration of other tracers,
unknown at the time step $n+1$.

If there are interactions between tracers or specific evolution laws of tracers,
a water quality model can be used to determine the internal sources of tracers
that can be involved in the transport equation.

Internal sources of tracers are computed in the water quality module \waqtel
at each time step, with respect to physical parameters and
the concentrations of different tracers.
Then they are given to \telemac{2d} or \telemac{3d} which compute
the tracer evolution (by advection and diffusion) taking the
source terms into account.\\

Several water quality modules are available in the \waqtel library:
\begin{itemize}
  \item O$_2$ module: simplified model of dissolved oxygen,
  \item BIOMASS module: phytoplankton biomass model,
  \item EUTRO model: eutrophication model (dissolved oxygen and algal biomass),
  \item MICROPOL: model for the evolution of heavy metals or radioelements,
    taking into account their interaction with fine sediments (suspended matter).
    Anyway, no modification of bottom is done,
  \item THERMIC module: model for the evolution of water temperature under the influence
    of atmospheric fluxes,
  \item AED2 model: the water quality and aquatic ecology model,
  \item degradation law.
\end{itemize}

The different modules and their features are described in the following chapters
in which the internal source terms are also detailed.

Note that the structure chosen for \waqtel enables to easily add new
water quality modules, by implementing terms relating to internal sources.


\input{latex/module_o2.tex}

\chapter{BIOMASS Module}

The BIOMASS module is a water quality module which allows the calculation of algal biomass.
It estimates the extent of vegetal colonization in terms of various parameters:
sunshine, water temperature, degree of fertilization, ratio of water renewal,
water turbidity and toxicity \cite{gosse_biomass_1983}.
It takes into account of five tracers:

\begin{itemize}
\item phytoplankton biomass PHY,
\item the principal nutrients that influence its production (phosphorus, nitrogen)
  as well as the associated mineral forms, namely:

\begin{itemize}
\item dissolved mineral phosphorus assimilable by phytoplankton PO$_4$,
\item degradable phosphorus not assimilable by phytoplankton POR,
\item dissolved mineral nitrogen assimilable by phytoplankton NO$_3$,
\item degradable nitrogen not assimilable by phytoplankton NOR.
\end{itemize}
\end{itemize}

These variables are all expressed in mg/l except biomass which is expressed in $\mu$g(Chlorophyl a)/l.\\

The hypothesis assumes these substances act as tracers,
i.e. they are carried and dispersed in the mass of water.
In addition, they react with each other through biochemical processes.\\

The following paragraphs explain the internal source terms relating to the 5 tracers.

\section{Phytoplankton}

\subsection{Algal growth}

The rate of algal growth $CP$ (d$^{-1}$) is given by the equation:

\begin{equation}
  CP = C_{max} RAY g_1 LNUT \alpha_1,
\end{equation}

with $C_{max}$ = maximum rate of algal growth at 20$^{\circ}$C, one can take $C_{max}$ = 2.
$RAY$ represents the effect of sunshine on algal growth;
this dimensionless parameter belongs to the interval [0;1];
$g_1$ represents the effect of temperature on algal growth;
$g_1 = T/20$, where $T$ is the water temperature ($^{\circ}$C) (valid for 5$^{\circ}$C < $T$ < 25$^{\circ}$C).
$LNUT$ represents the effects of phosphoric and nitric nutrients on algal growth.
$\alpha_1$ = water toxicity coefficient for algae ($\alpha_1$ = 1 in the absence of toxicity).\\
$RAY$ is calculated by the Smith formula integrated over the vertical:

\begin{equation}
  RAY = \frac{1}{k_e h} \log \left( \frac{I_0 + \sqrt{IK^2+I_0^2} }{ I_h + \sqrt{IK^2+I_h^2} }  \right),
\end{equation}

where $k_e$ designates the extinction coefficient of solar rays in water (m$^{-1}$).
It is calculated either by the Secchi depth $Z_s$ by the Atkins formula: $k_e$ = 1.7/$Z_s$, or,
if $Z_s$ is unknown, by the Moss relation: $k_e$ = $k_{pe}$+$ \beta $ [PHY],
where $k_{pe}$ is the coefficient of vegetal turbidity without phytoplankton (m$^{-1}$)
and $ \beta $  the coefficient of the Moss relation ($ \beta $  = about 0.015).
$IK$ is a calibrating parameter in the Smith formula (W/m$^2$), of an order of magnitude 100.
$I_0$ is the flux density of solar radiation on the surface (W/m$^2$)
and $I_h$ is the flux density of solar radiation at the bottom (W/m$^2$), calculated by the following formula:

\begin{equation}
  I_h = I_0 \exp (-k_e h).
\end{equation}

$LNUT$ is calculated by the formula:

\begin{equation}
  LNUT = \min \left( \frac{[PO_4]}{KP+[PO_4]}, \frac{[NO_3]}{KN+[NO_3]} \right),
\end{equation}

with $KP$ = constant of phosphate half-saturation (mg/l) (about 0.005 mgP/l),
and $KN$ = constant of nitrate half-saturation (mg/l) (about 0.03 mgN/l).\\

Note: Nutrients affect phytoplankton growth PHY only by means of the limiting factor $LNUT$.
When [PO$_4$] and [NO$_3$] are big enough, $LNUT$ is close to 1 and
phytoplankton evolution no longer depends on nutrients.
In this case, there is no need to model the cycles of phosphorus and nitrogen
in order to simulate the evolution of phytoplankton.

\subsection{Algal disappearance}

The rate of algal disappearance $DP$ (d$^{-1}$) is given by the equation:

\begin{equation}
  DP = (RP+MP) g_2,
\end{equation}

with $RP$ = the rate of algal biomass respiration at 20$^{\circ}$C (d$^{-1}$),
$MP$ = the rate of algal biomass disappearance at 20$^{\circ}$C (d$^{-1}$).
$g_2$ represents the effect of temperature on algal disappearance.
$g_2 = T/20$ (valid for 5$^{\circ}$C < $T$ < 25$^{\circ}$C).
$MP$ is given by the following relation:

\begin{equation}
  MP = M_1 + M_2 [PHY] + \alpha_2,
\end{equation}

with $M_1$ and $M_2$ = coefficients of algal mortality at 20$^{\circ}$C,
$\alpha_2$ = coefficient of water toxicity for algae.

\section{Nitric and phosphoric nutrients}

The following physical and biochemical parameters are used
to describe the processes influencing the evolution of nitric and phosphoric nutrients:

\begin{itemize}
\item $fp$: average proportion of phosphorus in the cells of living phytoplankton (mgP/$\mu$gChlA),
\item $dtp$: proportion of directly assimilable phosphorus in dead phytoplankton ($\%$),
\item $k_1$: rate of transformation of POR into PO$_4$ through bacterial mineralization (d$^{-1}$),
\item $k_2$: rate of transformation of NOR into NO$_3$ through heterotrophic
  and autotrophic bacterial mineralization (d$^{-1}$),
\item $fn$: average proportion of directly assimilable nitrogen in living phytoplankton (mgN/$\mu$gChlA),
\item $dtn$: proportion of directly assimilable nitrogen in dead phytoplankton ($\%$),
\item $F_{POR}$: deposition flux of non-algal organic phosphorus (g/m$^2$s).
  $F_{POR} = W_{POR} [POR]$, $W_{POR}$ is the velocity of sedimentation of non-algal organic phosphorus (m/s),
\item $F_{NOR}$: deposition flux of non-algal organic nitrogen (g/m$^2$s).
  $F_{NOR} = W_{NOR} [NOR]$, $W_{NOR}$ is the velocity of sedimentation of non-algal organic nitrogen (m/s).
\end{itemize}

Note: the bottom and the processes that occur there are not modeled in the BIOMASS model.
Deposition is only represented by the deposition flux and,
once organic matter is deposited,
it no longer appears in the equations and can no longer be resuspended.
These deposition fluxes therefore correspond to a definitive loss of mass.

\section{Solved equations}

The equations of the BIOMASS model are described below, detailing the internal source terms $S_{intern\_i}$.\\

Tracer $\#$1: phytoplankton biomass

\begin{equation}
  F([PHY]) = (CP-DP) [PHY].
\end{equation}

Tracer $\#$2: assimilable mineral phosphorus

\begin{equation}
  F([PO_4]) = fp(dtp DP - CP) [PHY] + k_1 g_2 [POR].
\end{equation}

Tracer $\#$3: non-assimilable phosphorus

\begin{equation}
  F([POR]) = fp(1-dtp) DP [PHY] - k_1 g_2 [POR] - \frac{F_{POR}}{h}.
\end{equation}

Tracer $\#$4: assimilable mineral nitrogen

\begin{equation}
  F([NO_3]) = fn (dtn DP - CP) [PHY] + k_2 g_2 [NOR].
\end{equation}

Tracer $\#$5: non-assimilable nitrogen

\begin{equation}
  F([NOR]) = fn (1- dtn) DP [PHY] - k_2 g_2 [NOR] - \frac{F_{NOR}}{h},
\end{equation}

with $C_1$ = [PHY], $C_21$ = [PO$_4$], $C_3$ = [POR], $C_4$ = [NO$_3$] and $C_5$ = [NOR],
the matrices (5 $\times$ 5) $[\lambda]$ and $[\mu]$ containing the coefficients
$\lambda_i^j$ and $\mu_i^j$ are written thus (only non-zero terms are included):

\begin{equation}
\lambda_i^j = \frac{1}{86400}
\left(
%  \begin{array}{c;{2pt/2pt}c;{2pt/2pt}c;{2pt/2pt}c;{2pt/2pt}c}
  \begin{array}{ccccc}
    CP-DP & 0 & 0 & 0 & 0 \\ %\hdashline[2pt/2pt]
    fp (dtp DP -CP) & 0 &  k_1 g_2 & 0 & 0 \\ %\hdashline[2pt/2pt]
    fp (1-dtp) DP   & 0 & -k_1 g_2 & 0 & 0 \\ %\hdashline[2pt/2pt]
    fn (dtn DP -CP) & 0 &        0 & 0 &  k_2 g_2 \\ %\hdashline[2pt/2pt]
    fn (1-dtn) DP   & 0 &        0 & 0 & -k_2 g_2
  \end{array}
\right)
\end{equation}

$$
  \mu_i^j = 
  \begin{pmatrix}
   0 & 0 & 0 & 0 &  0 \\
   0 & 0 & 0 & 0 &  0 \\
   0 & 0 & -W_{POR} & 0 & 0 \\
   0 & 0 & 0 & 0 &  0 \\
   0 & 0 & 0 & 0 & -W_{NOR}
  \end{pmatrix}
$$  

The terms $\lambda_i^0$ and $\mu_i^0$ are zero for every $i$.\\

Divisions by 86,400 are performed to scale down time to one second.


\input{latex/module_eutro.tex}

\chapter{MICROPOL Module}

The MICROPOL module simulates the evolution of a micropollutant (radioelement or heavy metal)
in the three compartments considered to be of major importance in a river ecosystem:
water, Suspended Particulate Matter (SPM) and bottom material.
Each of these compartments represents an homogeneous class:
SPM and sediments represent the grain-size class of clay and silt
(cohesive fine sediments, of diameter about less than 20 to 25 $\mu$m),
likely to attach the majority of micropollutants.\\

Due to adsorption and desorption of micropollutants,
SPM is one of the first links in the chain of contamination.
SPM is carried and dispersed in the water mass
as a tracer and is also subject to the laws of sedimentary physics:
it settles in calm waters and produces bottom sediments,
and can be re-suspended by a high flow.
Deposits cannot move. They are treated as tracers that can be neither advected
nor dispersed by the water mass, but are likely to be re-suspended.\\

The model considers 5 tracers:

\begin{itemize}
\item suspended matter (SS),
\item bottom sediments (SF), neither advected nor dispersed,
\item dissolved form of micropollutant,
\item the fraction adsorbed by suspended particulate matter,
\item the fraction adsorbed by bottom sediments, neither advected nor dispersed.
\end{itemize}

\subsubsection{Notes, and limitations of the MICROPOL module}

\begin{itemize}
\item whether in suspension or deposited on the bottom, the matter is considered
  to be a passive tracer:
  in other words, it does not influence the flow (no feedback).
  This hypothesis involves that the deposits depth must be negligible compared
  to the water depth (the bed is assumed to be unmodified).
\item there is no direct adsorption/desorption of dissolved micropollutants
  on the deposited matter, only on the SPM
  (the model assumes a preponderance of water – SPM exchanges over direct water
  – bottom sediment exchanges).
  Bottom sediments only become radioactive by means of polluted SPM deposition. 
\end{itemize}

\section{Suspended matter}

\subsection{Description of phenomena}

The model describing the evolution of SPM and bottom sediments involved in MICROPOL
is a classic representation of the deposition laws and re-suspension
of cohesive SPM, that are the laws of Krone \cite{krone_flume_1962}
and Partheniades \cite{partheniades_erosion_deposition_1965}.\\

Both processes require the knowledge of characteristic constants:

\begin{itemize}
\item deposition occurs when bottom shear stress $\tau_b$,
  which varies according to the flow conditions, becomes lower than a threshold value $\tau_s$,
  known as the critical shear stress for sedimentation.
  It is then assumed that the SPM settles at a constant velocity $w$
  (known as the settling velocity or velocity of sedimentation),
\item re-suspension occurs when a threshold $\tau_r$,
  known as the critical shear stress for re-suspension, is exceeded.
  Its importance is weighted by a constant $e$, the rate of erosion characteristic
  of deposited SPM (also known as the Partheniades constant).
\end{itemize}

\subsection{Equations}

These phenomena translate into the following expressions of deposition flux ($SED$) and erosion ($RS$),
in kg/m$^2$/s:

\begin{equation}
  SED = \left\{
    \begin{array}{ccl}
      wSS \left( 1 - \frac{\tau_b}{\tau_s} \right) & \rm{if} & \tau_b < \tau_s\\
      0 & \rm{if} & \tau_b \ge \tau_s
    \end{array}
    \right .
\end{equation}

\begin{equation}
  RS = \left\{
    \begin{array}{ccl}
% \varepsilon(SF)
      e \left( \frac{\tau_b}{\tau_r} -1 \right) & \rm{if} & \tau_b > \tau_r\\
      0 & \rm{if} & \tau_b \le \tau_r
    \end{array}
    \right .
\end{equation}

%where $\varepsilon(x)$ is a function such that $\varepsilon(x) = 0$
%if $x = 0$ and $\varepsilon(x) = 1$ else.\\

The bottom shear stress $\tau_b$ (in Pa) is given by $\tau_b = \frac{1}{2} \rho C_f U^2$,
with $C_f$ = the friction coefficient and $U^2$ the square of the velocity.\\

The equations of the evolution of SPM tracers (variable $SS$)
and bottom sediments (variable $SF$) are as follows:\\

Tracer $\#$1: suspended particulate matter

\begin{equation}
  F(SS) = \frac{RS-SED}{h}.
\end{equation}

Tracer $\#$2: bottom sediments (tracer neither advected nor diffused)\\

\begin{equation}
  \frac{\partial (SF)}{\partial t} = SED - RS.
\end{equation}

The model relating to SPM has four parameters: the velocity of sedimentation $w$,
the erosion rate $e$, the critical shear stress for deposition $\tau_s$
and the critical shear stress for erosion $\tau_r$.

\section{Micropollutants}

\subsection{Description of phenomena}

The model representing the evolution of micropollutants assumes
that the transfers of micropollutants (radioelement, metal)
between the dissolved and particulate phases correspond to either
direct adsorption or ionic exchanges modeled by a reversible reaction,
of 1$^{\rm{st}}$ kinetic order \cite{ciffroy_doubs_1995}.
In the case of direct adsorption, the reaction can be represented in the form of the following formula:\\

\begin{figure}[H]
  \centering
  \includegraphics[scale=0.4]{graphics/image60.png}
\end{figure}

%$R^\ast +.- S    k_1/k_{-1}     R - S^\ast$

with R$^\ast$ = micropollutant in dissolved form, – S = surface site associated with SPM,
R–S$^{\ast}$ = adsorbed micropollutant.\\

It is a reversible reaction, controlled by adsorption ($k_1$ in l/g/s)
and desorption velocities ($k_{-1}$ in s$^{-1}$).
It leads to an equilibrium state, and then a distribution of micropollutants
between the dissolved and particulate phase described
by the distribution coefficient $K_d$ (in l/g):

\begin{equation}
  K_d = \frac{[R-S^\ast]}{[R^\ast]} = \frac{k_1}{k_{-1}},
\end{equation}

where [R$^\ast$] is the activity (or concentration of micropollutant)
in dissolved phase (in Bq/m$^3$ or kg/m$^3$), [R-S$^\ast$] is the activity
(or concentration of micropollutant) associated to SPM (in Bq/kg or kg/kg).\\

Once adsorbed, the fixed micropollutants act like SPM (deposition, re-suspension)
and can also produce areas of polluted sediment.\\

The model includes an exponential decay law (radioactive decay type) of micropollutant
concentrations in each compartment of the modeled ecosystem,
through a constant written $L$ (expressed in s$^{-1}$).\\

\subsection{Equations}

The system includes an equation for each micropollutant phase, namely 3 tracers:

\begin{itemize}
\item $C$: concentration of micropollutants in water (Bq/m$^3$),
\item $C_s$: concentration of micropollutants adsorbed by SPM (Bq/kg) (dry weight),
\item $C_f$: concentration of micropollutants adsorbed by bottom sediments (Bq/kg).
\end{itemize}

Note: The unit of concentration chosen for the demonstration is Bq/m$^3$,
but it could also be written in kg/m$^3$ (for example, in the case of a metal).\\

The internal sources of each of these tracers correspond to the phenomena
of adsorption/desorption, deposition/re-suspension and exponential decay.
Taking these phenomena into account leads to the following equations
in each of the three compartments, water, SPM and bottom sediments
(with section \ref{waq_models} notations):

\begin{itemize}
\item dissolution phase
\begin{equation}
  F(C) = -k_{-1}.SS (K_d.C - C_s ) - L.C,
\end{equation}

\item adsorption by suspended particulate matter phase
\begin{equation}
  F(SS.C_s) = k_{-1}.SS (K_d.C - C_s ) + \frac{RS.C_f-SED.C_s}{h} - L.SS.C_s,
\end{equation}

\item adsorption by bottom sediments (tracer neither advected nor diffused)
\begin{equation}
  \frac{\partial (SF.C_f)}{\partial t} = SED.C_s - RS.C_f - L.SF.C_f.
\end{equation}

\end{itemize}

Setting the new variables as: $C_{ss}$  = $SS.C_s$, in Bq/m$^3$ of water,
$C_{ff} = SF.C_f$, in Bq/m$^2$ of bottom sediments,
and $SEDP = \frac{SED}{SS} = w \left (1-\frac{\tau_b}{\tau_s} \right)$
if $\tau_b < \tau_s$ or 0 otherwise, the equations become:\\

Tracer $\#$3: dissolution phase

\begin{equation}
  F(C) = -k_{-1}.SS.K_d.C + k_{-1}.C_{SS} - L.C.
\end{equation}

Tracer $\#$4: phase of adsorption by suspended particulate matter

\begin{equation}
  F(C_{SS}) = k_{-1}.K_d.SS.C - k_{-1}.C_{SS} + \frac{\frac{RS}{SF}C_{ff}-SEDP.C_{SS}}{h}- L.C_{SS}.
\end{equation}

Tracer $\#$5: phase of adsorption by bottom sediments

\begin{equation}
  \frac{\partial C_{ff}}{\partial t} = SEDP.C_{SS} - \frac{RS}{SF} C_{ff} - L.C_{ff}.
\end{equation}

Terms with $SF$ as denominator are nullified when $SF$ is equal to 0.\\

Therefore, there are three parameters of the micropollutant model:
the distribution coefficient at equilibrium $K_d$,
the kinetic constant of desorption $k_{-1}$,
and the exponential decay constant $L$ (radioactive decay, for example).
In total, the MICROPOL model includes 7 physical parameters.

\section{Solved equations}

Noting $C_1 = SS$, $C_2 = SF$, $C_3 =C$, $C_4 = C_{ss}$, $C_5 = C_{ff}$ ,
the matrices (5 $\times$ 5) $[\lambda]$ and $[\mu]$
%containing the coefficients $\lambda_i^j$ and $\mu_i^j$
are written as:

$$  \lambda = 
  \begin{pmatrix}
    0 & 0 & 0 & 0 & 0\\
    SEDP & 0 & 0 & 0 & 0\\
    0 & 0 & -L -k_{-1}.SS.K_d & k_{-1} & 0\\
    0 & 0 & -k_{-1}.K_d.SS & -k_{-1} - L & 0\\
    0 & 0 & 0 & SEDP & -\frac{RS}{SF}-L
  \end{pmatrix}
$$  

$$  \mu = 
  \begin{pmatrix}
    -SEDP & 0 & 0 & 0 & 0\\
    0 & 0 & 0 & 0 & 0\\
    0 & 0 & 0 & 0 & 0\\
    0 & 0 & 0 & -SEDP & \frac{RS}{SF}\\
    0 & 0 & 0 & 0 & 0
  \end{pmatrix}
$$  

Note the case where $SF$ = 0: $C_{ff} = 0$, $\lambda_5^5 = 0$ and $\mu_4^5 = 0$.
The only non-zero terms $\lambda_i^0$ and $\mu_i^0$ are $\lambda_2^0 = -RS$ and $\mu_2^0 = RS$.


\chapter{The THERMIC module}
\label{subs:therm:mod}
For a majority of water quality processes, the interaction with atmosphere is a key parameter.
The THERMIC module is activated by setting \telkey{WATER QUALITY PROCESS} = 11.
The neighboring conditions are taken into account through a meteorological file
like the one described in section \ref{subs:meteo:file}.
It is important to underline that the data contained in this file
can vary depending on the considered case.
The subroutine \telfile{meteo.f} can be edited by the user to customize it to his specific model.

The evolution of temperature of water is tightly linked to heat fluxes through the free surface.
These fluxes (in W/m$^2$) are of 5 natures:

\begin{itemize}
\item sun ray flux $RS$,
\item atmospheric radiation flux $RA$,
\item free surface radiation flux $RE$,
\item heat flux due to advection $CV$,
\item heat flux due to evaporation $CE$.
\end{itemize}

The final balance of (surface) source terms is given by:
\[S_{surf}=RS+RA-RE-CV-CE\]
We will give a brief description for each of these terms, for more details see \cite{El-Kadi2012}.
This surface source term is treated explicitly in \telemac{2d},
the following term is added in the explicit source term of advection-diffusion equation
of tracer$\frac{S_{surf}}{\rho C_pH}$.


\section{Sun ray flux RS}

Sun ray flux is simply provided in the \telkey{ASCII ATMOSPHERIC DATA FILE}.
In a majority of cases, when no measurements are available,
this flux is estimated using the method of Perrin \& Brichambaut (\cite{El-Kadi2012}),
which uses the cloud cover of the sky that varies during the day (function of time).
So far, this flux is considered constant in space.
For more real cases, the user is invited to use the ``heat exchange'' module
(in folder sources/telemac3d).
A sun ray flux varying in space, common between \telemac{2d} and \telemac{3d} will be implemented in next releases.


\section{Atmospheric radiation RA}

The atmospheric radiation $RA$ is estimated with meteorological
data collected at the ground level.
It takes into account energy exchanges with the ground, water
(and energy) exchanges with the underground, etc.
In this module, $RA$ is estimated mainly by the air temperature, like:
\begin{equation*}
RA=e_{air}\sigma\left(T_{air}+273.15 \right)^4\left(1+k\left(\frac{c}{8}\right)^2 \right),
\end{equation*}
where:
\begin{itemize}
\item $e_{air}$ is a calibrating coefficient given by the keyword
  \telkey{COEFFICIENTS FOR CALIBRATING ATMOSPHERIC RADIATION} (default 0.97),
\item $\sigma$ is the constant of Stefan-Boltzmann (= 5.67.10${}^{-8}$ Wm${}^{-2}$K${}^{-4}$),
\item $T_{air}$ is air temperature given in the \telkey{ASCII ATMOSPHERIC DATA FILE},
\item $k$ is the coefficient that represents the nature and elevation of clouds,
it has a mean value of 0.2 (keyword \telkey{COEFFICIENT OF CLOUDING RATE}).
To simplify calculations, an average value of $k$ = 0.2 is usually taken
(default value).
However, it varies like indicated in Table \ref{tab:kcloud}.
\end{itemize}

\begin{table}
  \centering
  \begin{tabular}{|l|c|}
     \hline
     Type of cloud & $k$ \\
     \hline \hline
     Cirrus & 0.04 \\
     Cirro-stratus & 0.08 \\
     Altocumulus & 0.17 \\
     Altostratus & 0.2 \\
     Cumulus & 0.2 \\
     Stratus & 0.24\\
     \hline
   \end{tabular}
  \caption{Values of $k$ depending on cloud type}\label{tab:kcloud}
\end{table}


\section{Free surface radiation RE}

The available water is assumed to behave like a grey body.
Radiation generated by this grey body through the free surface is given by:
\begin{equation*}
RE = e_{water}\sigma\left(T_{water}+273.15 \right)^4,
\end{equation*}
where:
\begin{itemize}
\item $T_{water}$ is the mean water temperature in ${}^\circ$C.
$T_{water}$ is given by the keyword \telkey{WATER TEMPERATURE} (default 7${}^\circ$C),
\item $e_{water}$ is a calibration coefficient which depends on the nature
of the site and obstacles around it.
This coefficient is given with
\telkey{COEFFICIENTS FOR CALIBRATING SURFACE WATER RADIATION} (default 0.97).
For instance, for a narrow river with lots of trees on its banks,
$e_{water}$ is around 0.97, for large rivers or lakes it is about 0.92.
\end{itemize}


\section{Advection heat flux CV}

This flux is estimated empirically:
\begin{equation*}
CV=\rho_{air}C_{p_{air}}\left(a+bV \right)\left(T_{water}-T_{air} \right),
\end{equation*}
where:
\begin{itemize}
  \item $\rho_{air}$ is the air density given by
${\rho }_{air}=\ \frac{100\ P_{atm}}{\left(T_{air}+273.15\right)287}$
where $P_{atm}$ is the atmospheric pressure,
introduced in the \telkey{ASCII ATMOSPHERIC DATA FILE} or using the keyword
\telkey{VALUE OF ATMOSPHERIC PRESSURE} (default 100,000~Pa),
this is a keyword of \telemac{2d} and \telemac{3d},
\item $C_{p_{air}}$ is the air specific heat (J/kg${}^\circ$C)
  given by \telkey{AIR SPECIFIC HEAT} (default 1,005),
\item $V$ is the wind velocity (m/s),
\item $a$, $b$ are empirical coefficients to be calibrated.
Their values are very close to 0.0025, but they can be changed
using \telkey{COEFFICIENTS OF AERATION FORMULA} (default = (0.002, 0.0012)).
\end{itemize}


\section{Evaporation heat flux CE}

 It is given by the following empirical formula:
\begin{equation*}
CE = L(T_{water})\rho_{air}\left(a+bV \right) \left(H^{sat}-H \right)
\end{equation*}
where:
\begin{itemize}
\item $L(T_{water})$ = 2,500,900 - 2,365.$T_{water}$ is the vaporization latent heat (J/Kg),
\item $H^{sat}=\frac{0.622P^{sat}_{vap}}{P_{atm}-0.378P^{sat}_{vap}}$
is the air specific moisture (humidity) at saturation (kg/kg),
\item $H = \frac{0.622P_{vap}}{P_{atm}-0.378P_{vap}}$ is the specific humidity of air (kg/kg),
\item $P_{vap}$ is the partial pressure of water vapour in the air (hPa)
which is given in the \telkey{ASCII ATMOSPHERIC DATA FILE},
\item $P^{sat}_{vap}$ is the partial pressure of water vapour at saturation (hPa) which is estimated with :
\begin{equation*}
P^{sat}_{vap} = 6.11 \exp \left(\frac{17.27T_{water}}{T_{water}+237.3} \right).
\end{equation*}
\end{itemize}

When $H{}^{sat} < H$, the atmospheric radiation $RA$ is corrected by multiplying it with 1.8.



\chapter{AED2 Module}

See the AED2 model technical manual (water quality and aquatic ecology model)
available on the AED2 website:

http://aed.see.uwa.edu.au/research/models/AED/downloads/AED\_ScienceManual\_v4\_draft.pdf


\chapter{Degradation law}

WAQTEL simulates the evolution of a tracer $C$ over time from an initial condition
according to a degradation law that is assumed to be of 1$^{\rm{st}}$ order (i.e. a tracer decrease):

\begin{equation}
  F([C]) = -k_1 [C],
\end{equation}

where $k_1$ is the constant of tracer kinetic degradation $C$ (d$^{-1}$),
to be specified by the user.


\chapter{Conclusion}

WAQTEL simulates the transport of several tracers in a river or the sea
(by resolution of the advection-diffusion equation) possibly coupled
(\textit{via} source terms of the equation).
\waqtel offers a structure that allows programming further water quality modules.\\

This technical manual first shows the method of resolving the convection-dispersion equation
and its application to water quality.\\

The water quality modules available in the WAQTEL tool library are described,
namely:

\begin{itemize}
\item O2: a simplified module for dissolved oxygen,
\item BIOMASS: a module for phytoplankton biomass,
\item EUTRO: a module for river eutrophication (dissolved oxygen and algal biomass),
\item MICROPOL: a module for heavy metals or radioelements,
  taking into account their interaction with fine sediments (suspended particulate matter),
\item THERMIC: a module for water temperature evolution under the influence of atmospheric fluxes,
\item AED2: the water quality and aquatic ecology model,
\item a degradation law.
\end{itemize}

%In the future, the following tasks are expected to be addressed:
%\begin{itemize}
%\item enabling simulation of tracer transport in a hydraulic network that involves floodways,
%\item enabling a distinction to be made between minor riverbeds and major riverbeds
%  in studies of water quality,
%\item finally, taking account of the presence of dead zones (dead/storage zones).
%  This may be useful for river applications during periods of low water. 
%\end{itemize}

%===========================================================================
% Bibliography
%===========================================================================

%\addcontentsline{toc}{section}{References}
\bibliographystyle{plainnat}
%\bibliography{latex/waqtel_theory_guide}

\bibliography{../../data/biblio}

%\printbibliography
\end{document}
