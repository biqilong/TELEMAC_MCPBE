\chapter{BIOMASS Module}

The BIOMASS module is a water quality module which allows the calculation of algal biomass.
It estimates the extent of vegetal colonization in terms of various parameters:
sunshine, water temperature, degree of fertilization, ratio of water renewal,
water turbidity and toxicity \cite{gosse_biomass_1983}.
It is activated by setting \telkey{WATER QUALITY PROCESS} = 3.

It takes into account of five tracers:

\begin{itemize}
\item phytoplankton biomass PHY,
\item the principal nutrients that influence its production (phosphorus, nitrogen)
  as well as the associated mineral forms, namely:

\begin{itemize}
\item dissolved mineral phosphorus assimilable by phytoplankton PO$_4$,
\item degradable phosphorus not assimilable by phytoplankton POR,
\item dissolved mineral nitrogen assimilable by phytoplankton NO$_3$,
\item degradable nitrogen not assimilable by phytoplankton NOR.
\end{itemize}
\end{itemize}

These variables are all expressed in mg/l except biomass which is expressed in $\mu$g(Chlorophyl a)/l.\\

The following paragraphs explain the internal source terms relating to the 5 tracers.

For more details about the theory of the O2 module,
the reader can refer to the \waqtel theory guide.


\section{Processes represented}

The bottom and the processes that occur there are not modeled in the BIOMASS model.
Deposition is only represented by the deposition flux and,
once organic matter is deposited,
it no longer appears in the equations and can no longer be resuspended.
These deposition fluxes therefore correspond to a definitive loss of mass.


\section{Phytoplankton}

\subsection{Algal growth}

The rate of algal growth $CP$ (d$^{-1}$) is given by the equation:

\begin{equation*}
  CP = C_{max} RAY g_1 LNUT \alpha_1,
\end{equation*}

with $C_{max}$ = maximum rate of algal growth at 20$^{\circ}$C,
one can set its value with the keyword \telkey{MAXIMUM ALGAL GROWTH RATE AT 20C} (default = 2).
$RAY$ represents the effect of sunshine on algal growth,
this dimensionless parameter belongs to the interval [0;1].
$g_1 = T/20$ represents the effect of temperature on algal growth.
$LNUT$ represents the effects of phosphoric and nitric nutrients on algal growth.
$\alpha_1$ = water toxicity coefficient for algae ($\alpha_1$ = 1 in the absence of toxicity),
this last value can be chosen with the 1$^{\textrm{st}}$ value of the keyword
\telkey{ALGAL TOXICITY COEFFICIENTS} (default = 1).\\
$RAY$ is calculated by the Smith formula integrated over the vertical:

\begin{equation*}
  RAY = \frac{1}{k_e h} \log \left( \frac{I_0 + \sqrt{IK^2+I_0^2} }{ I_h + \sqrt{IK^2+I_h^2} }  \right),
\end{equation*}

where $k_e$ designates the extinction coefficient of solar rays in water.
The formula to compute $k_e$ can be chosen with the keyword
\telkey{METHOD OF COMPUTATION OF RAY EXTINCTION COEFFICIENT} (default = 1~m$^{-1}$):
\begin{itemize}
  \item 1: Atkins formula, it is calculated either by the Secchi depth $Z_s$
(keyword \telkey{SECCHI DEPTH}, default = 0.9~m, if it is a constant value),
    %$k_e$ = 1.7/$Z_s$,
  \item 2: the Moss relation: $k_e$ = $k_{pe}$+$ \beta $ [PHY] if $Z_s$ is unknown,
    where $k_{pe}$ is the coefficient of vegetal turbidity without phytoplankton
    provided with the keyword \telkey{VEGETAL TURBIDITY COEFFICIENT WITHOUT PHYTO}
    (default = 0~m$^{-1}$)
    and $\beta$ the coefficient of the Moss relation ($\beta$ = 0.015).
\end{itemize}

$IK$ is a calibrating parameter in the Smith formula (W/m$^2$)
associated to the keyword \telkey{PARAMETER OF CALIBRATION OF SMITH FORMULA}
(default = 120~W/m$^2$) of an order of magnitude 100.
$I_0$ is the flux density of solar radiation on the surface
which can be set with the keyword \telkey{SUNSHINE FLUX DENSITY ON WATER SURFACE}
(default = 0~W/m$^2$)
and $I_h$ is the flux density of solar radiation at the bottom (W/m$^2$), calculated by the following formula:

\begin{equation*}
  I_h = I_0 \exp (-k_e h).
\end{equation*}

$LNUT$ is calculated by the formula:

\begin{equation*}
  LNUT = \min \left( \frac{[PO_4]}{KP+[PO_4]}, \frac{[NO_3]}{KN+[NO_3]} \right),
\end{equation*}

with $KP$ = constant of phosphate half-saturation
set with the keyword \telkey{CONSTANT OF HALF-SATURATION WITH PHOSPHATE}
(default = 0.005 mgP/l),
and $KN$ = constant of nitrate half-saturation
set with the keyword \telkey{CONSTANT OF HALF-SATURATION WITH NITROGEN}
(default = 0.03~mgN/l).\\

\subsection{Algal disappearance}

The rate of algal disappearance $DP$ (d$^{-1}$) is given by the equation:

\begin{equation*}
  DP = (RP+MP) g_2,
\end{equation*}

with $RP$ = the rate of algal biomass respiration at 20$^{\circ}$C
given by the keyword \telkey{RESPIRATION RATE OF ALGAL BIOMASS}
(default = 0.05~d$^{-1}$),
$MP$ = the rate of algal biomass disappearance at 20$^{\circ}$C (d$^{-1}$).
$g_2 = T/20$ represents the effect of temperature on algal disappearance.
$MP$ is given by the following relation:

\begin{equation*}
  MP = M_1 + M_2 [PHY] + \alpha_2,
\end{equation*}

with $M_1$ and $M_2$ = coefficients of algal mortality at 20$^{\circ}$C
which can be set with the keyword
\telkey{COEFFICIENTS OF ALGAL MORTALITY AT 20C} (default = (0.1;0.003)),
$\alpha_2$ = coefficient of water toxicity for algae,
this last value can be chosen with the 2$^{\textrm{nd}}$ value of the keyword
\telkey{ALGAL TOXICITY COEFFICIENTS} (default = 0).

\section{Nitric and phosphoric nutrients}

The following physical and biochemical parameters are used
to describe the processes influencing the evolution of nitric and phosphoric nutrients:

\begin{itemize}
\item \telkey{PROPORTION OF PHOSPHORUS WITHIN PHYTO CELLS}
  for the average proportion of phosphorus in the cells of living phytoplankton $fp$ (0.0025~mgP/$\mu$gChlA),
\item \telkey{PERCENTAGE OF PHOSPHORUS ASSIMILABLE IN DEAD PHYTO}
  for the proportion of directly assimilable phosphorus in dead phytoplankton $dtp$ (default = 0.5),
\item \telkey{RATE OF TRANSFORMATION OF POR TO PO4}
  for the rate of transformation of POR into PO$_4$ through bacterial mineralization $k_1$ (default = 0.03~d$^{-1}$),
\item \telkey{RATE OF TRANSFORMATION OF NOR TO NO3}
  for the rate of transformation of NOR into NO$_3$ through heterotrophic
  and autotrophic bacterial mineralization $k_2$ (default = 0~d$^{-1}$),
\item \telkey{PROPORTION OF NITROGEN WITHIN PHYTO CELLS}
  for the average proportion of directly assimilable nitrogen in living phytoplankton $fn$ (0.0035~mgN/$\mu$gChlA),
\item \telkey{PERCENTAGE OF NITROGEN ASSIMILABLE IN DEAD PHYTO}
  for the proportion of directly assimilable nitrogen in dead phytoplankton $dtn$ (default = 0.5),
\item $F_{POR}$: deposition flux of non-algal organic phosphorus (g/m$^2$s).
  $F_{POR} = W_{POR} [POR]$,
  $W_{POR}$ is the velocity of sedimentation of non-algal organic phosphorus
  given by the keyword \telkey{SEDIMENTATION VELOCITY OF ORGANIC PHOSPHORUS}
  (default = 0~m/s),
\item $F_{NOR}$: deposition flux of non-algal organic nitrogen (g/m$^2$s).
  $F_{NOR} = W_{NOR} [NOR]$, $W_{NOR}$ is the velocity of sedimentation of non-algal organic nitrogen
  given by the keyword \telkey{SEDIMENTATION VELOCITY OF NON ALGAL NITROGEN}
  (default = 0~m/s).
\end{itemize}
