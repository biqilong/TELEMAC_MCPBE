\chapter{Practical aspects}

To activate the water quality module \waqtel, the integer keyword
\telkey{WATER QUALITY PROCESS} must be set to a value different from 1 (default = 1)
in the \telemac{2d} or \telemac{3d} \telkey{STEERING FILE}.
In \waqtel, a new dictionary is created and fully dedicated to water quality applications.
The description of this new dictionary will be the subject of a separate manual
(see the \waqtel reference manual).
%\telemac{2d} or \telemac{3d} will not read automatically this dictionary,
%unless the keyword \telkey{WAQ DICTIONARY} is used.
%This keyword is the path to dictionary.
To introduce WAQ parameters, a separate steering file is necessary.
It is read with the use of the keyword \telkey{STEERING FILE}
and its name has to be declared in the \telemac{2d} or \telemac{3d}
\telkey{STEERING FILE}.
In the next sections which are dedicated to water quality,
all the keywords to be introduced to the water quality steering file
will be written in \telkey{THIS FONT}.

Depending on the \waqtel module, \telemac{2d} or \telemac{3d}
automatically increase the number of tracers of the model.
In fact, the total number of tracers (variable \telfile{NTRAC})
is increased by 3 (for O$_2$ process), by 5 (for BIOMASS and MICROPOL processes)
or by 8 (for EUTRO process).
For the THERMIC module, the number of tracers is increased only
if no temperature is present in the set of tracers already existing in the model.

Some general parameters can also be chosen for any water quality process.
For instance, user can give a title for the study by using \telkey{WAQ CASE TITLE}.
He/she can introduce water density (\telkey{WATER DENSITY}).
%viscosity of water (\telkey{KINEMATIC WATER VISCOSITY}), read but not used at the moment


\section{The meteo file or ASCII ATMOSPHERIC DATA FILE}
\label{subs:meteo:file}
Most of the water quality processes are tightly linked to
meteorological data on the domain.
Besides the specific treatment for wind, atmospheric pressure
and rain described in the section 6.3 of the \telemac{2d} User Manual
or in the section 5.5 of the \telemac{3d} User Manual,
water quality needs different meteorological data such as
nebulosity, air temperature, evaporation pressure etc.
Hereafter we give an example of \telkey{ASCII ATMOSPHERIC DATA FILE}
(see Table \ref{tab:meteo}).
The user can edit this file and add new parameters.
However, in this case, he must edit subroutine \telfile{meteo.f} consequently.

\begin{table}
    \centering
  \begin{tabular}{|l|c|c|c|c|c|c|c|c|}
     \hline \hline
     % after \\: \hline or \cline{col1-col2} \cline{col3-col4} ...
     Time & $T_{air}$ & $P_{vap}$ & Wind speed & Wind dir. & Nebulo. & Ray. & $P_{atm}$ & Rain \\
     \hline \hline
     s & $^\circ$C & hPa & m/s & $^\circ$ & Octa & W/m$^2$ & mbar & mm \\
     \hline \hline
     0 & 20 & 10 & 0.5 & 70 & 5 & 160 & 1012.7 & 0 \\
     3600 & 20 & 10 & 0.5 & 70 & 5 & 160 & 1012.7 & 0 \\
     7200 & 20 & 10 & 0.5 & 70 & 5 & 160 & 1012.7 & 0 \\
     \hline
   \end{tabular}
  \caption{Example of meteo data file }\label{tab:meteo}
\end{table}
