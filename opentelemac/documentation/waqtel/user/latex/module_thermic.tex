\chapter{The THERMIC module}
\label{subs:therm:mod}
For a majority of water quality processes, the interaction with atmosphere is a key parameter.
The THERMIC module is activated by setting \telkey{WATER QUALITY PROCESS} = 11.
The neighboring conditions are taken into account through a meteorological file
like the one described in section \ref{subs:meteo:file}.
It is important to underline that the data contained in this file
can vary depending on the considered case.
The subroutine \telfile{meteo.f} can be edited by the user to customize it to his specific model.

The evolution of temperature of water is tightly linked to heat fluxes through the free surface.
These fluxes (in W/m$^2$) are of 5 natures:

\begin{itemize}
\item sun ray flux $RS$,
\item atmospheric radiation flux $RA$,
\item free surface radiation flux $RE$,
\item heat flux due to advection $CV$,
\item heat flux due to evaporation $CE$.
\end{itemize}

The final balance of (surface) source terms is given by:
\[S_{surf}=RS+RA-RE-CV-CE\]
We will give a brief description for each of these terms, for more details see \cite{El-Kadi2012}.
This surface source term is treated explicitly in \telemac{2d},
the following term is added in the explicit source term of advection-diffusion equation
of tracer$\frac{S_{surf}}{\rho C_pH}$.


\section{Sun ray flux RS}

Sun ray flux is simply provided in the \telkey{ASCII ATMOSPHERIC DATA FILE}.
In a majority of cases, when no measurements are available,
this flux is estimated using the method of Perrin \& Brichambaut (\cite{El-Kadi2012}),
which uses the cloud cover of the sky that varies during the day (function of time).
So far, this flux is considered constant in space.
For more real cases, the user is invited to use the ``heat exchange'' module
(in folder sources/telemac3d).
A sun ray flux varying in space, common between \telemac{2d} and \telemac{3d} will be implemented in next releases.


\section{Atmospheric radiation RA}

The atmospheric radiation $RA$ is estimated with meteorological
data collected at the ground level.
It takes into account energy exchanges with the ground, water
(and energy) exchanges with the underground, etc.
In this module, $RA$ is estimated mainly by the air temperature, like:
\begin{equation*}
RA=e_{air}\sigma\left(T_{air}+273.15 \right)^4\left(1+k\left(\frac{c}{8}\right)^2 \right),
\end{equation*}
where:
\begin{itemize}
\item $e_{air}$ is a calibrating coefficient given by the keyword
  \telkey{COEFFICIENTS FOR CALIBRATING ATMOSPHERIC RADIATION} (default 0.97),
\item $\sigma$ is the constant of Stefan-Boltzmann (= 5.67.10${}^{-8}$ Wm${}^{-2}$K${}^{-4}$),
\item $T_{air}$ is air temperature given in the \telkey{ASCII ATMOSPHERIC DATA FILE},
\item $k$ is the coefficient that represents the nature and elevation of clouds,
it has a mean value of 0.2 (keyword \telkey{COEFFICIENT OF CLOUDING RATE}).
To simplify calculations, an average value of $k$ = 0.2 is usually taken
(default value).
However, it varies like indicated in Table \ref{tab:kcloud}.
\end{itemize}

\begin{table}
  \centering
  \begin{tabular}{|l|c|}
     \hline
     Type of cloud & $k$ \\
     \hline \hline
     Cirrus & 0.04 \\
     Cirro-stratus & 0.08 \\
     Altocumulus & 0.17 \\
     Altostratus & 0.2 \\
     Cumulus & 0.2 \\
     Stratus & 0.24\\
     \hline
   \end{tabular}
  \caption{Values of $k$ depending on cloud type}\label{tab:kcloud}
\end{table}


\section{Free surface radiation RE}

The available water is assumed to behave like a grey body.
Radiation generated by this grey body through the free surface is given by:
\begin{equation*}
RE = e_{water}\sigma\left(T_{water}+273.15 \right)^4,
\end{equation*}
where:
\begin{itemize}
\item $T_{water}$ is the mean water temperature in ${}^\circ$C.
$T_{water}$ is given by the keyword \telkey{WATER TEMPERATURE} (default 7${}^\circ$C),
\item $e_{water}$ is a calibration coefficient which depends on the nature
of the site and obstacles around it.
This coefficient is given with
\telkey{COEFFICIENTS FOR CALIBRATING SURFACE WATER RADIATION} (default 0.97).
For instance, for a narrow river with lots of trees on its banks,
$e_{water}$ is around 0.97, for large rivers or lakes it is about 0.92.
\end{itemize}


\section{Advection heat flux CV}

This flux is estimated empirically:
\begin{equation*}
CV=\rho_{air}C_{p_{air}}\left(a+bV \right)\left(T_{water}-T_{air} \right),
\end{equation*}
where:
\begin{itemize}
  \item $\rho_{air}$ is the air density given by
${\rho }_{air}=\ \frac{100\ P_{atm}}{\left(T_{air}+273.15\right)287}$
where $P_{atm}$ is the atmospheric pressure,
introduced in the \telkey{ASCII ATMOSPHERIC DATA FILE} or using the keyword
\telkey{VALUE OF ATMOSPHERIC PRESSURE} (default 100,000~Pa),
this is a keyword of \telemac{2d} and \telemac{3d},
\item $C_{p_{air}}$ is the air specific heat (J/kg${}^\circ$C)
  given by \telkey{AIR SPECIFIC HEAT} (default 1,005),
\item $V$ is the wind velocity (m/s),
\item $a$, $b$ are empirical coefficients to be calibrated.
Their values are very close to 0.0025, but they can be changed
using \telkey{COEFFICIENTS OF AERATION FORMULA} (default = (0.002, 0.0012)).
\end{itemize}


\section{Evaporation heat flux CE}

 It is given by the following empirical formula:
\begin{equation*}
CE = L(T_{water})\rho_{air}\left(a+bV \right) \left(H^{sat}-H \right)
\end{equation*}
where:
\begin{itemize}
\item $L(T_{water})$ = 2,500,900 - 2,365.$T_{water}$ is the vaporization latent heat (J/Kg),
\item $H^{sat}=\frac{0.622P^{sat}_{vap}}{P_{atm}-0.378P^{sat}_{vap}}$
is the air specific moisture (humidity) at saturation (kg/kg),
\item $H = \frac{0.622P_{vap}}{P_{atm}-0.378P_{vap}}$ is the specific humidity of air (kg/kg),
\item $P_{vap}$ is the partial pressure of water vapour in the air (hPa)
which is given in the \telkey{ASCII ATMOSPHERIC DATA FILE},
\item $P^{sat}_{vap}$ is the partial pressure of water vapour at saturation (hPa) which is estimated with :
\begin{equation*}
P^{sat}_{vap} = 6.11 \exp \left(\frac{17.27T_{water}}{T_{water}+237.3} \right).
\end{equation*}
\end{itemize}

When $H{}^{sat} < H$, the atmospheric radiation $RA$ is corrected by multiplying it with 1.8.
