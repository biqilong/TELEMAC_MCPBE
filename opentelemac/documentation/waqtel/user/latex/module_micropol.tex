\chapter{MICROPOL Module}

The MICROPOL module simulates the outcome of a micropollutant (radioelement or heavy metal)
in the three compartments considered to be of major importance in a river ecosystem:
water, suspended particulate matter (S.P.M.) and the bottom sediments.

It is activated by setting \telkey{WATER QUALITY PROCESS} = 7.\\

Each of these compartments represents an homogeneous class:
S.P.M. and sediments taken into account represent the grain-size class of clays and silts
(cohesive-type fine sediments, of about < 20-25 $\mu$m diameter),
likely to attach the majority of micropollutants.\\

Thanks to its capacity for adsorption and desorption of micropollutants,
suspended particulate matter (S.P.M.) is one of the first links in the chain of contamination.
Suspended particulate matter is carried and dispersed in the mass of water
like a tracer and is also subject to the laws of sedimentary physics:
it settles in calm waters and produces bottom sediments,
and can be re-suspended by a strong flow.
Deposits cannot move. They are treated as tracers that can be neither advected
nor dispersed by the mass of water, but are likely to be re-suspended.\\

The model considers 5 tracers:

\begin{itemize}
\item suspended matter (SS),
\item bottom sediments (SF), neither advected nor dispersed,
\item dissolved form of micropollutant,
\item the fraction adsorbed by suspended particulate matter,
\item the fraction adsorbed by bottom sediments, neither advected nor dispersed.
\end{itemize}

\subsubsection{Notes, and limitations of the MICROPOL module}

\begin{itemize}
\item whether in suspension or deposited on the bottom, the matter is considered
  to be a passive tracer:
  in other words, it does not influence the flow (no feedback).
  This hypothesis involves that the depth of deposits must be negligible compared
  to the depth of water (the bed is assumed to be unmodified).
\item there is no direct adsorption/desorption of dissolved micropollutants
  on the deposited matter, only on the suspended particulate matter
  (the model assumes a preponderance of water – S.P.M. exchanges over direct water
  – bottom sediment exchanges).
  Bottom sediments only become radioactive by means of polluted S.P.M. deposition. 
\end{itemize}

\section{Suspended matter}

The model describing the evolution of S.P.M. and bottom sediments involved in MICROPOL
is a classic representation of the laws of deposition and re-suspension
of cohesive types of S.P.M., that are the laws of Krone \cite{krone_flume_1962}
and Partheniades \cite{partheniades_erosion_deposition_1965}.\\

Both processes require the knowledge of two characteristic constants:

\begin{itemize}
\item deposition occurs when shear stress on the bottom $\tau_b$,
  which varies according to the flow rate, becomes less than a threshold value $\tau_s$,
  known as the shear stress critical for sedimentation
  and which can be set with the keyword \telkey{SEDIMENTATION CRITICAL STRESS}
  (default = 5~Pa) .
  It is then assumed that the S.P.M. sediments at a constant velocity $w$
  (known as the settling velocity or velocity of sedimentation)
  with the keyword \telkey{SEDIMENT SETTLING VELOCITY}
  (default = 6.10$^{-6}$~m/s),
\item re-suspension occurs when a new threshold $\tau_r$,
  known as the shear stress critical for re-suspension, is exceeded.
  It can be set with the keyword \telkey{CRITICAL STRESS OF RESUSPENSION}
  (default = 1,000~Pa).
  Its importance is weighted by a constant $e$, the rate of erosion characteristic
  of deposited S.P.M. (also known as the Partheniades constant),
  which associated keyword is \telkey{EROSION RATE} (default = 0).
\end{itemize}

\section{Micropollutants}

The model representing the evolution of micropollutants assumes
that the transfers of micropollutants (radioelement, metal)
between the dissolved and particulate phases correspond to either
direct adsorption or ionic exchanges modeled by a reversible reaction,
of kinetic order 1.
%\cite{ciffroy_doubs_1995}.
In the case of direct adsorption, the reaction can be represented in the form of
a reversible reaction, controlled by velocities of adsorption ($k_1$ in l/g/s)
and desorption ($k_{-1}$ in s$^{-1}$)
which last associated keyword is \telkey{CONSTANT OF DESORPTION KINETIC}
(default = 2.5 10$^{-7}$~m/s).
It leads to an equilibrium, and then a distribution of micropollutant
between the dissolved and particulate phase described
by the distribution coefficient $K_d = \frac{k_1}{k_{-1}}$
(set with the keyword
\telkey{COEFFICIENT OF DISTRIBUTION}, default = 1,775~l/g).
Once adsorbed, the fixed micropollutants act like S.P.M. (deposition, re-suspension)
and can also produce areas of polluted sediment.\\

The model includes a law of exponential decay (radioactive decay type) of micropollutant
concentrations in each compartment of the modeled ecosystem,
through a constant written $L$
which can be set with the keyword
\telkey{EXPONENTIAL DESINTEGRATION CONSTANT} (default = 1.13 10$^{-7}~\textrm{s}^{-1}$).
