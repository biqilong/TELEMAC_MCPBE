\chapter{MICROPOL Module}

The MICROPOL module simulates the evolution of a micropollutant (radioelement or heavy metal)
in the three compartments considered to be of major importance in a river ecosystem:
water, Suspended Particulate Matter (SPM) and bottom material.

It is activated by setting \telkey{WATER QUALITY PROCESS} = 7.\\

Each of these compartments represents an homogeneous class:
SPM and sediments represent the grain-size class of clay and silt
(cohesive fine sediments, of diameter about less than 20 to 25 $\mu$m),
likely to attach the majority of micropollutants.\\

Due to adsorption and desorption of micropollutants,
SPM is one of the first links in the chain of contamination.
SPM is carried and dispersed in the water mass
as a tracer and is also subject to the laws of sedimentary physics:
it settles in calm waters and produces bottom sediments,
and can be re-suspended by a high flow.
Deposits cannot move. They are treated as tracers that can be neither advected
nor dispersed by the water mass, but are likely to be re-suspended.\\

The model considers 5 tracers:

\begin{itemize}
\item suspended matter (SS),
\item bottom sediments (SF), neither advected nor dispersed,
\item dissolved form of micropollutant,
\item the fraction adsorbed by suspended particulate matter,
\item the fraction adsorbed by bottom sediments, neither advected nor dispersed.
\end{itemize}

\subsubsection{Notes, and limitations of the MICROPOL module}

\begin{itemize}
\item whether in suspension or deposited on the bottom, the matter is considered
  to be a passive tracer:
  in other words, it does not influence the flow (no feedback).
  This hypothesis involves that the deposits depth must be negligible compared
  to the water depth (the bed is assumed to be unmodified).
\item there is no direct adsorption/desorption of dissolved micropollutants
  on the deposited matter, only on the SPM
  (the model assumes a preponderance of water – SPM exchanges over direct water
  – bottom sediment exchanges).
  Bottom sediments only become radioactive by means of polluted SPM deposition. 
\end{itemize}

\section{Suspended matter}

The model describing the evolution of SPM and bottom sediments involved in MICROPOL
is a classic representation of the deposition laws and re-suspension
of cohesive SPM, that are the laws of Krone \cite{krone_flume_1962}
and Partheniades \cite{partheniades_erosion_deposition_1965}.\\

Both processes require the knowledge of characteristic constants:

\begin{itemize}
\item deposition occurs when bottom shear stress $\tau_b$,
  which varies according to the flow conditions, becomes lower than a threshold value $\tau_s$,
  known as the critical shear stress for sedimentation
  and which can be set with the keyword \telkey{SEDIMENTATION CRITICAL STRESS}
  (default = 5~Pa) .
  It is then assumed that the SPM settles at a constant velocity $w$
  (known as the settling velocity or velocity of sedimentation)
  with the keyword \telkey{SEDIMENT SETTLING VELOCITY}
  (default = 6.10$^{-6}$~m/s),
\item re-suspension occurs when a threshold $\tau_r$,
  known as the critical shear stress for re-suspension, is exceeded.
  It can be set with the keyword \telkey{CRITICAL STRESS OF RESUSPENSION}
  (default = 1,000~Pa).
  Its importance is weighted by a constant $e$, the rate of erosion characteristic
  of deposited SPM (also known as the Partheniades constant),
  which associated keyword is \telkey{EROSION RATE} (default = 0).
\end{itemize}

\section{Micropollutants}

The model representing the evolution of micropollutants assumes
that the transfers of micropollutants (radioelement, metal)
between the dissolved and particulate phases correspond to either
direct adsorption or ionic exchanges modeled by a reversible reaction,
of 1$^{\rm{st}}$ kinetic order.
%\cite{ciffroy_doubs_1995}.
In the case of direct adsorption, the reaction can be represented in the form of
a reversible reaction, controlled by adsorption ($k_1$ in l/g/s)
and desorption velocities ($k_{-1}$ in s$^{-1}$)
which last associated keyword is \telkey{CONSTANT OF DESORPTION KINETIC}
(default = 2.5 10$^{-7}$~m/s).
It leads to an equilibrium state, and then a distribution of micropollutants
between the dissolved and particulate phase described
by the distribution coefficient $K_d = \frac{k_1}{k_{-1}}$
(set with the keyword
\telkey{COEFFICIENT OF DISTRIBUTION}, default = 1,775~l/g).
Once adsorbed, the fixed micropollutants act like SPM (deposition, re-suspension)
and can also produce areas of polluted sediment.\\

The model includes an exponential decay law (radioactive decay type) of micropollutant
concentrations in each compartment of the modeled ecosystem,
through a constant written $L$
which can be set with the keyword
\telkey{EXPONENTIAL DESINTEGRATION CONSTANT} (default = 1.13 10$^{-7}~\textrm{s}^{-1}$).
