In this chapter, we describe the time-scheme used in TELEMAC-3D in 
a continuous-space framework. The aim is to help the reader understand
what comes in the chapter \ref{Chapter4}, where the space-time discretisation
is described\footnote{It is important to bear in mind that demonstrations 
of conservation can only be done with the full space-time discretisation.}.
In TELEMAC-3D, a fractional-steps method is used for the time discretisation.
An important aspect of the algorithm is that the horizontal and vertical velocities are
not treated in the same way. Moreover, there are two definitions of the velocity field
in the algorithm: the one calculated on the basis of the continuity equation so as to ensure
mass conservation and the one calculated on the basis of the momentum equation. Later on,
we refer to them as the conservative velocity field (denoted by $\vec{u}^C$) and the momentum
velocity field (denoted by $\vec{u}$), respectively. 
We recall that the 2D vector corresponding to the $x$ and $y$
components of a 3D vector is denoted with a $2D$ subscript. 
On the other hand, we denote with an $A$ superscript the fields obtained after an advection step.
%while we denote with a $D$ superscript the fields obtained after an advection-diffusion step.
The algorithm then consists of several steps, which order depends on the
option \texttt{DYNAMIC PRESSURE IN WAVE EQUATION}. 
In TELEMAC-3D, we call wave equation the equation solved to calculate
the new water depth values. It is indeed an equation of wave propagation
since we do not solve the coupled system linking $h^{n+1}$ and $\vec{u}^{Cn+1}$,
but rather write $\vec{u}^{Cn+1}$ as a function of $h^{n+1}$ and introduce
this formula into the equation on $h^{n+1}$. There are two options in TELEMAC-3D:
either the dynamic pressure is calculated after the values of $h^{n+1}$ and
$\vec{u}^{Cn+1}$ are known, or it is calculated once $\vec{u}^{Cn+1}$ has been written 
as a function of $h^{n+1}$ based on an intermediate velocity, which is
thereafter corrected through a projection step, and then only $h^{n+1}$ is
calculated. We list the different steps below, for each value of the key-word.

\section{Dynamic pressure calculated after the wave equation 
resolution}
This corresponds to the option \texttt{DYNAMIC PRESSURE IN WAVE EQUATION=NO}, 
which is the one by default in TELEMAC-3D. The algorithm then consists of 
the following steps:
\begin{enumerate}
\item \textbf{an advection step for the momentum horizontal velocities}, 
using the conservative velocity for the advection:
\begin{equation}
    \dfrac{\vec{u}_{2D}^{A}-\vec{u}_{2D}^n}{\delta t} 
    + \left(\vec{u}^{Cn}\cdot\Grad\right)\vec{u}_{2D}^{*}=0 \medskip \\
\end{equation}
\begin{CommentBlock}{Remarks about the advection term $\left(\vec{u}^{C}\cdot\Grad\right)\vec{u}_{2D}^{*}$:}
\begin{itemize}
\item $\vec{u}^{Cn}$ is used as the advecting velocity because it is 
specifically computed so as to be divergence-free at step 5. 
When using a conservative advection scheme, this makes it possible to ensure mass-conservation 
at machine precision, with any discretisation.
It is a very important feature of TELEMAC-3D, especially for dilution studies where scalar conservation is
of primary importance\footnote{In the studies, the meshes are usually quite coarse,
on the one hand because bathymetric data have a limited accuracy, but also to limit
computational times, so that this is really a key-feature.}.
\item The value of $\vec{u}^{C}$ from the current time-step, $\vec{u}^{Cn}$, is used for the advection of the velocity,
while the value of the next time-step, $\vec{u}^{Cn+1}$, is calculated when solving the wave equation.
However, for the sake of simplicity in the notations, in the next chapter about space-time discretisation
we will not write $\vec{u}^{Cn}$ or $\vec{u}^{Cn+1}$, but only $\vec{u}^{C}$ to have lighter notations.
\item the time-discretisation of the advected field $\vec{u}_{2D}$ is not specified yet, which is accounted for by the $*$ superscript. 
Indeed, in TELEMAC-3D there are various methods for the computation
of the advection term: the methods of characteristics, distributive schemes or finite elements can be used.
The basic forms of distributive schemes -- N, PSI, are explicit, while more complex forms can be semi-implicited
in time. On the other hand, the SUPG scheme is implicit and the characteristics method consists of an exact
time-integration between two time-steps. This is dealt with in the section \ref{sec:advection_schemes}.
\end{itemize}
\end{CommentBlock}

\item \textbf{an advection-diffusion step for the momentum vertical velocity}, 
using the conservative velocity field for the advection and the momentum velocity for the diffusion:
\begin{equation}
    \dfrac{\tilde{w}^{aux}-w^n}{\delta t}+
    \left(\vec{u}^{Cn}\cdot\Grad\right)w^{*}=
    -g+ \tilde{F}_z+\Div(\nu_Ew^*\cdot\vec{e}_z)
\end{equation}
\begin{CommentBlock}{Remark:}
In this system, the term $\tilde{F}_z$ contains the buoyancy forces, 
Coriolis force, tidal force, source points inputs and outputs 
(isolated or linked through the culvert fomulation), rain and evaporation. 
If present, these terms are treated explicitly.
\end{CommentBlock}
Then, the diffusion term can be partially or fully implicited using 
a coefficient of implicitation for the diffusion $\theta_d$, 
so that this step reads:
\begin{equation}
    \dfrac{\tilde{w}^{aux}-w^n}{\delta t}+
    \left(\vec{u}^{Cn}\cdot\Grad\right)w^{*}=
    -g+ \tilde{F}_z+\Div\left(\nu_E\theta_{d}\Grad \tilde{w}^{aux} + \nu_E(1-\theta_{d})\Grad w^n\right)
\end{equation}
\begin{CommentBlock}{Remark:}
When applying the Finite Elements space-discretisation to this step,
some diffusion boundary terms appear (friction terms), which may actually be treated
implicitely or explicitely. This is described in the section \ref{etape hydrostatique}.
\end{CommentBlock}

\item \textbf{a hydrostatic step}, which consists in computing the new depth
together with the conservative horizontal velocities:
\begin{equation}\label{eq:hydrostatic_first_system}
  \left\{\begin{array}{l}
    \dfrac{\eta^{n+1}-\eta^{n}}{\delta t} 
    + \nabla_{2D}\cdot \displaystyle{\int_b^\eta\tilde{\vec{u}}_{2D}^{n+1}~dz}=0 \medskip \\
    \dfrac{\tilde{\vec{u}}_{2D}^{n+1}-\vec{u}_{2D}^{A}}{\delta t}=-g\Grad_{2D}\eta^{n+1}
    + \tilde{\vec{F}}_{2D}
    +\BDiv (\nu_E \Grad \vec{u}_{2D}^n)\medskip \\
  \end{array}\right.
\end{equation}

This system corresponds to the lines 1, 2 and 4 of \eqref{eq:NS_FS_expanded}, excluding the advection terms
since they have already been treated in the previous fractional step.
%We defined an intermediate velocity field $\vec{u}^{aux}$ such as:
%\begin{equation}
%  \dfrac{\vec{u}_{2D}^{aux}-\vec{u}_{2D}^{A}}{\delta t}= -g\Grad_{2D}\eta^n
%     + \tilde{\vec{F}}_{2D}
%     +\nu_E \left(\theta_d\Lap \vec{u}_{2D}^{aux}+(1-\theta_d)\Lap \vec{u}_{2D}^{n}\right)
%\end{equation}
The diffusion term was written here in a fully explicit form for the sake of simplicity but
the user actually has the choice to partially implicit it using a coefficient $\theta_d$.
This will be described with more details in the section \ref{momentumvariational}.\\

By replacing $\eta$ by $h + b$ in the second line of \eqref{eq:hydrostatic_first_system}
and partially impliciting the water depth with a 
coefficient $\theta_h$, the term $\Grad_{2D}\eta^{n+1}$ of the second line is actually re-written:
\begin{equation}
\Grad_{2D}(\theta_h\eta^{n+1}+(1-\theta_h)\eta^n) = \Grad_{2D}\left(\theta_h h^{n+1}+(1-\theta_h)h^n+b\right)
\end{equation}
We used the fact that $b$ is considered constant in time here.
The user also has the choice to partially implicit the velocity field during this step, using a coefficient
$\theta_u$ in the first line of the system \eqref{eq:hydrostatic_first_system}.
The system \eqref{eq:hydrostatic_first_system} is then modified and reads:
\begin{equation}\label{eq:hydrostatic_second_system}
  \left\{\begin{array}{l}
    \dfrac{h^{n+1}-h^{n}}{\delta t} 
    + \nabla_{2D}\cdot \displaystyle{\int_b^\eta\left(\theta_u\tilde{\vec{u}}_{2D}^{n+1}+(1-\theta_u)\vec{u}_{2D}^n\right)~dz}=0 \medskip \\
    \dfrac{\tilde{\vec{u}}_{2D}^{n+1}-\vec{u}_{2D}^{A}}{\delta t}=-g \theta_h \Grad_{2D}(h^{n+1}-h^n) -g\Grad_{2D}\eta^n
     + \tilde{\vec{F}}_{2D}
     +\BDiv(\nu_E \Grad \vec{u}_{2D}^n)\medskip \\
  \end{array}\right.
\end{equation}
For the resolution of this system,
the expression of $\tilde{\vec{u}}_{2D}^{n+1}$ given by the second 
line is injected into the first line. This yields a propagation equation on $\eta^{n+1}$.
The system \eqref{eq:hydrostatic_first_system} is then modified and reads:
\begin{equation}
   \left\{\begin{array}{l}
   \begin{array}{ll}
    \dfrac{h^{n+1}-h^{n}}{\delta t}+ \nabla_{2D}\cdot \displaystyle{\int_b^\eta
      -\delta t\theta_ug \theta_h \Grad_{2D}(h^{n+1}-h^n)~dz}= &\medskip \\
     & \hspace{-7cm}-\nabla_{2D}\cdot \displaystyle{\int_b^\eta
      \left[\theta_u\vec{u}_{2D}^{aux}+(1-\theta_u)\vec{u}_{2D}^n\right]~dz} \\
  \end{array}\medskip\\
  \dfrac{\vec{u}_{2D}^{aux}-\vec{u}_{2D}^{A}}{\delta t}= -g\Grad_{2D}\eta^n
     + \tilde{\vec{F}}_{2D}
     +\BDiv\left(\nu_E \theta_d\Grad \vec{u}_{2D}^{aux}+\nu_E(1-\theta_d)\Grad \vec{u}_{2D}^{n}\right)\medskip\\
    \tilde{\vec{u}}_{2D}^{n+1}=\vec{u}_{2D}^{aux}-g \theta_h \delta t\Grad_{2D}(h^{n+1}-h^n)
        \end{array}\right.
\label{eq:wave_equation_time}
\end{equation}
where we display the scheme using $\theta_d$ for partial implicitation of the diffusion.
Solving the first line of this system provides the value of $h^{n+1}$. 
The field $\vec{u}_{2D}^{Cn+1}$ to be used for the advection at the next time-step is then defined by:
\begin{equation}
\vec{u}_{2D}^{Cn+1}=\theta_u\vec{u}_{2D}^{aux}+(1-\theta_u)\vec{u}_{2D}^n
\label{eq:uc}
\end{equation}
so as to be consistent with the conservation of the scalars (see the section
\ref{masse traceur}).
\item \textbf{a step of mesh motion} where the new layers elevations are 
calulcated based on the new values of water depth.
\item \textbf{a pressure step} for computing the dynamic pressure and calculating the momentum velocity field:
\begin{equation}
  \left\{\begin{array}{l}
    \Lap p_d^{n+1} = \dfrac{\rho}{\delta t}\Div \tilde{\vec{u}}^{n+1} \medskip \\
    \dfrac{\vec{u}^{n+1}-\tilde{\vec{u}}^{n+1}}{\delta t} = -\dfrac{1}{\rho}\Grad p_d^{n+1}
  \end{array}\right.
\end{equation}
where $\tilde{\vec{u}}^{n+1}$ is defined by:
%denotes the non divergence-free vector of components $(u^C, v^C, w^C)$.
\begin{equation}
\tilde{\vec{u}}^{n+1}=\vec{u}_{2D}^{aux}-\delta tg\theta_h\Grad_{2D}(h^{n+1}-h^n)+\tilde{w}^{aux}\vec{e}_z
\end{equation}

\begin{CommentBlock}{Remark:}
This step corresponds to the resolution of the Navier--Stokes equations using a Chorin-Temam type 
projection method (see the section \ref{sec:Chorin}) where $\tilde{\vec{u}}^{n+1}$ is the predicted velocity field.
\end{CommentBlock}

\item a step for the \textbf{computation of the conservative vertical velocity}:
\begin{equation}
  \dfrac{\partial w^{Cn+1}}{\partial z}=-\dfrac{\partial u^{Cn+1}}{\partial x}-\dfrac{\partial v^{Cn+1}}{\partial y}, ~w^{C}|_b=0
\end{equation}
\item \textbf{a final advection-diffusion step for scalars} 
(including $k$ and $\epsilon$ when the $k-\epsilon$ turbulence model is used), 
using the conservative velocity field for the advection:
\begin{equation}\label{eq:keps_discreteTime}
  \left\{
    \begin{array}{l}
      \dfrac{k^{n+1}-k^n}{\delta t} +\vec{u}^{Cn+1}\cdot\Grad k^{*}= \mathbb{P}^n + \mathbb{G}^n - \epsilon^n \dfrac{k^{n+1}}{k^n} + \Div(\nu_{k}^n\theta_d\Grad k^{n+1}+\nu_{k}^n(1-\theta_d)\Grad k^{n}) \smallskip \\
      \dfrac{\epsilon^{n+1}-\epsilon^n}{\delta t} +\vec{u}^C\cdot\Grad \epsilon^{*}= \dfrac{\epsilon^n}{k^n}\left(C_{\epsilon_1}\mathbb{P}^n
        +C_{\epsilon_3}\mathbb{G}^n-C_{\epsilon_2,Y}^n\epsilon^{n+1}\right)+\Div(\nu_{\epsilon}^n\theta_d\Grad \epsilon^{n+1}+\nu_{\epsilon}^n(1-\theta_d)\Grad \epsilon^n)
    \end{array}
  \right.
\end{equation}
See the section \ref{sec:kepsequations} for the definitions of $\mathbb{P}$, $\mathbb{G}$, $\mu_{k}$, $\mu_{\epsilon}$, etc. 
$\nu_T^{n+1}$ is then calculated through the equation \eqref{eq:nut} 
and for remaining scalars (temperature, salinity, etc), the following time-discretised equation is solved:
\begin{equation}
  \dfrac{T^{n+1}-T^n}{\delta t}+\vec{u}^{Cn+1}\cdot\Grad T^{*} = F_{source}+\Div(K_T^{n+1}\theta_d\Grad T^{n+1}+K_T^{n+1}(1-\theta_d)\Grad T^n)
\end{equation}

%\textcolor{red}{Termes source explicites / implicites ??} 

\begin{CommentBlock}{Hydrostatic model}
In TELEMAC-3D, it is possible to solve the Navier--Stokes equations with the assumption 
of hydrostatic pressure: the pressure step (step 5) is then entirely skipped.
However, we highly recommend the use the non-hydrostatic option given the poor results
obtained with the hydrostatic option on some cases, in particular when active scalars have a significant
effect on the flow or for wave simulations.
\end{CommentBlock}
\end{enumerate}

\section{Dynamic pressure calculated during the wave equation
resolution}\label{sec:dynamicpressureinwaveequation}
This corresponds to the option \texttt{DYNAMIC PRESSURE IN WAVE EQUATION=YES}.
The algorithm is then modified compared to the one above:
\begin{enumerate}
\item \textbf{the advection step for the momentum horizontal velocities} is unchanged;
\item \textbf{the advection-diffusion step for the momentum vertical velocity}
is also unchanged;
\item \textbf{the hydrostatic step}, is modified. The equations to be solved
now read:
\begin{equation}\label{eq:hydrostatic_third_system}
  \left\{\begin{array}{l}
    \dfrac{h^{n+1}-h^{n}}{\delta t} 
    + \nabla_{2D}\cdot \displaystyle{\int_b^\eta\left(\theta_u\vec{u}_{2D}^{n+1}+(1-\theta_u)\vec{u}_{2D}^n\right)~dz}=0 \medskip \\
    \begin{array}{ll}\dfrac{\vec{u}_{2D}^{n+1}-\vec{u}_{2D}^{n}}{\delta t}&=-g \theta_h \Grad_{2D}(h^{n+1}-h^n) -g\Grad_{2D}\eta^n
     + \tilde{\vec{F}}_{2D}\medskip \\
     &+\BDiv(\nu_E \theta_d\Grad \tilde{\vec{u}}_{2D}^{aux}+\nu_E(1-\theta_d)\Grad \vec{u}_{2D}^{n})
     -\dfrac{1}{\rho}\Grad p_d^n
  \end{array}
\end{array}\right.
\end{equation}
First, a predicted velocity field $\tilde{\vec{u}}^{aux}$ is calculated:
\begin{equation}
  \dfrac{\tilde{\vec{u}}_{2D}^{aux}-\vec{u}_{2D}^{A}}{\delta t}= -g\Grad_{2D}\eta^n
     + \tilde{\vec{F}}_{2D}
     +\BDiv(\nu_E \theta_d\Grad \tilde{\vec{u}}_{2D}^{aux}+\nu_E(1-\theta_d)\Grad \vec{u}_{2D}^{n})
\end{equation}
The projection method is then applied so as to calculate the dynamic pressure 
at this stage and to find a divergence-free velocity field $\vec{u}^{aux}$:
\begin{equation}
  \left\{\begin{array}{l}
    \Lap p_d^{n+1} = \dfrac{\rho}{\delta t}\Div \tilde{\vec{u}}^{aux} \medskip \\
    \dfrac{\vec{u}^{aux}-\tilde{\vec{u}}^{aux}}{\delta t} = -\dfrac{1}{\rho}\Grad p_d^{n+1}
  \end{array}\right.
\end{equation}

The expression of $\vec{u}_{2D}^{aux}$ is then used to replace $\vec{u}_{2D}^{n+1}$
in the first line of \eqref{eq:hydrostatic_third_system}:
\begin{equation}
  \begin{array}{ll}
    \dfrac{h^{n+1}-h^{n}}{\delta t}+ \nabla_{2D}\cdot \displaystyle{\int_b^\eta
      -\delta t\theta_u
            g \theta_h \Grad_{2D}(h^{n+1}-h^n) 
      ~dz}= &\medskip \\

     & \hspace{-7cm}-\nabla_{2D}\cdot \displaystyle{\int_b^\eta
      \left[\theta_u\vec{u}_{2D}^{aux}+(1-\theta_u)\vec{u}_{2D}^n\right]~dz}
  \end{array}
\label{eq:wave_equation_time2}
\end{equation}
Solving this system provides the value of $h^{n+1}$. The value of $\vec{u}_{2D}^{Cn+1}$ is then obtained through:
\begin{equation}
\vec{u}_{2D}^{Cn+1}=\theta_u\vec{u}_{2D}^{aux}+(1-\theta_u)\vec{u}_{2D}^n
\end{equation}
%by solving the second line of \eqref{eq:hydrostatic_second_system}.

\item \textbf{a step of mesh motion}, which is unchanged;

\item a step for the \textbf{computation of the conservative vertical velocity},
which is unchanged;
\item \textbf{a final advection-diffusion step for scalars}, also unchanged.
\end{enumerate}

