\chapter{Some recommendations for TELEMAC-3D simulations}

If it is not too bad for you to increase your computational time,
here are some recommendations about the options to use.

\section{Advection}
\begin{itemize}
\item use the predictor-corrector distributive scheme with local implicitation for tidal flats,
based on the PSI scheme, for the velocity and all tracers, including $k$ and $\epsilon$
if you have them:\\
  / PSI scheme\\
  \telkey{SCHEME FOR ADVECTION OF VELOCITIES} = 5\\
  \telkey{SCHEME FOR ADVECTION OF K-EPSILON} = 5\\
  \telkey{SCHEME FOR ADVECTION OF TRACERS} = 5\\
  / predictor-corrector with local implicitation implicite\\
  \telkey{SCHEME OPTION FOR ADVECTION OF VELOCITIES} = 4\\
  \telkey{SCHEME OPTION FOR ADVECTION OF K-EPSILON} = 4\\
  \telkey{SCHEME OPTION FOR ADVECTION OF TRACERS} = 4\\
These are the new default values since release 8.1.\\
\item without tidal flats, use the predictor-corrector without tidal flats treatment to gain some time\\
  / Predictor-corrector without tidal flats treatment\\
  \telkey{SCHEME OPTION FOR ADVECTION OF VELOCITIES} = 2\\
  \telkey{SCHEME OPTION FOR ADVECTION OF K-EPSILON} = 2\\
  \telkey{SCHEME OPTION FOR ADVECTION OF TRACERS} = 2\\
\end{itemize}
\section{Friction}
\begin{itemize}
\item use the Nikuradse law, which is the only one that is really 3D compatible:
the other options have been implemented to ensure an easy correspondance between 2D and 3D simulations,
but their use is not recommended for fully 3D studies:\\
  \telkey{LAW OF BOTTOM FRICTION} = 5,\\
\item use a friction coefficient that is representative of the physics of your case:
with the Nikuradse law, the friction coefficient corresponds to the characteristic height of the roughness \\
  \telkey{FRICTION COEFFICIENT FOR THE BOTTOM} = 3 $\times$ d90.\\
  For a natural bed with sediment of characteristic size d90 and without riddles or dunes,
its value should be taken from van Rijn, equal to 3 $\times$ d90.
In the presence of dunes, van Rijn proposes another formula,
\item depending on the nature of the lateral walls, either use a Nikuradse law or no friction \\
  \telkey{LAW OF FRICTION ON LATERAL BOUNDARIES} = 0 or 5,
\item as for the bed, use a physical value for \telkey{FRICTION COEFFICIENT FOR LATERAL SOLID BOUNDARIES},
\item remark: contrary to the friction coefficients used with the laws derived
from the shallow water context (Manning, Strickler, Chézy, Haaland),
the friction coefficient with the Nikuradse law does not "hide" physical effects
like diffusion-dispersion, like it does with the shallow-water equations.
Thus, it is less prone to being used as a calibration parameter in the studies.
\end{itemize}

\section{Mass-lumping}
\begin{itemize}
\item avoid using mass-lumping, unless you really have a strong constraint on computational time
(this is the default behaviour of TELEMAC-3D): it introduces artificial dispersion in the results.
\end{itemize}

\section{Semi-implicitation}
\begin{itemize}
\item do not change the default value for\\
  \telkey{IMPLICITATION FOR DEPTH} ( = 0.55)\\
  and use \\
  \telkey{IMPLICITATION FOR VELOCITIES} = 0.55, new default value since release 8.1. \\
%instead of the default value of 1.
Warning: never use a value lower or equal to 0.5 for these coefficients,
otherwise the scheme becomes unconditionnally unstable. Avoid using the value 1 too, we have observed a
strange behaviour of the scheme in this case: it seems to introduce quite a lot of spurious diffusion,
even though 0.99 does not... We should perform more testing and analysis to better understand this behaviour.
\item the keyword \telkey{IMPLICITATION FOR DIFFUSION} can be left to its value by default, 1.
\end{itemize}

\section{Linear solvers}
\begin{itemize}
\item use the GMRES solver (= 7) for the matrices inversion, except diffusion
(\telkey{SOLVER FOR PPE, SOLVER FOR PROPAGATION}).
For the diffusion, the conjugate gradients should be good enough (solver number = 1).
These are the new default values since release 8.1,
\item use an accuracy equal to $10^{-8}$ or lower for all solvers
(\telkey{ACCURACY FOR PPE, ACCURACY FOR PROPAGATION, ACCURACY FOR DIFFUSION OF VELOCITIES,
ACCURACY FOR DIFFUSION OF K-EPSILON, ACCURACY FOR DIFFUSION OF TRACERS, ACCURACY FOR DIFFUSION OF SEDIMENT}).
These are the new default values since release 8.1,
\item be careful to always check in your listing that your solvers have actually converged,
otherwise the simulation may complete but with wrong results.
In case you do not reach the accuracy you asked for with these options,
try increasing the value of \telkey{OPTION OF SOLVER FOR [...]} to 10.
It sets the size of the Krylov subspaces used in the GMRES solver.
You can also try using conjugate gradients or Bi-CGSTAB,
or increasing the maximum number of iterations of the solvers,
or try using the preconditionning number 34 to make the matrix inversions faster.
\end{itemize}

\section{Fractional steps method}
\begin{itemize}
\item the keyword \telkey{DYNAMIC PRESSURE IN WAVE EQUATION} can be left to its value by default NO,
but setting it to YES may slightly reduce the numerical diffusion
(altough we need to test it further to really characterize the behaviour of these two choices).
\end{itemize}

\section{Free surface}
\begin{itemize}
\item if you observe free-surface instabilities (wiggles) in your results,
you can try using a lower value for the\\
\telkey{FREE SURFACE GRADIENT COMPATIBILITY}\\
than the one by default, for example 0.9.
\end{itemize}

\section{Liquid boundaries}
\begin{itemize}
\item we would suggest trying not to use the Thompson boundary conditions,
because they are actually only valid in the shallow water context.
However, they may help stabilise the simulations involving liquid boundaries,
so it is not always that easy not to use them.
You should try without it before trying to use it.
\end{itemize}

\section{Tidal flats}
\begin{itemize}
\item use \telkey{TREATMENT OF NEGATIVE DEPTHS} = 2 if possible,
\item use \telkey{TREATMENT ON TIDAL FLATS FOR TRACERS} = 1 to ensure conservation.
\end{itemize}

\section{Turbulence}
\begin{itemize}
\item trying to use the $k$-$\epsilon$ model (= 3) on the vertical and the horizontal directions,
or the Spalart-Allmaras one (= 5), also in both directions,
instead of the mixing length model which is not flexible, since it is a zero-equation model.
However, for stratification simulations, the mixing length model
may provide better results than any other (the $k-\epsilon$ model tends to mix too much).
\end{itemize}

\section{Other numerical options}
\begin{itemize}
\item do not change the matrix storage, simply do not include it from the steering file.
\end{itemize}
