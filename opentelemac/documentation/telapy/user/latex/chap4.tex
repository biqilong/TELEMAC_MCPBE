\chapter{Developer manual}
\label{ch:hydrod:sim}

This section will describe firstly how the API is implemented in \fortran.
The first section will describe the instance type used in the API\@. This
structure shows all variables accessible by the \fortran API\@. Then, a
section is dedicated to guidelines and advices to allow a user to add access to
a new variable from \telemac{2D} to the API\@.
Then the second part is focused on the Python guidelines and convention
of the TelApy module.
%
\section{Instance}
%
As already explained in section~\ref{ch:TelApy_description}, in order to
control the data used by \telemac{2D} a \fortran Structure called "instance"
containing all global variables of \telemac{2D} is used. A part of this
strucrure is presented below.  In addition, some functions, described below,
are available in order to handle that structure. All the instance functions can
be found in the module
\verb!m_instance_t2d!.

\begin{lstlisting}[language=Fortran]
type instance_t2d
   ! run position
   integer myposition
   ! list of all the variable for model
   type(bief_obj), pointer :: hbor
   type(bief_obj), pointer :: ubor
   type(bief_obj), pointer :: vbor
   type(bief_obj), pointer :: u
   type(bief_obj), pointer :: v
   type(bief_obj), pointer :: chestr
   double precision, pointer :: flux_boundaries(:)
   double precision, pointer :: cote(:)
   double precision, pointer :: debit(:)
!
   type(bief_mesh), pointer :: mesh
!
   type(bief_obj), pointer :: lihbor
   type(bief_obj), pointer :: liubor
   type(bief_obj), pointer :: livbor
   type(bief_obj), pointer :: numliq
!
   integer,        pointer :: nit
   integer,        pointer :: lt
!
   type(bief_file), pointer :: t2d_files(:)
   integer :: maxlu_t2d
   integer :: maxkey
   integer, pointer :: t2dres
   integer, pointer :: t2dgeo
   integer, pointer :: t2dcli
!
   character(len=144), pointer :: coupling
!
   type(bief_obj), pointer :: te5
   type(bief_obj), pointer :: zf
   type(bief_obj), pointer :: h
   type(bief_obj), pointer :: dh
!
   integer, pointer :: debug
!
   end type ! model_t2d
\end{lstlisting}

During the use of the API, two arrays are available in order to keep track the
used instances.
\begin{lstlisting}
INTEGER, PARAMETER :: MAX_INSTANCES=10
TYPE(INSTANCE_T2D), POINTER :: INSTANCE_LIST(:)
LOGICAL, ALLOCATABLE :: USED_INSTANCE(:)
\end{lstlisting}

Where \verb!INSTANCE_LIST! will contain all the instances used and
\verb!USED_INSTANCE! tells you if the id is used.

\subsection{Instance functions}

In addition to the instance definition, the API includes all routines needed to
manipulate it:

\begin{itemize}
\item \verb!CHECK_INSTANCE_T2D!. This function just checks that the id number
  is valid (between 1 and max\_instances).
\item \verb!CREATE_INSTANCE_T2D!. This function creates new instance and
  returns the id of that instance.
\item \verb!DELETE_INSTANCE_T2D!. This function deletes the instance and make
  the id available.
\end{itemize}
%
\subsection{How to add access to a new variable}
\label{var_add}
%
In order to add access to a new variable via the API, the following steps must
be done:

\begin{enumerate}
\item Get the name of the variable in \verb!declarations_telemac2d! and add ",
  TARGET" in its declaration.
\item Add that variable in the instance structure (file
  \verb!m_instance_t2d.f!).
\item Add the initialisation in \verb!create_instance_t2d!.
\item Add the variable in the get/set function that befits it.
\item Add the size of the variable in \verb!get_var_size_t2d_d!.
\item Add the type of the variable in \verb!get_var_type_t2d_d!.
\item Add the variable in \verb!GET_VAR_LIST_T2D_D!.
\item Increase the value of \verb!t2d_nb_var! in \verb!m_handle_var.f!.
\end{enumerate}

Or you can use the new script \verb!scripts/add_api_variable.sh!. That take in
input a file describing the variables to add. You can see and example in
\verb!scripts/desc_file.csv!

Then to run the script just do:
\begin{lstlisting}
./add_api_variable.sh desc_file.csv
\end{lstlisting}


The last thing to do is to add ", TARGET" in the
declarations\_\textit{module}.f file.

%
%--------------------------------------------------------------------------------
\section{Coding Convention}
%--------------------------------------------------------------------------------
%
\begin{WarningBlock}{Warning:}
\centering
Be careful, the node numbering is dependent of the convention code used.
When \fortran API are used the node numbering is considered from
$1$ to $npoin$ and in python is considered from $0$ to $(npoin-1)$
\end{WarningBlock}

\subsection{\fortran part of API}

The \fortran part of \telemacsystem API is submitted to the main rules presented
in the developer guide.

\subsection{TelApy module}

The Python part is developed with the python convention "PEP 8". The aim of
PEP 8 guidelines used in the TelApy module is to improve the readability of
code and make it consistent across the wide spectrum of Python code. A style
guide is about consistency. Consistency with this style guide is important.
Consistency within a project is more important. Consistency within one module
or function is the most important.

The PEP 8 convention coding can be easily checked using the Pylint code
analyser \url{https://www.pylint.org}. Pylint is a tool that checks for
errors in Python code, tries to enforce a coding standard and looks for code
smells. It can also look for certain type errors, it can recommend suggestions
about how particular blocks can be refactored and can offer you details about
the code’s complexity. Pylint will display a number of messages as it analyzes
the code and it can also be used for displaying some statistics about the
number of warnings and errors found in different files. The messages are
classified under various categories such as errors and warnings.

