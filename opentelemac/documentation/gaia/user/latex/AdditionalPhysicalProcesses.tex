%-------------------------------------------------------------------------------
\chapter[Additional Physical Processes]{Additional Physical Processes}
%-------------------------------------------------------------------------------

\section{Rigid beds}
Non-erodable beds are treated numerically by limiting the bed erosion and letting incoming sediment pass over. The problem of rigid beds is conceptually trivial but numerically complex.

For finite elements the minimum water depth algorithm allows a natural treatment of rigid
beds, see~\cite{Hervouet11}. The sediment is managed as a layer with a
\textcolor{blue}{\sout{depth}} \textcolor{blue}{mass} that must remain positive, and the Exner equation is solved similarly to the shallow water continuity equation in the subroutine {\ttfamily positive\_depths.f} of the \textsc{Bief} library.

The \textcolor{blue}{\sout{space location and position of the rigid bed}} \textcolor{blue}{thickness of the erodible layer} can be modified in the subroutine \textcolor{blue}{\texttt{bed\_init\_user.f}}. \textcolor{blue}{By default, the thickness of the erodible layer is 100 m, then \gaia{} computes automatically the location of the rigid bed}. \textcolor{blue}{\sout{By default, the position of the rigid bed is located at $z=-100$m.}}

\section{Tidal flats}
Tidal flats are areas of the computational domain where the water depth can become zero during the simulation. For finite elements the minimum water depth algorithm allows a natural treatment of tidal flats, see~\cite{Hervouet11} and the Exner equation is solved similarly to the shallow water continuity equation in the subroutine {\ttfamily positive\_depths.f} of the \textsc{Bief} library.

Improvements of the numerical results where wetting and drying processes are present can be achieved by using the keyword \telkey{MINIMUM DEPTH FOR BEDLOAD} (real type variable, set to {\ttfamily = 1.E-2}m by default), which cancels sediment fluxes to and from dry points.

The default value can be modified by the user. As a guideline, we can suggest a value in the range $[2-3]\times d_{50}$, being $d_{50}$ the median sediment diameter.

A complete treatment of tidal flats is given in the \telemac{2D} users manual.


%-------------------------------------------------------------------------------
\section{Morphological factor}
%-------------------------------------------------------------------------------
\textcolor{blue}{This option has not been tested in \gaia{}. Still working? Does it make sense if now the coupling period is always equal to 1?}.

The morphological factor, keyword \telkey{MORPHOLOGICAL FACTOR} (real type, set to {\ttfamily = 1.0} by default), increases the bottom change rates with a constant factor $N$. The new bed level represents a simulation period of $N$ hydrodynamic time steps. For example, using $1$ semi-diurnal tide ($\approx$12 hours) and a morphological factor of $10$ will result in an actual simulated time period of $120$ hours.

In theory, assuming that the morphodynamic changes are small compared to the hydrodynamic changes,
this approach reduces the computational effort without significant loss of model quality. Further details can be found in~\cite{Knaapen12} and references therein.

%-------------------------------------------------------------------------------
\section{Sediment Slide}
%-------------------------------------------------------------------------------
\textcolor{blue}{Not done. Work in progress.}

An iterative algorithm prevents the bed slope to become greater than the maximum friction angle ($\theta_s \approx 32^\circ-40^\circ$). A rotation of each element is then performed in order to insure: \textit{(i)} mass continuity and \textit{(ii)} bed slope $<$ friction angle: $|\Grad(z_b)| < \tan(\theta_s)$. The subroutine \texttt{maxslope.f} was recently modified to avoid stability issues by controlling the amount of sediment that slides. This quantity was set at 10\% of the quantity needed to achieve the required slopes, therefore slowing down the slope correction.\\

This option is activated by the keyword \texttt{SEDIMENT SLIDE = YES} (logical type variable, set to {\ttfamily = NO} by default). The friction angle can be modified with the keyword {\ttfamily FRICTION ANGLE OF SEDIMENT} (real type variable, {\ttfamily = 40.} by default). Further details can be found in~\cite{ElKadiAbderrezzak201675}.

%-------------------------------------------------------------------------------
\section{Mass balance considerations}
%-------------------------------------------------------------------------------
\textcolor{blue}{To be moved, I don't know yet where...}

The mass balance in \gaia{} is done in two ways. At every time step the evoluted mass per time step ({\ttfamily RMASCLA(ICLA)}) should be the same than the sum of mass over the boundaries in this time step ({\ttfamily MCUMUCLA(ICLA)}), the cumulated mass due to erosion and deposition fluxes from suspension
({\ttfamily (DEPOSITION\_FLUX(ICLA)-EROSION(ICLA))*DT}) and the mass added by nestor ({\ttfamily MASSNESTOR(ICLA)}). The relative mass error is computed by dividing the absolute mass error by the initial mass of the active layer.

For the final mass balance the evoluted masses, the boundaries masses, the erosion deposition masses and the nestor masses are cumulated over all time steps and balanced. Additionally a mass balance is done over the whole sediment body. The initial mass ({\ttfamily MASS0TOT(ICLA)}) is balanced with the final mass ({\ttfamily MASSTOT(ICLA)}), the cumulated mass due to erosion and deposition fluxes from suspension
({\ttfamily (DEPOSITION\_FLUX(ICLA)-EROSION(ICLA))*DT}) and the mass added by nestor ({\ttfamily MASSNESTOR(ICLA)}). Again a relative mass error is computed by dividing the absolute mass error by the initial mass.
In case of a multi-class model the final mass balance is done for every class and also for the sums over all classes.
