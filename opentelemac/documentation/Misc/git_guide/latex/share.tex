\chapter{Share your project}

\section{Create a central \soft{git} repository}

\soft{git} is a powerful tool to make it easier for several people to develop in one code.
The first thing to do is to create a central repository where people will share their developments.
Everyone will keep synchronised with that repository and a newcomer on the project may clone it to start working on the project.

The central repository may be cloned from an existing repository in \important{bare} mode.
It will not contain the source files and no one will be able to develop in it.
It will only be used to push and pull developments.
To clone a repository in bare mode type:
\begin{lstlisting}
  git clone --bare user@ipAddress:\$PATH/myRepository
\end{lstlisting}%$

\important{Important}: It is necessary to create such a bare repository, otherwise there will be interferences between people and developments may be lost or broken during the sharing process.

It is also possible to deposit a central repository on \soft{github}. To do so register an account on \url{http://www.github.com}.
Providing an academic email address will allow you to deposit private repositories on \soft{github} for free.
Create a new private repository in \soft{github}. Then, in your local repository set the \important{origin} to be the \soft{github} repository:
\begin{lstlisting}
  git remote add origin http://www.github.com/userName/repositoryName.git
\end{lstlisting}
Then you can push the branches you want (\textit{e.g.} master) onto the \soft{github} repository:
\begin{lstlisting}
  git push origin master
\end{lstlisting}

All the contributors to the project may then clone the central repository and begin working on their local clones in their own branches.

\section{Share developments between \soft{git} repositories}

A possibility is to use \important{master} only for stable versions of the code and have a \important{candidate} branch to share the ongoing developments to be included in the future release.
To share your branch \important{myBranch} with other developers \important{push} it on the central repository:
\begin{lstlisting}
  git push -u origin myBranch
\end{lstlisting}
The \important{-u} option establishes a tracking connection between your local branch and the remote one.
To update your local branches with all new developments that were pushed on the central repository type:
\begin{lstlisting}
  git pull origin
\end{lstlisting}
You can pull only one branch from origin if you prefer:
\begin{lstlisting}
  git pull origin master
\end{lstlisting}

You can share with other remote repositories than origin.
You just need to define a remote:
\begin{lstlisting}
  git remote add name@ip.address:\$PATH/repository remote1
\end{lstlisting}%$
Then pull any branch from \important{remote1}:
\begin{lstlisting}
  git pull remote1 branch_dev
\end{lstlisting}

Here are more details about sharing work with remote repositories:
\url{http://www.git-tower.com/learn/ebook/command-line/remote-repositories/publish-local-branch}.

\section{Share developments between \soft{git} and \soft{svn} repositories}\label{sec:gitsvn}
All the commits you do in the \soft{git} clone of the \soft{svn} branch are local to the git repository.
In order to upload your commits on the \soft{svn} branch, type :
\begin{lstlisting}
  git svn dcommit
\end{lstlisting}
This pushes the master onto the \soft{svn} branch. Before your push you can check the commits with:
\begin{lstlisting}
  git svn dcommit --dry-run
\end{lstlisting}

The main difficulty will then be to merge your developments into the trunk of the svn repository.
\soft{git} and \soft{svn} do not treat the merge operation in the same way, so that to avoid any trouble my advice is to do as follows:\\

- appart from the \soft{git} clone, create an \soft{svn} clone of the branch you're working on:\\
\begin{lstlisting}
  svn checkout http://svn.opentelemac.org/svn/opentelemac/branches/myBranch
\end{lstlisting}
- to update the svn repository with you \soft{git svn} commits, type:
\begin{lstlisting}
  svn update
\end{lstlisting}
- in this \soft{svn} clone, always perform the \important{merge} operations: \\
Launch the merge command with $rev$ being the last revision of the trunk
you are up to date with. If this is the first time you are updating it is the
revision at which your branch was created (can be found in the log given by the
"svn log" command) otherwise you can get that value using the following
command:
\begin{lstlisting}[language=bash]
  svn propget svn:mergeinfo .
\end{lstlisting}
It should return something like that:
\begin{lstlisting}[language=bash]
/branches/guppy:187-256,4145-4591
/branches/jewelpuffer:4665-4793
/branches/rainbowfish:2559-2958,4070-4614,4623-4798
/branches/salmon:138-254,272-286
/trunk:541-3423,4222-4817
\end{lstlisting}
The value you want is in the line \verb+/trunk:541-3423,4222-4817+. You need
the last digits i.e. $4817$.  Then you replace in the following command rev by
that number.
\begin{lstlisting}[language=bash]
  svn merge -r rev:HEAD http://svn.opentelemac.org/svn/opentelemac/trunk .
\end{lstlisting}
The $HEAD$ value will be automatically replaced by the latest revision of the
trunk.\\
If the merge generates any conflict you will need to resolve them.\\
When everything is resolved. You need to commit the merged version. Add
the trunk revision number to the commit message for information.
\begin{lstlisting}[language=python]
  svn commit -m ``Merged with revision rev of the trunk''
\end{lstlisting}
- once the merge is done, update the remotes/git-svn branch in your git repository:
\begin{lstlisting}[language=python]
  git svn fetch
\end{lstlisting}

\important{Remark: to keep everything organised I have an \textit{opentelemac} repository containing all the svn and git clones, organised into two subfolders: \textit{git} and \textit{svn}.
The latter contain all the git and svn clones respectively, named after the branches.}
