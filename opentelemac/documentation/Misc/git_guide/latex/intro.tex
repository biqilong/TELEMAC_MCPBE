%---------------------------------------------------------------------------
\chapter{Introduction}
%---------------------------------------------------------------------------

\soft{git} is a version control software that allows you to track changes in your project.

The principle is that you will tell \soft{git} to track changes on some files and take snapshots of these files throughout the evolution of the project by \important{commiting} your changes.
A \important{commit} is then a snapshot of your project at a given time, that tracks the changes made by one person on one or several files compared to the previous version.
Every \important{commit} is a version of your project.

\important{Important}:\\
The size of your \soft{git} repository should remain small: you may add any type of files to your repository but make sure they are not too large.
Usually only text files are tracked (source code, \LaTeX~files, etc.), not pdf, mp4, results files, etc.

Here is an overview of all the \soft{git} commands:\\
\url{https://training.github.com/kit/downloads/github-git-cheat-sheet.pdf}

%\section{Required packages}
%---------------------------------------------------------------------------
The only required package is \soft{git}, and \soft{gitk} for the user interface.

\important{The \soft{gitk} user interface is very nice to visualise all the changes made by you or by others in the branches, I highly recommend to use it.}
For a tutorial, see \url{https://lostechies.com/joshuaflanagan/2010/09/03/use-gitk-to-understand-git/} for example.
