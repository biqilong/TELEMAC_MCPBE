\chapter{Work on branches in \soft{git}}\label{sec:useGit}

\section{Create branches}
\soft{git} saves snapshots of your project when you make commits, and \important{always} attributes them to the \important{branch} on which you are located.
A branch is composed of a ``homogeneous'' set of commits: for example, you can have a branch containing all the stable versions of you project, a branch containing the future stable version,
development branches where you introduce a new feature in the code, etc.

The branch \important{master} is created by default.
We usually use it for stable versions of the code.


To add a new branch, type:
\begin{lstlisting}
git branch myBranch
\end{lstlisting}
To remove a branch, type:
\begin{lstlisting}
git branch -d myBranch
\end{lstlisting}

To tell \soft{git} you want to work on a specific branch, type:
\begin{lstlisting}
git checkout myBranch
\end{lstlisting}
Then all the subsequent commits will be done in \important{myBranch}.

\section{Develop in branches}

Once you have told \soft{git} to position you on a branch you can start to work in it.
Commit your developments very regularly.
For each sub-development a commit should be done: when you create a new function, a new variable, a new class, when you fix a bug, etc.
Don't hesitate to do a lot of commits: it will be easier for bug tracking.
A commit may only contain modifications of a few lines in one file as long as that modification has a significant impact.

You can choose to commit all the modifications that have been made in the currently tracked files by typing:
\begin{lstlisting}
git commit -a
\end{lstlisting}
Although this does not add new files you may have created. To add new files you must always type:
\begin{lstlisting}
git add fileName.ext
\end{lstlisting}

You can add for the next commit all the modifications you have made in a file by typing:
\begin{lstlisting}
git add fileName.ext
\end{lstlisting}
You can also choose to add for the next commit only part of the modifications you have made in one file.
To do so type:
\begin{lstlisting}
git add -p fileName.ext
\end{lstlisting}
Then \soft{git} asks you directly from the terminal if you want to stage ($\equiv$ to add) each part of your modifications (\soft{git} identifies them as blocks of modifications).
If you want to add it type \important{y} and enter, otherwise \important{n} and enter.
When you are ready to commit type \important{git commit}.

The commit messages are quite important, here are some tips to make better commit messages:
\url{http://robots.thoughtbot.com/5-useful-tips-for-a-better-commit-message}

Structure your commit message like this (from \url{http://git-scm.com/book/ch5-2.html}):
\begin{lstlisting}
Short (50 chars or less) summary of changes

More detailed explanatory text, if necessary.  Wrap it to about 72
characters or so.  In some contexts, the first line is treated as the
subject of an email and the rest of the text as the body.  The blank
line separating the summary from the body is critical (unless you omit
the body entirely); tools like rebase can get confused if you run the
two together.

Further paragraphs come after blank lines.

 \- Bullet points are okay, too

 \- Typically a hyphen or asterisk is used for the bullet, preceded by a
   single space, with blank lines in between, but conventions vary here
\end{lstlisting}

One useful thing is that you can navigate in the branch to find out where a bug was introduced (making a dichotomy is usually fast).
Therefore, it is important that \important{for every commit you make the code must compile without errors.}
Otherwise bug tracking is made very difficult.

\soft{git} assigns an id number to each commit.
You can see this number in \soft{gitk} or by typing:
\begin{lstlisting}
  git log
\end{lstlisting}
which lists all the commits in the current branch.

You can revert the modifications introduced in a commit by typing:
\begin{lstlisting}
git revert commitId
\end{lstlisting}

You can see the differences in a file between the current and a previous version by typing:
\begin{lstlisting}
git difftool commitID fileName.ext
\end{lstlisting}
where difftool may be \soft{vimdiff} or \soft{meld} for instance.
To set the tool used by git type:
\begin{lstlisting}
git config diff.tool meld
\end{lstlisting}


\section{Move inside or in between branches}

You can always come back to any version of your project by coming back to the corresponding commit.
The command:
\begin{lstlisting}
git log
\end{lstlisting}
shows the history of your branches, where you can see that an id was assigned to each commit.
To move to a given branch type:
\begin{lstlisting}
git checkout branchName
\end{lstlisting}
To move to a given commit type:
\begin{lstlisting}
git checkout commitId
\end{lstlisting}
where \important{commitId} is the number assigned ot the commit by \soft{git}, visible in \soft{gitk} or when typing \important{git log}.

\soft{git} will usually refuse to move if you have uncommited changes in your project.
To see the uncommited changes type:
\begin{lstlisting}
git status
\end{lstlisting}
If you want to \important{move and erase the uncommited changes} (which may be quite risky), type:
\begin{lstlisting}
git checkout -f commitId
\end{lstlisting}
If you want to temporarily save your changes so as to be able to recover them later type:
\begin{lstlisting}
git stash
\end{lstlisting}
The stash is a space for saving temporary changes that you may want to apply later.
You may now move around in the history unhindered.
You may have several sets of modifications in the stash, you can list them by typing:
\begin{lstlisting}
git stash list
\end{lstlisting}
Each stash has a number assigned to it (stash$\{0\}$, stash$\{1\}$, etc.).

To apply the changes contained in one of the stashes (here the number 3), type:
\begin{lstlisting}
git stash apply stash{3}
\end{lstlisting}
Typing only \important{git stash apply} will apply the changes contained in stash$\{0\}$.

To erase what's contained in a given stash (here the number 3), type:
\begin{lstlisting}
git stash drop stash{3}
\end{lstlisting}
Typing only \important{git stash drop} will drop the changes contained in stash$\{0\}$.

To apply and erase the changes contained in one of the stashes (here the number 3), type:
\begin{lstlisting}
git stash pop stash{3}
\end{lstlisting}
Typing only \important{git stash pop} will apply and erase the changes contained in stash$\{0\}$.


\section{Merge developments in between branches}

A very interesting feature of \soft{git} is that it allows you to apply the changes made in one branch to another one quite easily.
There are two commands for this, namely \important{git merge} and \important{git rebase}.
Here I'll mostly talk about rebasing. The two techniques are quite equivalent, but it's nicer to use only one of them for a given project (set it as a coding rule).
Let's say you have created the branch \important{myDev} and have made your development in it (the branch names may be completely different).
In the meanwhile, someone has made developments in the branch \important{master} and you now want to apply your development to \important{master}.
First of all, if \important{master} is a shared branch, you may not want to share all the commits you have done with such degree of detail as for your personal use.
Therefore, I usually follow the process below:
\begin{itemize}
\item Create a temporary branch for the rebase and checkout that branch:
  \begin{lstlisting}
    git checkout -b tempRebase
  \end{lstlisting}

\item Modify the history of commits by using an interactive rebase:
\begin{lstlisting}
git rebase -i HEAD~5
\end{lstlisting}
This command will prompt the text editor where the 5 (any number works of course) latest commits in the branch are displayed.
Yan can then decide to squash some of them (put them together), reword the commit messages, edit the content of the commits.
The way you can do it is explained in the text file that \soft{git} prompted for the interactive rebase.
If you change the position of the commit in the column it will modify the order of the commit in the branch accordingly.
Deleting a commit in the text editor also deletes it in your branch.
Note that giving a commit id for the interactive rebase also works:
\begin{lstlisting}
git rebase -i commitId
\end{lstlisting}

\item Once you cleaned the branch as you wish, rebase it on master:
\begin{lstlisting}
git rebase master
\end{lstlisting}
This will move the branch \important{tempRebase} on top of \important{master}.
In case the same file was modified at the same place in the two branches \soft{git} won't be able to automatically do the rebase and will notify a conflict in the file.
It will stop the rebase and edit the conflictual files, leaving in them both versions to let you choose what you want to keep.
Open the files in any text editor and modify them so as to remove all the conflicts.
\important{Check that the code correctly compiles after your modifications.}
Once the files are edited, you need to add them:
\begin{lstlisting}
git add conflictualFiles
\end{lstlisting}
then you can continue the rebase:
\begin{lstlisting}
git rebase --continue
\end{lstlisting}
In case you don't want to carry on with the rebase type:
\begin{lstlisting}
git rebase --abort
\end{lstlisting}

\item Once the rebase is done and you checked everything works as expected (compilation, test-cases), update master and remove the \important{tempRebase} branch:
\begin{lstlisting}
git checkout master
git reset --hard tempRebase
git branch -d tempRebase
\end{lstlisting}

\end{itemize}
