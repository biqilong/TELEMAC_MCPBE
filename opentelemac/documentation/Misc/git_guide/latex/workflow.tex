\chapter{Workflow with \soft{git}}

As a conclusion of this document I'll share here my development workflow.

\begin{itemize}

\item First make sure all your shared branches are up-to-date with origin or with the svn branch:
  \begin{lstlisting}
    git pull origin
  \end{lstlisting}
or
  \begin{lstlisting}
    git svn fetch
    git checkout master
    git merge remotes/git-svn
  \end{lstlisting}
\item Create a new local branch and start working on it (make one branch for each new feature you plan to introduce in the code):
  \begin{lstlisting}
    git checkout -b myNewBranch
  \end{lstlisting}
  Commit often, making small commits containing one change at a time:
  \begin{lstlisting}
    git add filesToAdd.ext
  \end{lstlisting}
  Use \important{git add -p} if necessary.
  \begin{lstlisting}
    git commit
  \end{lstlisting}
  \important{Make sure you have documented all your changes and that you respect the coding rules of the project.}\\
  \important{ALL THE COMMITS MUST COMPILE for debugging purposes.}
\item Once your development is finished, create a new branch for the merge with \important{master}:
  \begin{lstlisting}
    git checkout -b rebase\_branch
  \end{lstlisting}
\item Clean the branch \important{rebase\_branch} using the interactive rebase on the necessary amount of commits (\textit{e.g.} 10):
  \begin{lstlisting}
    git rebase -i HEAD~10
  \end{lstlisting}
  Follow the text editor instructions to clean the history of \important{rebase\_branch} as you need.
  The idea is that all the developers don't need to have a trace of all you did step by step in your developments.
\item Once the branch is clean, make a backup of it
  \begin{lstlisting}
    git branch backup
  \end{lstlisting}
\item Rebase the branch \important{rebase\_branch} on top of \important{master}:
  \begin{lstlisting}
    git rebase master
  \end{lstlisting}
\item Solve each conflict \soft{git} encounters during the rebase and \important{make sure that the code compiles before continuing the rebase}.
\item Once the rebase is finished, update master:
  \begin{lstlisting}
    git checkout master
  \end{lstlisting}
  \begin{lstlisting}
    git merge rebase\_branch
  \end{lstlisting}
\item \important{Once you have finished and you have checked that the compilation is working and that the test-cases run correctly}, delete the rebase\_branch and the backup branch:
  \begin{lstlisting}
    git branch -d rebase\_branch
    git branch -d backup
  \end{lstlisting}
\item When the branch \important{master} is stable, the integration process can start. With \soft{git svn}, perform the merges with the trunk regularly in the \soft{svn} directory, as described in chapter~\ref{sec:gitsvn}.
\item With \soft{git svn}, after your development was integrated in the trunk, refrech you svn branch :
To do that efficiently use the following script:
\begin{lstlisting}[language=bash]
#!/bin/bash
mybranch=path_to_branch
# Moving into the branch folder
cd \$mybranch
# Removing all local modifications execpt in the "configs"
# and "builds"folder
svn stat|grep -i ? |grep -vi configs/|grep -vi builds|sed -e "s/^?      //g"|tr '\n' ' '|xargs rm -rvf
# Copying all the stuff from the latest revision of the trunk to the branch
svn export --force http://svn.opentelemac.org/svn/opentelemac/trunk .
# Adding all the new files
svn stat|grep -i ? |grep -vi configs/|grep -vi builds|sed -e "s/^?      //g"|tr '\n' ' '|xargs svn add
# Committing the fresh version
svn commit -m "Fresh start from recision rev"
\end{lstlisting}
Where $path_to_branch$ must be replaced by the path to your branch and $rev$ by
the revision of the trunk.
%
\\
\textbf{WARNING}:
This process will completely erase any modifications you have locally on your
repository.  The "svn export" command will overwrite your local files with the
latest version of the trunk.
%
Copy the text in a file and run \verb!bash file! with $file$ the name of your
file. Replace $rev$ by the revision of the trunk.
\end{itemize}

