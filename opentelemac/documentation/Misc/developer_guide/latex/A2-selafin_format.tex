\chapter{The SELAFIN format}\label(sec:srffmt)

Note: for unclear historical reasons this format is also sometimes called SERAFIN

This is a binary file.

This format can be `SERAFIN `, for single precision storage, or `SERAFIND' for
double precision storage.  Double precision storage can be used for cleaner
restarts, but may not be understood by all post-processors.


The records are listed below. Records are given in the FORTRAN sense. It means
that every record corresponds to a FORTRAN WRITE:

1 record containing the title of the study (80 characters)

1 record containing the two integers NBV(1) and
NBV(2) (NBV(1) the number of variables, NBV(2)
with the value of 0),

NBV(1) records containing the names and units of each variable (over 32
characters),

1 record containing the integers table IPARAM (10 integers, of which only 4 are
currently being used).

If IPARAM (3) is not 0: the value corresponds to the x-coordinate of the origin
in the mesh

If IPARAM (4) is not 0: the value corresponds to the y-coordinate of the origin
in the mesh

These coordinates in metres may be used by post-processors to retrieve
geo-referenced coordinates, while the coordinates of the mesh are relative to
keep more digits.

If IPARAM (7) is not 0: the value corresponds to the number of planes on the
vertical (in prisms.)

If IPARAM (8) is not 0: the value corresponds to the number of boundary points
(in parallel).

If IPARAM (9) is not 0: the value corresponds to the number of interface points
(in parallel).

if IPARAM (10) = 1: a record containing the computation starting date in 6
integers : year, month, day, hour, minute, second

1 record containing the integers NELEM,NPOIN,NDP,1 (number of elements, number
of points, number of points per element and the value 1),

1 record containing table IKLE (integer array of dimension (NDP,NELEM) which is
the connectivity table. Beware: in \telemac{2D}, the dimensions of this array
are (NELEM,NDP)),

1 record containing table IPOBO (integer array of dimension NPOIN); the value
is 0 for an internal point, and gives the numbering of boundary points for the
others. This array is never used (its data can be retrieved by another way). In
parallel the table KNOLG is given instead, keeping track of the global numbers
of points in the original mesh.

1 record containing table X (real array of dimension NPOIN containing the
abscissas of the points),

1 record containing table Y (real array of dimension NPOIN containing the
ordinates of the points),

Next, for each time step, the following are found:
\begin{itemize}
  \item 1 record containing time T (real),
  \item NBV(1)+NBV(2) records containing the results arrays for each variable
    at time T.
\end{itemize}

